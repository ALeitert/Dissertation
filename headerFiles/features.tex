% --------------------------------
%  Header file.
%
%  Adds features to the document.
% --------------------------------

\makeatletter

% ------------------
% - - - Lists - - -

\usepackage
    [
        inline,
        shortlabels,
    ]
    {enumitem}

\setlist[1,2]
{
    topsep=2pt plus 2pt minus 2pt,
    partopsep=0pt,
    parsep=0pt,
    itemsep=2pt plus 2pt minus 2pt,
}


% ------------------
% - - - Labels - - -

% Global counter for custom labels.
\newcounter{glLabelCounter}
\newcommand{\setLabel}[1]
{%
    \refstepcounter{glLabelCounter}%
    \protected@edef\@currentlabel{#1}%
}


% -----------------
% - - - Boxes - - -

\newenvironment{spaceBox}
{%
    \trivlist%
    \item%
}
{%
    \endtrivlist%
}


\usepackage
    [
        framemethod=TikZ,
    ]
    {mdframed}

\newenvironment{colourFrame}[2]
{%
    \begin{spaceBox}
        \tikzset{external/export=false}
        \begin{mdframed}
            [
                innerleftmargin = 4pt,
                innerrightmargin = 4pt,
                innertopmargin = 4pt,
                innerbottommargin = 4pt,
                linecolor = #1,
                middlelinewidth = 0.5pt,
                backgroundcolor = #2,
                splitbottomskip = 4pt,
                splittopskip = 12pt,
            ]
}
{%
        \end{mdframed}
    \end{spaceBox}
}


% --------------------
% - - - Theorems - - -

\newcommand{\thrmNo}[1]
{%
    \thechapter.\the\value{localThrmCounter#1}%
}

\newcommand{\initThrmCounter}[1]
{%
    \newcounter{localThrmCounter#1}[chapter]%
}

\newcommand{\stepThrmCounter}[1]
{%
    \stepcounter{localThrmCounter#1}%
    \setLabel{\thrmNo{#1}}%
}

% Creates a plain theorem.
%   #1: Code name
%   #2: Display name
%   #3: Header font
%   #4: Body font
\newcommand{\createPlainTheorem}[4]
{
     \initThrmCounter{#1}
     \newenvironment{#1}[1][]
     {%
         \ifthenelse{\equal{##1}{\empty}}
         {%
            \stepThrmCounter{#1}%
            {#3#2\unskip~\thrmNo{#1}.}%
         }
         {%
            \stepThrmCounter{#1}%
            {#3#2\unskip~\thrmNo{#1}~(##1).}%
         }%
         #4\unskip \hskip 0.5em%
     }
     {}
}

% Creates a theorem with a line break after the header.
%   #1: Code name
%   #2: Display name
%   #3: Header font
%   #4: Body font
\newcommand{\createBreakTheorem}[4]
{
     \initThrmCounter{#1}
     \newenvironment{#1}[1][]
     {%
         \ifthenelse{\equal{##1}{\empty}}
         {%
            \stepThrmCounter{#1}%
            {#3#2\unskip~\thrmNo{#1}.}%
         }
         {%
            \stepThrmCounter{#1}%
            {#3#2\unskip~\thrmNo{#1}~(##1).}%
         }
         #4\hfil\par\@afterheading\vspace*{-\parskip}%
     }
     {}
}


% Theorem
\createPlainTheorem{theoremCore}
    {%
        Theorem
    }
    {%
        \rmfamily
        \bfseries
        \boldmath
    }
    {%
        \itshape
    }

\newenvironment{theorem}
{%
    \begin{spaceBox}%
        \begin{theoremCore}%
}
{%
        \end{theoremCore}%
    \end{spaceBox}%
}

% Lemma
\createPlainTheorem{lemmaCore}
    {%
        Lemma
    }
    {%
        \rmfamily
        \bfseries
        \boldmath
    }
    {%
        \itshape
    }

\newenvironment{lemma}
{%
    \begin{spaceBox}%
        \begin{lemmaCore}%
}
{%
        \end{lemmaCore}%
    \end{spaceBox}%
}

% Observation
\createPlainTheorem{observationCore}
    {%
        Observation
    }
    {
        \rmfamily
        \bfseries
        \boldmath
    }
    {%
        \itshape
    }

\newenvironment{observation}
{%
    \begin{spaceBox}%
        \begin{observationCore}%
}
{%
        \end{observationCore}%
    \end{spaceBox}%
}

% Corollary
\createPlainTheorem{corollaryCore}
    {%
        Corollary
    }
    {
        \rmfamily
        \bfseries
        \boldmath
    }
    {%
        \itshape
    }

\newenvironment{corollary}
{%
    \begin{spaceBox}%
        \begin{corollaryCore}%
}
{%
        \end{corollaryCore}%
    \end{spaceBox}%
}

% Conjecture
\createPlainTheorem{conjectureCore}
    {%
        Conjecture
    }
    {
        \rmfamily
        \bfseries
        \boldmath
    }
    {%
        \itshape
    }

\newenvironment{conjecture}
{%
    \begin{spaceBox}%
        \begin{conjectureCore}%
}
{%
        \end{conjectureCore}%
    \end{spaceBox}%
}

% Definition
\createPlainTheorem{definitionCore}
    {%
        Definition
    }
    {%
        \rmfamily
        \bfseries
        \boldmath
    }
    {%
        \itshape
    }

\newenvironment{definition}
{%
    \begin{spaceBox}%
        \begin{definitionCore}%
}
{
        \end{definitionCore}%
    \end{spaceBox}%
}

\colorlet{clToDoFrame}{clRed}
\colorlet{clToDoBack}{clLight80Red}

\newenvironment{todo}
{%
    \begin{colourFrame}
        {clToDoFrame}
        {clToDoBack}
%
    {\bfseries\rmfamily To Do.}%
    \hfil\par\@afterheading\vspace*{-\parskip}%
}
{%
    \end{colourFrame}%
}

% ----------------------------
% - - - Proofs an Claims - - -

% Counter
\newcounter{proofLevel}
\newcounter{globalProofCounter}
\newcounter{localClaimCounter}[globalProofCounter]%
\newcounter{skipQedCounter}

% QedHere
\newcommand{\qedhere}
{%
    \tag*{\qed}%
    \addtocounter{skipQedCounter}{1}%
}
% Claim
\newcommand{\claimNo}
{%
    \the\value{localClaimCounter}%
}

\newenvironment{claim*}
{%
    \trivlist\item[\hskip\labelsep\textit{Claim.}\;]%
    \itshape%
}
{%
    \endtrivlist
}

\newenvironment{claim}
{%
    \stepcounter{localClaimCounter}%
    \setLabel{\claimNo}%
    \trivlist\item[\hskip\labelsep\textit{Claim~\claimNo.}\;]%
    \itshape%
}
{%
    \endtrivlist
}

% Proof
\newenvironment{proof}[1][]
{%
    \addtocounter{proofLevel}{1}%
    \ifnum \value{proofLevel} = 1%
        \stepcounter{globalProofCounter}%
    \fi%
    \ifthenelse{\equal{#1}{\empty}}
    {%
        \trivlist\item[\hskip\labelsep\textit{Proof.}\;]%
    }
    {%
        \trivlist\item[\hskip\labelsep\textit{Proof\enspace(#1).}\;]
    }%
}
{%
    \ifnum \value{skipQedCounter} > 0%
        \addtocounter{skipQedCounter}{-1}%
    \else%
        \qed%
    \fi%
    \endtrivlist%
    \addtocounter{proofLevel}{-1}%
}


% ----------------------
% - - - Algorithms - - -

\usepackage
    [%
        resetcount,
        algochapter,
        ruled,
        vlined,
        linesnumbered,
        algo2e,
    ]
    {algorithm2e}

% Define algorithm-environment
\newenvironment{algorithm}
{%
    \begin{algorithm2e}%
}
{%
    \end{algorithm2e}%
}


\makeatother
