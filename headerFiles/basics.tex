% -------------------------------------------
%  Header file.
%
%  Contains basic definitions and packages.
% -------------------------------------------

% UTF-8 input.
\usepackage[utf8]{inputenc}

% Makes ligatures readable in the pdf.
\input{glyphtounicode}
\pdfgentounicode=1

% Font (everything but Times)
\usepackage[T1]{fontenc}
\usepackage
    [
        rm={lining,tabular},
        sf={lining,tabular},
        tt={lining,tabular,monowidth}
    ]{cfr-lm}
\usepackage[defaultsans,scale=0.90]{opensans}
\linespread{1.1}

% Fixes some spacing errors.
% Requires to use \( \) instead of $ $.
% See https://tex.stackexchange.com/a/271282
\usepackage{mathtools}
\mathtoolsset{mathic=true}


% Microtype 
\usepackage[kerning=true]{microtype}

\SetExtraKerning [ unit = space ]
{
    encoding = *
}
{
    - = {135,0},
    \textendash = {300,300}
}


% Document in English
\usepackage[english]{babel}

% AMS packages
\usepackage{amsmath}
\usepackage{amsfonts}
\usepackage{amssymb}

% Allows to include pictures
\usepackage{graphicx}

% Adds line-numbers.
\usepackage{lineno}
\modulolinenumbers[5]

% Links and metadata for pdf-file
\usepackage
    [
        pdfusetitle,
    ]
    {hyperref}

\hypersetup
{
    unicode,
    pdfborder = 0 0 0,
}


% More if-clauses
\usepackage[]{xifthen}

% Easy way to change borders
\usepackage
    [
        % Works with A4 and US-letter paper.
        paperwidth=170mm,
        paperheight=259mm,
        left=1.12cm,
        right=1.12cm,
        top=2.0cm,
        bottom=2.0cm
    ]
    {geometry}

% Bibliography
\usepackage[numbers]{natbib}
\bibliographystyle{splncs03}
\renewcommand\bibsection
{%
    \chapter*{\bibname}
}

