\chapter
[%
    Computing Decompositions with Small Breadth%
]
{%
    Computing Decompositions with Small Breadth%
    \chapterNote
    {
        Results from this chapter have been published partially
        at~\emph{SWAT}~2014, Copenhagen, Denmark~\cite{DragKoehLeit2014},
        at~\emph{COCOA}~2016, Hong Kong, China~\cite{LeiterDragan2016}, and
        in~\emph{Algorithmica}~\cite{DragKoehLeit2017}.
    }%
}
\label{cha:compDeco}
%

For each graph class~$\calC$, a central problem is its recognition.
That is, given a graph~$G$, is $G$ a member of~$\calC$?
In case of tree- and path-decompositions, we can define recognition as decision problem or as optimization problem.
Using tree-breadth as example, the decision problem asks, for a given graph~$G$ and a given integer~$\rho \geq 1$, if $\tb(G) \leq \rho$.
The corresponding optimisation problem asks for a decomposition of~$G$ such that the breadth of this decomposition is minimal.
In this chapter, we show approaches to determine a tree- or path-decomposition with small breadth (or length) for a given graph.
We show results for general graphs as well as for special graph classes.

We know from \name{Lokshtanov}~\cite{Lokshtanov2010} and \name{Ducoffe} et al.\,\cite{DucoLegaNiss2016} that it is NP-hard to determine any of the parameters tree-breadth, tree-length, path-breadth, and path-length for a given graph.

\begin{theorem}
    [\name{Lokshtanov}~\cite{Lokshtanov2010}]
    \label{theo:TreeLengthNPc}
For a given graph~\( G \) and a fixed \( \lambda \geq 2 \), it is NP-complete to decide if \( \tl(G) \leq \lambda \).
\end{theorem}

\begin{theorem}
    [\name{Ducoffe} et al.\,\cite{DucoLegaNiss2016}]
Given a graph~\( G \) and an integer~\( k \), it is NP-complete to decide any of the following:
Is \( \tb(G) \leq k \)?
Is \( \pb(G) \leq k \)?
Is \( \pl(G) \leq k \)?
\end{theorem}


\section{Approximation Algorithms}

In this section, we present approaches to compute a decomposition for a given graph which approximates the graph's tree-breadth, tree-length, path-breadth, or path-length.
Additionally to showing that it is hard to compute any of these parameters optimally, \name{Lokshtanov}~\cite{Lokshtanov2010} and \name{Ducoffe} et al.\,\cite{DucoLegaNiss2016} show that a certain approximation quality is hard, too.

\begin{theorem}
    [\name{Lokshtanov}~\cite{Lokshtanov2010}]
If \( \mathit{P} \neq \mathit{NP} \), then there is no polynomial time algorithm to calculate a tree-decomposition, for a given graph~\( G \), of length smaller than~\( \frac{3}{2} \tl(G) \).
\end{theorem}

\begin{theorem}
    [\name{Ducoffe} et al.\,\cite{DucoLegaNiss2016}]
For any $\epsilon > 0$, it is NP-hard to approximate the tree-breadth of a given graph by a factor of~\( (2 - \epsilon) \).
\end{theorem}


\subsection{Layering Based Approaches}

One approach for computing a tree-decomposition for a given graph is to use a layering partition.
Lemma~\ref{lem:LayPartDist} shows that the radius and diameter of the clusters of a layering partition are bounded by the tree-breadth and tree-length of the underlying graph.

\begin{lemma}
    [%
        \name{Dourisboure} et al.\,\cite{DouDraGavYan2007},
        \name{Dragan} and \name{Köhler}~\cite{DraganKohler2014}%
    ]
    \label{lem:LayPartDist}
In a graph~\( G \) with \( \tb(G) = \rho \) and \( \tl(G) = \lambda \),
let \( C \) be a cluster of an arbitrary layering partition for~\( G \).
There is a vertex~\( w \) with \( d_G(w, u) \leq 3 \rho \) for each \( u \in C \).
Also, \( d_G(u, v) \leq 3 \lambda \) for all \( u, v \in C \).
\end{lemma}

Recall that the clusters created by a layering partition form a rooted tree.
This tree can be transformed into a tree-decomposition by expanding its clusters as follows.
For a cluster~$C$, add all vertices from the parent of~$C$ which are adjacent to a vertex in~$C$.
That is, for each cluster~$C \subseteq L_i^{(s)}$, create a bag~$B_C = C \cup \left(N_G(C) \cap L_{i-1}^{(s)}\right)$.
As shown in Lemma~\ref{lem:LayPartDist}, the radius and diameter of a cluster are at most three times larger than the tree-breadth and -length of~$G$, respectively.
Therefore, the created tree-decomposition has length at most~$3 \lambda + 1$ and breadth at most~$3 \rho + 1$.
\name{Abu-Ata} and \name{Dragan}~\cite{AbuAtaDragan2016} slightly improve this observation and show that the breadth of such a tree-decomposition is indeed at most~$3 \rho$.

\begin{corollary}
    [%
        \name{Dourisboure} et al.\,\cite{DouDraGavYan2007},
        \name{Abu-Ata} and \name{Dragan}~\cite{AbuAtaDragan2016}%
    ]
For a given graph~\( G \), a tree-decomposition with breadth~\( 3 \tb(G) \) and length~\( 3 \tl(G) + 1 \) can be computed in linear time.
\end{corollary}

A similar strategy as above can be used to approximate the path-breadth and -length of a graph.
However, there are two major differences.
First, a layering partition creates a tree and, second, can start at any vertex.
The first difference can be addressed by using a simple BFS-layering.
For the second, however, we need to find the right start vertex.

Consider a given start vertex~$s$ and the layers~$L_i^{(s)}$.
We define an \emph{extended layer~\( \mathbb{L}_i^{(s)} \)} as follows:
\[
    \mathbb{L}_i^{(s)} = L_{i}^{(s)} \cup \left \{ \, u \ \middle| \ uv \in E, u \in L_{i-1}^{(s)}, v \in L_{i}^{(s)} \, \right \}
\]

\begin{lemma}
    \label{lem:bfsLayApproxDist}
Let \( \calP = \{ X_1, X_2, \ldots, X_p \} \) be a path-decomposition for~\( G \) with length~\( \lambda \), breadth~\( \rho \), and \( s \in X_1 \).
Then each extended layer \( \mathbb{L}_i^{(s)} \) has diameter at most~\( 2 \lambda \) and radius at most~\( 3 \rho \).
\end{lemma}

\begin{proof}
Let $x$ and~$y$ be two arbitrary vertices in $L_{i}^{(s)}$.
Also let $x'$ and~$y'$ be arbitrary vertices in $L_{i-1}^{(s)}$ with $xx', yy' \in E$.

First, we show that $\max \big \{ d_G(x, y), d_G(x, y'), d_G(x', y) \big \} \leq 2 \lambda$.
By induction on~$i$, we may assume that $d_G(y', x') \leq 2 \lambda$ as $x', y' \in L_{i-1}^{(s)}$.
If there is a bag in~$\calP$ containing both vertices $x$ and~$y$, then $d_G(x, y) \leq \lambda$ and, therefore, $d_G(x, y') \leq \lambda + 1 \leq 2 \lambda$ and $d_G(y, x') \leq \lambda + 1 \leq 2 \lambda$.
Assume now that all bags containing~$x$ are earlier in~$\calP = \{ X_1, X_2, \ldots, X_p \}$ than the bags containing~$y$.
Let $B$ be a bag of~$\calP$ containing both ends of edge~$xx'$.
By the position of this bag~$B$ in~$\calP$ and the fact that $s \in X_1$, any shortest path connecting $s$ with~$y$ must have a vertex in~$B$.
Let $w$ be a vertex of~$B$ that is on a shortest path of~$G$ connecting vertices $s$ and~$y$ and containing edge~$yy'$.
Such a shortest path must exist because of the structure of the layering that starts at~$s$ and puts $y'$ and~$y$ in consecutive layers.
Since $x, x', w \in B$, we have $\max \big \{ d_G(x, w), d_G(x', w) \big \} \leq \lambda$.
If $w = y'$, we are done; $\max \big \{ d_G(x, y), d_G(x, y'), d_G(x', y) \big \} \leq \lambda + 1 \leq 2 \lambda$.
So, assume that $w \neq y'$.
Since $d_G(x, s) = d_G(s, y) = i$ (by the layering) and $d_G(x, w) \leq \lambda$, we must have $d_G(w, y') + 1 = d_G(w, y) = d_G(s, y) - d_G(s, w) = d_G(s, x) - d_G(s, w) \leq d_G(w, x) \leq \lambda$.
Hence, $d_G(y, x) \leq d_G(y, w) + d_G(w, x) \leq 2 \lambda$, $d_G(y, x') \leq d_G(y, w) + d_G(w, x') \leq 2 \lambda$, and $d_G(y', x) \leq d_G(y', w) + d_G(w, x) \leq 2 \lambda - 1$.
Therefore, we conclude that the distance between any two vertices in $\mathbb{L}_i^{(s)}$ is at most~$2 \lambda$.

As shown above, the radius of each extended layer~$\mathbb{L}_i^{(s)}$ is at most $4 \rho$ (because $\lambda \leq 2\rho$).
Consider an extended layer~$\mathbb{L}_i^{(s)}$ and a family~$\calF = \Big \{ \, N_G^{2 \rho}[u] \Bigm| u \in \mathbb{L}_i^{(s)} \, \Big \}$ of disks of~$G$.
Since $d_G(u, v) \leq 4 \rho$ for every pair $u, v \in \mathbb{L}_i^{(s)}$, the disks of~$\calF$ pairwise intersect.
Note that each disk~$N_G^{2 \rho}[u] \in \calF$ induces a subpath of~$\calP$.
If subtrees of a tree~$T$ pairwise intersect, they have a common node in~$T$~\cite{Berge1984}.
Therefore, there is a bag~$X_j \in \calP$ such that each disk in~$\calF$ intersects $X_j$.
Let $w$ be the center of~$X_j$, \ie $X_j \subseteq N_G^\rho[w]$.
Hence, for each vertex~$u$ with $N_G^{2 \rho}[u] \in \calF$, $d_G(w, u) \leq 3 \rho$.
Thus, for each $i \in \{ 1, \ldots, \ecc(s) \}$ there is a vertex~$w_i$ with $\mathbb{L}_i^{(s)} \subseteq N_G^{3 \rho}[w_i]$.
\end{proof}

Using Lemma~\ref{lem:bfsLayApproxDist}, Algorithm~\ref{algo:LayPathApprox} below computes a $3$-approximation for the path-breadth and a $2$-approximation for the path-length of a given graph.

\begin{algorithm}
    [htb]
    \caption
    {%
        A $2$-approximation algorithm for computing the path-length of a graph (respectively, $3$-approximation for path-breadth).
    }
    \label{algo:LayPathApprox}

\KwIn
{%
    A graph $G = (V, E)$.
}

\KwOut
{%
    A path-decomposition for~$G$.
}

Compute the pairwise distances of all vertices in~$G$.

\ForEach
{%
    $s \in V$
}
{%
    Calculate a decomposition $\calL(s) = \left \{ \, \mathbb{L}_i^{(s)} \Bigm| 1 \leq i \leq \ecc(s) \, \right \}$.

    Determine the length~$l(s)$ of $\calL(s)$ (breadth~$b(s)$, respectively).
}

Output a decomposition $\calL(s)$ for which $l(s)$ ($b(s)$, respectively) is minimal.
\end{algorithm}

\begin{theorem}
    \label{theo:algoLayPathApprox}
Algorithm~\ref{algo:LayPathApprox} computes a path-decomposition with length~\( 2 \pl(G) \) (with breadth~\( 3 \pb(G) \), respectively) of a given graph~\( G \) in \( \calO \big( n^3 \big) \) time.
\end{theorem}

\begin{proof}
Let $\pb(G) = \rho$ and $\pl(G) = \lambda$.
By Lemma~\ref{lem:bfsLayApproxDist}, there is a vertex~$s$ for which each extended layer~$\mathbb{L}_i^{(s)}$ has diameter at most~$2 \lambda$ (radius at most~$3 \rho$, respectively).
Thus, the algorithm creates and outputs a decomposition~$\calL(s)$ with length at most~$2 \lambda$ (breadth at most~$3 \rho$, respectively).

Determining pairwise distances and creating the decomposition~$\calL(s)$ for each vertex~$s$ can be done in $\calO(nm)$ total time.
Then, for a single vertex~$s$, the length and breadth of the decomposition~$\calL(s)$ can be calculated in $\calO \big( n^2 \big)$~time.
Thus, Algorithm~\ref{algo:LayPathApprox} runs in $\calO \big( n^3 \big)$ total time.
\end{proof}

\subsection{Neighbourhood Based Approaches}

In the previous subsection, we use distance layerings to construct a decomposition for a given graph.
While this approach is simple, it has limitations.
Consider the graph~$G$ in Figure~\ref{fig:LayPartApprox}.
$G$ has tree-breadth~$1$ and -length~$2$.
However, a layering partition starting at~$s$ creates a cluster~$C$ with radius~$3$ and diameter~$6$.
Now, one can create a graph~$H$ containing two copies of~$G$ such that the $s$-vertices are connected by a path.
Therefore, any layering partition of~$H$ creates such a cluster~$C$.

\begin{figure}
    [htb]
    \centering
    \tikzsetnextfilename{fig_deco_LayPartApprox}
\begin{tikzpicture}
    [yscale=-1]

\foreach \i in {-3,...,3}
{
    \node [nN] (n0\i) at (\i,0) {};
}

\foreach \i in {-2,...,2}
{
    \node [nN] (n1\i) at (\i,1) {};
}

\node [nN] (n2-1) at (-1,2) {};
\node [nN] (n21) at (1,2) {};

\node [nN] (s) at (0,3) {};

\begin{pgfonlayer}{background}

% fill=clLight60Green!30,
% draw=clDark25Green

\begin{scope}
    [
        every node/.style=
        {
            fill=clLight60Green!30,
            draw=clDark50Green,
        },
        every fit/.append style=text badly centered
    ]

    \node [fit=(n0-3)(n03)] (l0) {};
    \node [fit=(n1-2)(n12)] (l1) {};
    \node [fit=(n2-1)(n21)] (l2) {};
    \node [fit=(s)] (l3) {};

\end{scope}

\node [rlbl] at (l0.east) {$C$};

\draw (0,3) -- (-3,0);
\draw (0,3) -- (3,0);

\draw (-3,0) -- (3,0);

\foreach \i in {-2,...,2}
{
    \draw (\i,1) -- (\i,0);
}

\draw (-1,2) -- (-2,1);
\draw (-1,2) -- (-1,1);
\draw (-1,2) -- (0,1);

\draw (1,2) -- (0,1);
\draw (1,2) -- (1,1);
\draw (1,2) -- (2,1);

\draw (-1,1) -- (-2,0);
\draw (-1,1) -- (0,0);

\draw (1,1) -- (0,0);
\draw (1,1) -- (2,0);

\draw (-1,1) -- (-2,1);
\draw (1,1) -- (2,1);
\end{pgfonlayer}

\node [blbl,yshift=-4pt] at (s.south) {$s$};
\node [blbl,yshift=-4pt] at (n0-3.south) {$x$};
\node [blbl,yshift=-4pt] at (n03.south) {$y$};

\node [rlbl] at (n1-1.east) {$a$};
\node [blbl,yshift=-4pt] at (n10.south) {$b$};
\node [llbl] at (n11.west) {$c$};

\end{tikzpicture}

    \caption
    [%
        A graph~$G$ where a layering partition creates a cluster~$C$ with radius~$3 \tb(G)$ and diameter~$3 \tl(G)$.
    ]
    {
        A graph~$G$ where a layering partition creates a cluster~$C$ with radius~$3 \tb(G)$ and diameter~$3 \tl(G)$.
        The family of disks~$\big \{ N[a], N[b], N[c], N[x], N[y] \big \}$ gives a tree-decomposition with breadth~$1$ and length~$2$.
        The cluster~$C$ has radius~$3$ (for all~$v \in C$, $d(s, v) = 3$) and diameter~$6$ ($d(x, y) = 6$).
    }
    \label{fig:LayPartApprox}
\end{figure}

An alternative approach to distance layers is to carfully create a bag from the neighbourhood of a vertex.
Then add the bag to the tree-decomposition, pick a new vertex, and repeat the process.
This approach is clearly more complicated.
However, it has a better chance to determine a good decomposition for a given graph.

In~\cite{DourisGavoil2007}, \name{Dourisboure} and \name{Gavoille} present such an algorithm to compute a tree-decomposition which approximates the tree-length of a given graph.
They show that, for a given graph~$G$, their algorithm successfully computes a tree-decomposition with length at most~$6 \tl(G) - 4$.
Additionally, they conjecture that their algorithm can also find a tree-decomposition with length at most~$2 \tl(G)$.
However, \name{Yancey}~\cite{Yancey2014} is able to construct a counter example for this conjecture.

In what follows, we present an algorithm which computes a tree-decomposition with breadth at most~$3 \tb(G) - 1$ for a given graph~$G$.
To do so, assume for the remainder of this subsection that we are given a graph~$G$ with $\tb(G) = \rho$.

For some vertex~$u$ and some positive integer~$\phi$, let $\calC_G^\phi[u]$ denote the set of connected components in~$G - N_G^\phi[u]$.
We say that a vertex~$v$ is a \emph{\( \phi \)-partner} of~$u$ for some~$C \in \calC_G^\phi[u]$ if $N_G^\phi[v] \supseteq N_G(C)$ and $N_G^\phi[v] \cap C \neq \emptyset$.

\begin{lemma}
    \label{lem:potPartnerInCC}
Let \( u \) be an arbitrary vertex of~\( G \).
If \( \phi \geq 3 \rho - 1 \), then \( G \) contains, for each connected component~\( C \in \calC_G^\phi[u] \), a vertex~\( v \) such that \( v \) is a \( \phi \)-partner of~\( u \) for~\( C \).
\end{lemma}

\begin{proof}
Let $T$ be a tree-decomposition for~$G$ with breadth~$\rho$ and let $B_u$ be a bag of~$T$ containing~$u$.
Without loss of generality, let $T$ be rooted in~$B_u$.
Now, let $B_C$ be the bag of~$T$ for which $B_C \cap C \neq \emptyset$ and which is closest to~$B_u$ in~$T$.
Additionally, let $B_C'$ be the parent of~$B_C$ and let $S = B_C \cap B_C'$.
Note that, by definition of~$B_C$, $S \subseteq N_G^\phi[u]$ and $B_C$ contains a vertex~$x$ with $d_G(u, x) > \phi$.
Recall that the distance between two vertices in a bag is at most~$2 \rho$.
Thus, $d_G(u, s) < \rho$ implies that $d_G(u, x) < 3 \rho \leq \phi$.
Therefore, for all~$s \in S$, $d_G(u, s) \geq \rho$.

Let $v$ be the center of~$B_C$.
Because $B_C$ intersects~$C$ and $T$ has breadth~$\rho$, clearly, $d_G(v, C) \leq \rho < \phi$ and, hence, $C \cap N_G^\phi[v] \neq \emptyset$.
Let $x$ be a vertex of~$G$ in $N_G(C)$.
By properties of tree-decompositions, each path from $u$ to~$x$ intersect~$S$.
Thus, there is a vertex~$s \in S$ with $d_G(u, x) = d_G(u, s) + d_G(s, x)$.
Because $d_G(u, s) \geq \rho \geq d_G(v, s)$, it follows that $d_G(v, x) \leq d_G(u, x) \leq \phi$ and, thus, $N(C) \subseteq N_G^\phi[v]$.
Therefore, $v$ is a $\phi$-partner of~$u$ for~$C$.
\end{proof}

\begin{lemma}
    \label{lem:potPartnerKeepCCs}
Let \( C \) be a connected component in \( G - B_u \) for some \( B_u \subseteq N_G^\phi[u] \) and some positive~\( \phi \).
Also, let  \( C \in \calC_G^\phi[u] \) and let \( v \) be a \( \phi \)-partner of~\( u \) for~\( C \).
Then, for all connected components~\( C_v \) in \( G[C] - N_G^\phi[v] \), \( C_v \in \calC_G^\phi[v] \).
\end{lemma}

\begin{proof}
Consider a connected component~$C_v$ in $G[C] - N_G^\phi[v]$.
Clearly, $C_v \subseteq C$ and there is a connected component~$C' \in \calC_G^\phi[v]$ such that $C' \supseteq C_v$.

Let $x$ be an arbitrary vertex in~$C'$.
Then, there is a path~$P \subseteq C'$ from $x$ to~$C_v$.
Because $N_G(C) \subseteq B_u$ and $v$ is a $\phi$-partner of~$u$ for~$C$, $N_G(C) \subseteq B_u \cap N_G^\phi[v]$.
Also, $N_G(C)$ separates all vertices in~$C$ from all other vertices in~$G$.
Therefore, $x \in C$ and $C' \subseteq C$; otherwise, $P$ would intersect $N_G^\phi[v]$.
It follows that each vertex in~$P$ is in the same connected component of~$G[C] - N_G^\phi[v]$ and, thus, $C_v = C'$.
\end{proof}

Based on Lemma~\ref{lem:potPartnerInCC} and Lemma~\ref{lem:potPartnerKeepCCs}, we can compute, for a given~$\phi \geq 3 \rho - 1$, a tree-decomposition with breadth~$\phi$ as follows.
Pick a vertex~$u$ and make it center of a bag~$B_u = N_G^\phi$.
By Lemma~\ref{lem:potPartnerInCC}, $u$ has a $\phi$-partner~$v$ for each connected component~$C_u \in \calC_G^\phi[u]$.
Then, $N_G^\phi[v]$ splits $C_u$ in more connected components~$C_v$.
Due to Lemma~\ref{lem:potPartnerKeepCCs}, any of these components~$C_v$ is in~$\calC_G^\phi[v]$.
Thus, $v$ has a $\phi$-partner~$w$ for~$C_v$.
Hence, create a bag~$B_v = N_G^\phi[v] \cap (B_u \cup C_u)$ and continue this until the whole graph is covered.
Algorithm~\ref{algo:3TbApprox} below implements this approach.

\begin{algorithm}
    [!htb]
    \caption
    {%
        Constructs, for a given graph~$G = (V, E)$ and a given positive integer~$\phi$, a tree-decomposition~$T$ with breadth~$\phi$.
    }
    \label{algo:3TbApprox}

\KwIn
{%
    A graph~$G = (V, E)$ and a positive integer~$\phi$.
}

\KwOut
{%
    A tree-decomposition~$T$ for~$G$ if $\phi \geq 3 \tb(G) - 1$.
}

Determine the pairwise distances of all vertices and create an empty tree-decomposition~$T$.
\label{line:3Tb_Distances}

Initialise an empty queue~$Q$ of triples~$(v, B, C)$ where $v$ is a vertex of~$G$ and $B$ is a bag and $C$ is a connected component, \ie, $B$ and~$C$ are vertex sets.
\label{line:3Tb_InitQ}

Pick an arbitrary vertex~$u$ and insert $(u, \emptyset, V)$ into~$Q$.
\label{line:3Tb_PickU}

\While
{%
    \( Q \) is non-empty.%
    \label{line:3Tb_Qloop}
}
{%
    Remove a triple~$(v, B_u, C_u)$ from~$Q$.

    Create a bag~$B_v := N_G^\phi[v] \cap (B_u \cup C_u)$ and add $B_v$ into~$T$.
    \label{line:3Tb_createBv}

    \ForEach
    {%
        connected component~\( C_v \) in \( G[C_u] - B_v \)
    }
    {%
        Find a $\phi$-partner~$w$ of~$v$ for~$C_v$.
        \label{line:3Tb_findPotPart}

        If no such~$w$ exists, \keyword{Stop}. $\phi < 3 \tb(G) - 1$.

        Insert $(w, B_v, C_v)$ into~$Q$.
        \label{line:3Tb_InsertTriple}
    }
}

Output~$T$.
\end{algorithm}

\begin{theorem}
Algorithm~\ref{algo:3TbApprox} successfully constructs, for a given graph~\( G \) and a positive integer~\( \phi \), a tree-decomposition~\( T \) with breadth~\( \phi \) in \( \calO \big( n^3 \big) \) time if \( \phi \geq 3 \tb(G) - 1 \).
\end{theorem}

\begin{proof}
    [Correctness]
To show the correctness of the algorithm, we have to show that the created tree-decomposition~$T$ is a valid tree-decomposition for~$G$ if $\phi \geq 3 \tb(G) - 1$.
To do so, we show the following invariant for the loop starting in line~\ref{line:3Tb_Qloop}:
\begin{enumerate*}[(i),mode=unboxed]
    \item
        \label{item:3Tb_TvalidTB}
        $T$ is a valid tree-decomposition with breadth~$\phi$ for the subgraph covered by~$T$,
    \item
        \label{item:3Tb_conCompInQ}
        for each connected component~$C$ in~$G - T$, $Q$ contains a triple~$(v, B, C)$, and
    \item
        \label{item:3Tb_conCompCcond}
        for each triple~$(v, B_u, C_u)$ in~$Q$, $B_u$ is in~$T$, $N_G(C_u) \subseteq B_u$, $C_u \in \calC_G[u]$, and $v$ is a $\phi$-partner of~$u$ for~$C_u$.
\end{enumerate*}

Let $u$ be the vertex of~$G$ selected in line~\ref{line:3Tb_PickU}.
In the first iteration of the loop, the triple~$(u, \emptyset, V)$ is removed from~$Q$.
Then, the algorithm creates the bag~$B_u = N_G^\phi[u] \cap (\emptyset \cup V) = N_G^\phi[u]$ (line~\ref{line:3Tb_createBv}) and adds it into~$T$.
Clearly, since $B_u$ is the only bag of~$T$ at this point, condition~\ref{item:3Tb_TvalidTB} is satisfied.
Additionally, each connected component~$C$ in~$G - T$ is also in~$\calC[u]$.
Next, the algorithms determines a $\phi$-partner~$v$ of~$u$ for each~$C_u \in \calC[u]$ (line~\ref{line:3Tb_findPotPart}).
Due to Lemma~\ref{lem:potPartnerInCC}, such a $v$ can be found for each such component~$C_u$.
Therefore, after inserting the triples~$(v, B_u, C_u)$ into~$Q$ (line~\ref{line:3Tb_InsertTriple}), condition~\ref{item:3Tb_conCompInQ} and condition~\ref{item:3Tb_conCompCcond} are satisfied after the first iteration of the loop.

Now, assume by induction that the invariant holds each time line~\ref{line:3Tb_Qloop} is checked.
If $Q$ is empty, it follows from condition~\ref{item:3Tb_conCompInQ} that $T$ covers the whole graph and, thus, by condition~\ref{item:3Tb_TvalidTB}, $T$ is a valid tree-decomposition for~$G$.
If $Q$ is not empty, it contains a triple~$(v, B_u, C_u)$ which is then removed by the algorithm.
Because of condition~\ref{item:3Tb_conCompCcond}, it follows that $B_v$ as created in line~\ref{line:3Tb_createBv} contains~$N_G(C_u)$ and, thus, $N_G(C_u) \subseteq B_u \cap B_v$.
Therefore, because $C_u$ is not covered by~$T$, $T$ remains a valid tree-decomposition after adding~$B_v$, \ie, condition~\ref{item:3Tb_TvalidTB} is satisfied.
The bag~$B_v$ splits $C_u$ in a set~$\calC_v'$ of connected components such that, for each $C' \in \calC_v'$, $N_G(C') \subseteq B_v$ and, by Lemma~\ref{lem:potPartnerKeepCCs}, $C' \in \calC[v]$.
Therefore, due to Lemma~\ref{lem:potPartnerInCC}, $v$ has a $\phi$-partner~$w$ for each $C' \in \calC_v'$.
Thus, finding such a $\phi$-partner (line~\ref{line:3Tb_findPotPart}) is successful and, after inserting the triples~$(w, B_v, C_v)$ into~$Q$ (line~\ref{line:3Tb_InsertTriple}), condition~\ref{item:3Tb_conCompInQ} and condition~\ref{item:3Tb_conCompCcond} are satisfied.
\end{proof}

\begin{proof}
    [Complexity]
Determining the pairwise distances of all vertices, initialising $Q$ and~$T$, as well as inserting the first triple~$(u, \emptyset, V)$ into~$Q$ (line~\ref{line:3Tb_Distances} to line~\ref{line:3Tb_PickU}) can be clearly done in $\calO(nm)$ time.

In each iteration of the loop starting in line~\ref{line:3Tb_Qloop}, the algorithm splits a connected component~$C_u$ into a set of connected components~$C_v$.
Because $v$ is a $\phi$-partner of~$u$, it follows that $B_v \cap C_v \neq \emptyset$ and, hence, $C_v \subset C_u$.
Note that the created components~$C_v$ are pairwise disjoint.
Therefore, the algorithm creates at most~$n$ connected components~$C_v$, the set of vertices covered by~$T$ strictly grows, and, thus, there are at most~$n$ iterations of the loop.

Because the pairwise distances are known, it takes at most~$\calO(n)$ time to crate the bag~$B_v$ and to add it into~$T$ (line~\ref{line:3Tb_createBv}).
Determining the connected components~$C_v$ of~$G[C_u] - B_v$ and determining $N_G(C_v)$ for each such~$C_v$ can be easily done in $\calO(m)$~time.
Therefore, when excluding line~\ref{line:3Tb_findPotPart}, a single iteration of the loop starting in line~\ref{line:3Tb_Qloop} takes at most $\calO(m)$~time.

Let $\mathbb{C}$ be the set of all connected components created by the algorithm.
Clearly, for each connected component~$C \in \mathbb{C}$, $|N_G(C)| \leq n$.
Therefore, because at most~$n$ connected component~$C_v$ are created, $\sum_{C \in \mathbb{C}} |N_G(C)| \leq n^2$.
Because distances can be checked in constant time, it follows that, for a single vertex~$w$, it takes at most $\calO \big( n^2 \big)$ total time to determine, for all $C \in \mathbb{C}$, if $w$ is a $\phi$-partner for~$C$.
Therefore, line~\ref{line:3Tb_findPotPart} requires at most $\calO \big( n^3 \big)$ total time over all iterations.

Hence, Algorithm~\ref{algo:3TbApprox} runs in total $\calO \big( n^3 \big)$ time.
\end{proof}

If we want to use Algorithm~\ref{algo:3TbApprox} to find a decomposition with small breadth for a given graph~$G$, we hav to try different values of~$\phi$.
One way is to perform a one-sided binary search over~$\phi$:
If Algorithm~\ref{algo:3TbApprox} creates a valid tree-decomposition, decrease~$\phi$.
Otherwise, increase~$\phi$.
Therefore:

\begin{corollary}
For a given graph~\( G \) with tree-breadth~\( \rho \), one can compute a tree-decomposition with breadth~\( 3 \rho - 1 \) in \( \calO \big ( n^3 \log \rho \big) \) time.
\end{corollary}


\section{Strong Tree-Breadth}


By definition, a tree-decomposition has breadth~$\rho$ if each bag~$B$ is the subset of the $\rho$-neighbourhood of some vertex~$v$, \ie, the set of bags is the set of \emph{subsets} of the $\rho$-neighbourhoods of \emph{some} vertices.
Recall that tree-breadth~$1$ graphs contain the class of dually chordal graphs which can be defined as follows:
A graph~$G$ is dually chordal if it admits a tree-decomposition~$T$ such that, for each vertex~$v$ in~$G$, $T$~contains a bag~$B = N_G[v]$ (see Theorem~\ref{theo:duallyChordDeco}, page~\pageref{theo:duallyChordDeco}).
That is, the set of bags in~$T$ is the set of \emph{complete} neighbourhoods of \emph{all} vertices.

In this chapter, we investigate the case which lays between dually chordal graphs and general tree-breadth~$\rho$ graphs.
In particular, tree-decompositions are considered where the set of bags are the \emph{complete} $\rho$-neighbourhoods of \emph{some} vertices.
We call this \emph{strong tree-breadth}.
The \emph{strong breadth} of a tree-decomposition is~$\rho$, if, for each bag~$B$, there is a vertex~$v$ such that $B = N_G^\rho[v]$.
Accordingly, a graph~$G$ has strong tree-breadth smaller than or equal to~$\rho$ (written as $\stb(G) \leq \rho$) if there is a tree-decomposition for~$G$ with strong breadth at most~$\rho$.


\subsection{NP-Completeness}

In this section, we show that it is NP-complete to determine if a given graph has strong tree-breadth~$\rho$ even if $\rho = 1$.
To do so, we show first that, for some small graphs, the choice of possible centers is restricted.
Then, we use these small graphs to construct a reduction.

\begin{lemma}
    \label{lem:C4AdjVert}
Let \( C = \{ v_1, v_2, v_3, v_4 \} \) be an induced \( C_4 \) in a graph~\( G \) with the edge set \( \{ v_1v_2, \allowbreak v_2v_3, \allowbreak v_3v_4, \allowbreak v_4v_1 \} \).
If there is no vertex~\( w \notin C \) with \( N_G[w] \supseteq C \), then \( N_G[v_1] \) and~\( N_G[v_2] \) cannot both be bags in the same tree-decomposition with strong breadth~\( 1 \).
\end{lemma}

\begin{proof}
Assume that there is a decomposition~$T$ with strong breadth~$1$ containing the bags $B_1 = N_G[v_1]$ and $B_2 = N_G[v_2]$.
Because $v_3$ and~$v_4$ are adjacent, there is a bag~$B_3 \supseteq \{ v_3, v_4 \}$.
Consider the subtrees $T_1$, $T_2$, $T_3$, and~$T_4$ of~$T$ induced by $v_1$, $v_2$, $v_3$, and~$v_4$, respectively.
These subtrees pairwise intersect in the bags $B_1$, $B_2$, and~$B_3$.
Because pairwise intersecting subtrees of a tree have a common vertex, $T$ contains a bag~$N_G[w] \supseteq C$.
Note that there is no $v_i \in C$ with $N_G[v_i] \supseteq C$.
Thus, $w \notin C$.
This contradicts with the condition that there is no vertex~$w \notin C$ with $N_G[w] \supseteq C$.
\end{proof}

Let $C = \{ v_1, \ldots, v_5 \}$ be a $C_5$ with the edges $E_5 = \{ v_1v_2, v_2v_3, \ldots, v_5v_1 \}$.
We call the graph $H = \big( C \cup \{ u \}, E_5 \cup \{ uv_1, uv_3, uv_4 \} \big)$, with $u \notin C$, an \emph{extended~\( C_5 \) of degree~\( 1 \)} and refer to the vertices $u$, $v_1$, $v_2$, and $v_5$ as \emph{middle}, \emph{top}, \emph{right}, and \emph{left} vertex of~$H$, respectively.
Based on $H = (V_H, E_H)$, we construct an \emph{extended~\( C_5 \) of degree~\( \rho \)} (with $\rho > 1$) as follows.
First, replace each edge~$xy \in E_H$ by a path of length~$\rho$.
Second, for each vertex~$w$ on the shortest path from $v_3$ to~$v_4$, connect $u$ with~$w$ using a path of length~$\rho$.
Figure~\ref{fig:extC5} gives an illustration.

\begin{figure}
    [htb]
    \centering
    \begin{subfigure}[b]{0.45\textwidth}
        \centering
        \tikzsetnextfilename{fig_strB_extC5}
\begin{tikzpicture}

\coordinate (c0) at (0,0);
\foreach \i in {1,...,5}
{
    \coordinate (c\i) at (162-\i*72:1);
}

\foreach \i [evaluate=\i as \ii using {int(mod(\i,5)+1)}] in {1,...,5}
{
    \draw (c\i) -- (c\ii);
}

\draw (c0) -- (c1);
\draw (c0) -- (c3);
\draw (c0) -- (c4);


\foreach \i in {1,...,5}
{
    \node[sN] (n\i) at (c\i) {};
}

\node [sN] (n0) at (c0) {};

\node[rlbl] at (n0.east) {$u$};
\node[tlbl] at (n1.north) {$v_1$};
\node[rlbl] at (n2.east) {$v_2$};
\node[rlbl] at (n3.east) {$v_3$};
\node[llbl] at (n4.west) {$v_4$};
\node[llbl] at (n5.west) {$v_5$};

\end{tikzpicture}

        \caption
        {%
            Degree 1.
        }
        \label{fig:extC5deg1}
    \end{subfigure}
    \hfil
    \begin{subfigure}[b]{0.45\textwidth}
        \centering
        \tikzsetnextfilename{fig_strB_extC5deg3}
\begin{tikzpicture}
    [scale=1.3]

\coordinate (c0) at (0,0);
\foreach \i in {1,...,5}
{
    \coordinate (c\i) at (162-\i*72:1);
}

\foreach \i [evaluate=\i as \ii using {int(mod(\i,5)+1)}] in {1,...,5}
{
    \path
        (c\i) -- (c\ii)
        coordinate [pos=0.333] (c\i_1)
        coordinate [pos=0.6666] (c\i_2);
}

\foreach \i in {1,3,3_1,3_2,4}
{
    \path
        (c0) -- (c\i)
        coordinate [pos=0.3333] (c0\i_1)
        coordinate [pos=0.6666] (c0\i_2);
}


\foreach \i [evaluate=\i as \ii using {int(mod(\i,5)+1)}] in {1,...,5}
{
    \draw (c\i) -- (c\ii);
}


\foreach \i in {1,3,3_1,3_2,4}
{
    \draw (c0) -- (c\i);
}

\foreach \i in {1,...,5}
{
    \node[sN] (n\i) at (c\i) {};
    \node[tN] (n\i_1) at (c\i_1) {};
    \node[tN] (n\i_2) at (c\i_2) {};
}

\foreach \i in {1,3,3_1,3_2,4}
{
    \node [tN] at (c0\i_1) {};
    \node [tN] at (c0\i_2) {};
}

\node [sN] (n0) at (c0) {};

\node[rlbl] at (n0.east) {$u$};
\node[tlbl] at (n1.north) {$v_1$};
\node[rlbl] at (n2.east) {$v_2$};
\node[rlbl] at (n3.east) {$v_3$};
\node[llbl] at (n4.west) {$v_4$};
\node[llbl] at (n5.west) {$v_5$};

\end{tikzpicture}

        \caption
        {%
            Degree 3.
        }
        \label{fig:extC5deg3}
    \end{subfigure}

    \caption
    [%
        Two extended $C_5$ of degree~1 and degree~3.
    ]
    {%
        Two \emph{extended~\( C_5 \)} of \protect\subref{fig:extC5deg1} degree~1 and \protect\subref{fig:extC5deg3} degree~3.
        We refer to the vertices $u$, $v_1$, $v_2$, and $v_5$ as middle, top, right, and left vertex, respectively.
    }
    \label{fig:extC5}
\end{figure}

\begin{lemma}
    \label{lem:C5centers}
Let \( H \) be an extended \( C_5 \) of degree~\( \rho \) in a graph~\( G \) as defined above.
Additionally, let \( H \) be a block of~\( G \) and its top vertex~\( v_1 \) be the only articulation point of~\( G \) in~\( H \).
Then, there is no vertex~\( w \) in~\( G \) with \( d_G(w, v_1) < \rho \) which is the center of a bag in a tree-decomposition for~\( G \) with strong breadth~\( \rho \).
\end{lemma}

\begin{proof}
Let $T$ be a tree-decomposition for~$G$ with strong breadth~$\rho$.
Assume that $T$ contains a bag~$B_w = N_G^\rho[w]$ with $d_G(w, v_1) < \rho$.
Note that the distance from $v_1$ to any vertex on the shortest path from $v_3$ to~$v_4$ is $2\rho$.
Hence, $G - B_w$ has a connected component~$C$ containing the vertices $v_3$ and~$v_4$.
Then, by Lemma~\ref{lem:decoBorderBags} (page~\pageref{lem:decoBorderBags}), there has to be a vertex~$w' \neq w$ in~$G$ and a bag~$B_w' = N_G^\rho[w']$ in~$T$ such that
\begin{enumerate*}[(i),mode=unboxed]
    \item
        \label{item:BsubsetNC}
        $B_w' \supseteq N_G(C)$ and
    \item
        \label{item:BintC}
        $B_w' \cap C \neq \emptyset$.
\end{enumerate*}
Thus, if we can show, for a given~$w$, that there is no such~$w'$, then $w$ cannot be center of a bag.

First, consider the case that $w$ is in~$H$.
We construct a set~$X =\{ x, y \} \subseteq N_G(C)$ such that there is a unique shortest path from $x$ to~$y$ in~$G$ passing~$w$.
If $w = v_1$, let $x = v_2$ and $y = v_5$.
If $w$ is on the shortest path from $v_1$ to~$u$, let $x$ and~$y$ be on the shortest path from $v_1$ to~$v_2$ and from $v_4$ to~$u$, respectively.
If $w$ is on the shortest path from $v_1$ to~$v_2$, let $x$ and~$y$ be on the shortest path from $v_1$ to~$v_5$ and from $v_2$ to~$v_3$, respectively.
In each case, there is a unique shortest path from $x$ to~$y$ passing~$w$.
Note that, for all three cases, $d_G(v_1, y) \geq \rho$.
Thus, each $w'$  with $d_G(w', y) \leq \rho$ is in~$H$.
Therefore, $w$ is the only vertex in~$G$ with $X \subseteq N_G^\rho[w]$, \ie, there is no vertex~$w' \neq w$ satisfying condition~\ref{item:BsubsetNC}.
This implies that $w$ cannot be center of a bag in~$T$.

Next, consider the case that $w$ is not in~$H$.
Without loss of generality, let $w$ be a center for which $d_G(v_1, w)$ is minimal.
As shown above, there is no vertex~$w'$ in~$H$ with $d_G(v_1, w') < \rho$ which is center of a bag.
Hence, $w'$ is not in~$H$ either.
However, because $v_1$ is an articulation point, $w'$ has to be closer to~$v_1$ than $w$ to satisfy condition~\ref{item:BintC}.
This contradicts with $d_G(v_1, w)$ being minimal.
Therefore, there is no vertex~$w'$ satisfying condition~\ref{item:BintC} and $w$ cannot be center of a bag in~$T$.
\end{proof}


\begin{theorem}
    \label{theo:stbNPc}
It is NP-complete to decide, for a given graph~\( G \), if \( \stb(G) = 1 \).
\end{theorem}

\begin{proof}
Clearly, the problem is in NP:
Select non-deterministically a set~$S$ of vertices such that their neighbourhoods cover each vertex and each edge.
Then, check deterministically if the neighbourhoods of the vertices in~$S$ give a valid tree-decomposition.
This can be done in linear time (see Lemma~\ref{lem:decoValidation}, page~\ref{lem:decoValidation}).

To show that the problem is NP-hard, we make a reduction from 1-in-3-SAT~\cite{Schaefer1978}.
That is, you are given a boolean formula in CNF with at most three literals per clause; find a satisfying assignment such that, in each clause, only one literal becomes true.

Let $\calI$ be an instance of 1-in-3-SAT with the literals~$\calL = \{ p_1, \ldots, p_n \}$, the clauses~$\calC = \{ c_1, \ldots, c_m \}$, and, for each $c \in \calC$, $c \subseteq \calL$.
We create a graph~$G = (V, E)$ as follows.
Create a vertex for each literal~$p \in \calL$ and, for all literals $p_i$ and~$p_j$ with $p_i \equiv \neg p_j$, create an induced $C_4 = \{ p_i, p_j, q_i, q_j \}$ with the edges $p_ip_j$, $q_iq_j$, $p_iq_i$, and $p_jq_j$.
For each clause $c \in \calC$ with $c = \{ p_i, p_j, p_k \}$, create an extended $C_5$ with $c$ as top vertex, connect $c$ with an edge to all literals it contains, and make all literals in $c$ pairwise adjacent, \ie, the vertex set $\{ c, p_i, p_j, p_k \}$ induces a maximal clique in~$G$.
Additionally, create a vertex~$v$ and make $v$ adjacent to all literals.
Figure~\ref{fig:stbNPc} gives an illustration for the construction so far.

\begin{figure}
    [htb]
    \centering

    \begin{subfigure}[b]{0.45\textwidth}
        \centering
        \tikzsetnextfilename{fig_strB_stbNPc}
\begin{tikzpicture}
    [scale=1.75]

\node [sN] (v) at (0,0) {};
\node [sN] (pi) at (120:1) {};
\node [sN] (pj) at (90:1) {};
\node [sN] (pk) at (60:1) {};
\node [sN] (c) at (90:1.3) {};

\node [sN] (pl) at (150:1) {};

\node [sN] (qi) at ($(pi)+(135:0.5176)$) {};
\node [sN] (ql) at ($(pl)+(135:0.5176)$) {};

\def\len{0.33}

\coordinate (c5) at ($(c)+(0,\len)$);
\foreach \i in {1,...,5}
{
    \coordinate (c5\i) at ($(c5)+(-18-72*\i:\len)$);
}

\begin{pgfonlayer}{background}

\draw (v.center) -- (pi.center);
\draw (v.center) -- (pl.center);
\draw (v.center) -- (pj.center);
\draw (v.center) -- (pk.center);

\draw (pi.center) -- (pj.center);
\draw (pi.center) to[out=-30,in=210] (pk.center);
\draw (pj.center) -- (pk.center);

\draw (pi.center) -- (pl.center);
\draw (pi.center) -- (qi.center);
\draw (ql.center) -- (qi.center);
\draw (ql.center) -- (pl.center);


\draw (c.center) -- (pi.center);
\draw (c.center) -- (pj.center);
\draw (c.center) -- (pk.center);

\draw (c5) -- (c51);
\draw (c5) -- (c53);
\draw (c5) -- (c54);

\draw (c51)
\foreach \i in {2,...,5}
{
    -- (c5\i)
}
-- cycle;

\end{pgfonlayer}

\node [blbl] at (v.south) {$v$};
\node [rlbl] at (c.east) {$c$};

\node [llbl] at (pi.west) {$p_i$};
\node [lbl,anchor=north west] at (pj.south east) {$p_j$};
\node [rlbl] at (pk.east) {$p_k$};

\node [tlbl] at (qi.north) {$q_i$};
\node [llbl] at (ql.west) {$q_l$};
\node [blbl] at (pl.south) {$p_l$};


\end{tikzpicture}

        \caption{}
        \label{fig:stbNPc}
    \end{subfigure}
    \hfil
    \begin{subfigure}[b]{0.45\textwidth}
        \centering
        \tikzsetnextfilename{fig_strB_stbNPcRS}
\begin{tikzpicture}
    [scale=1.33]

\node [sN] (pi) at (0,0) {};
\node [sN] (pj) at (0:1) {};
\node [sN] (pk) at (60:1) {};

\node [sN] (r_ijk) at (-90:1) {};
\node [sN] (r_jki) at ($(pj.center)+(30:1)$) {};
\node [sN] (r_kij) at ($(pk.center)+(150:1)$) {};


\node [sN] (s_ijk) at ($(pj.center)+(-90:1)$) {};
\node [sN] (s_jki) at ($(pk.center)+(30:1)$) {};
\node [sN] (s_kij) at (150:1) {};


\begin{pgfonlayer}{background}

\draw (pi.center) -- (pj.center);
\draw (pi.center) -- (pk.center);
\draw (pj.center) -- (pk.center);


\draw (pi.center) -- (r_ijk.center);
\draw (pj.center) -- (r_jki.center);
\draw (pk.center) -- (r_kij.center);

\draw (pj.center) -- (s_ijk.center);
\draw (pk.center) -- (s_jki.center);
\draw (pi.center) -- (s_kij.center);

\draw (pk.center) to[out=-90,in=135] (s_ijk.center);
\draw (pi.center) to[out=30,in=-105] (s_jki.center);
\draw (pj.center) to[out=150,in=15] (s_kij.center);

\draw (r_ijk.center) -- (s_ijk.center);
\draw (r_jki.center) -- (s_jki.center);
\draw (r_kij.center) -- (s_kij.center);

\end{pgfonlayer}

\node [lbl,anchor=north east] at (pi.south west) {$p_i$};
\node [lbl,anchor=north west] at (pj.south east) {$p_j$};
\node [tlbl] at (pk.north) {$p_k$};

\node [blbl] at (r_ijk.south) {$r_{(ij|k)}$};
\node [lbl,anchor=120] at (r_jki.-60) {$r_{(jk|i)}$};
\node [llbl] at (r_kij.west) {$r_{(ki|j)}$};

\node [blbl] at (s_ijk.south) {$s_{(ij|k)}$};
\node [rlbl] at (s_jki.east) {$s_{(jk|i)}$};
\node [lbl,anchor=60] at (s_kij.-120) {$s_{(ki|j)}$};

\end{tikzpicture}

        \caption{}
        \label{fig:stbNPcRS}
    \end{subfigure}

    \caption
    [%
        Illustration to the proof of Theorem~\ref{theo:stbNPc}.
    ]
    {%
        Illustration to the proof of Theorem~\ref{theo:stbNPc}.
        The graphs shown are subgraphs of~$G$ as created by a clause~$c = \{ p_i, p_j, p_k \}$ and a literal~$p_l$ with $p_i \equiv \neg p_l$.
    }
    \label{fig:stbNPc_full}
\end{figure}

Next, for each clause $\{ p_i, p_j, p_k \}$ and for each $(xy|z) \in \big \{ (ij|k), (jk|i), (ki|j) \big \}$, create the vertices $r_{(xy|z)}$ and~$s_{(xy|z)}$, make $r_{(xy|z)}$ adjacent to $s_{(xy|z)}$ and~$p_x$, and make $s_{(xy|z)}$ adjacent to $p_y$ and $p_z$.
See Figure~\ref{fig:stbNPcRS} for an illustration.
Note that $r_{(ij|k)}$ and~$s_{(ij|k)}$ are specific for the clause $\{ p_i, p_j, p_k \}$.
Thus, if $p_i$ and~$p_j$ are additionally in a clause with~$p_l$, then we also create the vertices $r_{(ij|l)}$ and~$s_{(ij|l)}$.
For the case that a clause only contains two literals $p_i$ and~$p_j$, create the vertices $r_{(ij)}$ and~$s_{(ij)}$, make $r_{(ij)}$ adjacent to $p_i$ and~$s_{(ij)}$, and make $s_{(ij)}$ adjacent to $p_j$, \ie, $\{ p_i, p_j, r_{(ij)}, s_{(ij)} \}$ induces a~$C_4$ in~$G$.


For the reduction, first, consider the case that $\calI$ is a \emph{yes}-instance for 1-in-3-SAT.
Let $f \colon \calP \rightarrow \{ T, F \}$ be a satisfying assignment such that each clause contains only one literal~$p_i$ with $f(p_i) = T$.
Select the following vertices as centers of bags: $v$, the middle, left and right vertex of each extended $C_5$, $p_i$ if $f(p_i) = T$, and $q_j$ if $f(p_j) = F$.
Additionally, for each clause $\{ p_i, p_j, p_k \}$ with $f(p_i) = T$, select the vertices $s_{(ij|k)}$, $r_{(jk|i)}$, and $r_{(ki|j)}$.
The neighbourhoods of the selected vertices give a valid tree-decomposition for~$G$.
Therefore, $\stb(G) = 1$.


Next, assume that $\stb(G) = 1$.
Recall that, for a clause $c = \{ p_i, p_j, p_k \}$, the vertex set $\{ c, p_i, p_j, p_k \}$ induces a maximal clique~$K_c$ in~$G$.
Because each maximal clique is contained completely in some bag (Lemma~\ref{lem:cliqueBag}, page~\pageref{lem:cliqueBag}), some vertex in~$K_c$ is center of such a bag.
By Lemma~\ref{lem:C5centers}, $c$ cannot be center of a bag because it is top of an extended $C_5$.
Therefore, at least one vertex in $\{ p_i, p_j, p_k \}$ must be center of a bag.
Without loss of generality, let $p_i$ be a center of a bag.
By construction, $p_i$ is adjacent to all $p \in \{ p_j, p_k, p_l \}$, where $p_l \equiv \neg p_i$.
Additionally, $p$ and~$p_i$ are vertices in an induced $C_4$, say $C$, and there is no vertex~$w$ in~$G$ with $N_G[w] \supseteq C$.
Thus, by Lemma~\ref{lem:C4AdjVert}, at most one vertex in $\{ p_i, p_j, p_k \}$ can be center of a bag.
Therefore, the function $f \colon \calL \rightarrow \{ T, F \}$ defined as
\[
    f(p_i) =
        \begin{cases}
            T & \text{if $p_i$ is center of a bag,} \\
            F & \text{else} \\
        \end{cases}
\]
is a satisfying assignment for~$\calI$.
\end{proof}

In \cite{DucoLegaNiss2016}, \name{Ducoffe} et al.\ show how to construct a graph~$G'_\rho$ based on a given graph~$G$ such that $\tb(G'_\rho) = 1$ if and only if $\tb(G) \leq \rho$.
We slightly extend their construction to achieve a similar result for strong tree-breadth.

Consider a given graph~$G = (V, E)$ with $\stb(G) = \rho$.
We construct $G'_\rho$ as follows.
Let $V = \{ v_1, v_2, \ldots, v_n \}$.
Add the vertices $U = \{ u_1, u_2, \ldots, u_n \}$ and make them pairwise adjacent.
Additionally, make each vertex~$u_i$, with $1 \leq i \leq n$, adjacent to all vertices in~$N_G^\rho[v_i]$.
Last, for each $v_i \in V$, add an extended $C_5$ of degree~$1$ with $v_i$ as top vertex.

\begin{lemma}
    \label{lem:Gprime}
\( \stb(G) \leq \rho \) if and only if \( \stb(G'_\rho) = 1 \).
\end{lemma}

\begin{proof}
First, consider a tree-decomposition~$T$ for~$G$ with strong breadth~$\rho$.
Let $T'_\rho$ be a tree-decomposition for~$G'_\rho$ created from $T$ by adding all vertices in~$U$ into each bag of~$T$ and by making the center, left, and right vertices of each extended $C_5$ centers of bags.
Because the set~$U$ induces a clique in~$G'_\rho$ and $N_G^\rho[v_i] = N_{G'_\rho}[u_i] \cap V$, each bag of $T'_\rho$ is the complete neighbourhood of some vertex.

Next, consider a tree-decomposition~$T'_\rho$ for~$G'_\rho$ with strong breadth~$1$.
Note that each vertex~$v_i$ is top vertex of some extended $C_5$.
Thus, $v_i$ cannot be center of a bag.
Therefore, each edge $v_iv_j$ is in a bag~$B_k = N_{G'_\rho}[u_k]$.
By construction of~$G'_\rho$, $B_k \cap V = N_G^\rho[v_k]$.
Thus, we can construct a tree-decomposition~$T$ for~$G$ with strong breadth~$\rho$ by creating a bag~$B_i = N_G^\rho[v_i]$ for each bag~$N_{G'_\rho}[u_i]$ of~$T'_\rho$.
\end{proof}


Next, consider a given graph~$G = (V, E)$ with $V = \{ v_1, v_2, \ldots, v_n \}$ and $\stb(G) = 1$.
For a given $\rho > 1$, we obtain the graph~$G_\rho^+$ by doing the following for each $v_i \in V$:
\begin{itemize}
    \item
        Add the vertices $u_{i,1}$, \ldots, $u_{i,5}$, $x_i$, and $y_i$.
    \item
        Add an extended $C_5$ of degree~$\rho$ with the top vertex~$z_i$.
    \item
        Connect
        \begin{itemize}
            \item
                $u_{i,1}$ and $x_i$ with a path of length $\lfloor \rho / 2 \rfloor - 1$,
            \item
                $u_{i,2}$ and $y_i$ with a path of length $\lfloor \rho / 2 \rfloor$,
            \item
                $u_{i,3}$ and $v_i$ with a path of length $\lceil \rho / 2 \rceil - 1$,
            \item
                $u_{i,4}$ and $v_i$ with a path of length $\lfloor \rho / 2 \rfloor$, and
            \item
                $u_{i,4}$ and $z_i$ with a path of length $\lceil \rho / 2 \rceil - 1$.
        \end{itemize}
    \item
        Add the edges $u_{i,1}u_{i,2}$, $u_{i,1}u_{i,3}$, $u_{i,2}u_{i,3}$, $u_{i,2}u_{i,4}$, and $u_{i,3}u_{i,4}$.
\end{itemize}
Note that, for small $\rho$, it can happen that $v_i = u_{i,4}$, $x_i = u_{i,1}$, $y_i = u_{i,2}$, or $z_i = u_{i,5}$.
Figure~\ref{fig:Gplus} gives an illustration.

\begin{figure}
    [htb]
    \centering
    \tikzsetnextfilename{fig_strB_Gplus}
\begin{tikzpicture}
    [scale=1.33]
%

\node [sN] (v) at (0.5,0) {};

\node [sN] (q0) at (1,2) {};
\node [sN] (q1) at (0,1) {};
\node [sN] (q2) at (0,2) {};
\node [sN] (q3) at (1,1) {};

\node [sN] (x) at (0,3) {};
\node [sN] (y) at (1,3) {};
\node [sN] (c) at (2.15,1) {};


\def\len{0.66}

\coordinate (c5) at ($(c)+(\len,0)$);
\foreach \i in {1,...,5}
{
    \coordinate (c5\i) at ($(c5)+(-108-72*\i:\len)$);
}

\begin{pgfonlayer}{background}

\path
    (v.center) --
    coordinate[pos=0.4] (l1)
    coordinate[pos=0.3] (p_v1_1)
    coordinate[pos=0.7] (p_v1_2)
    (q1.center);

\path
    (v.center) --
    coordinate[pos=0.4] (l2)
    coordinate[pos=0.3] (p_v3_1)
    coordinate[pos=0.7] (p_v3_2)
    (q3.center);

\path
    (x.center) --
    coordinate[pos=0.3] (p_x2_1)
    coordinate[pos=0.7] (p_x2_2)
    (q2.center);

\path
    (y.center) --
    coordinate[pos=0.3] (p_y0_1)
    coordinate[pos=0.7] (p_y0_2)
    (q0.center);

\path
    (c.center) --
    coordinate[pos=0.5] (l3)
    coordinate[pos=0.3] (p_c3_1)
    coordinate[pos=0.7] (p_c3_2)
    (q3.center);

\foreach \p/\q/\a in {x/2/0,y/0/0,c/3/-90,v/1/210,v/3/150}
{
\draw
    (\p.center)
    .. controls +(90+\a:-0.15) and +(90+\a:0.15) .. ($(p_\p\q_1)+(\a:-0.06)$)
    .. controls +(90+\a:-0.15) and +(90+\a:0.15) .. ($(p_\p\q_2)+(\a:0.06)$)
    .. controls +(90+\a:-0.15) and +(90+\a:0.15) .. (q\q.center);
    -- (q\q.center);
}

\draw (q0.center) -- (q1.center);
\draw (q0.center) -- (q2.center);
\draw (q0.center) -- (q3.center);
\draw (q1.center) -- (q2.center);
\draw (q1.center) -- (q3.center);

\draw (c5) -- (c51);
\draw (c5) -- (c53);
\draw (c5) -- (c54);

\draw (c51)
\foreach \i in {2,...,5}
{
    -- (c5\i)
}
-- cycle;

\end{pgfonlayer}

\node [llbl] at (v.west) {$v_i$};
\node [llbl] at (x.west) {$x_i$};
\node [rlbl] at (y.east) {$y_i$};
\node [blbl] at (c.south) {$z_i$};

\node [llbl] at (q2.west) {$u_{i,1}$};
\node [rlbl] at (q0.east) {$u_{i,2}$};
\node [llbl] at (q1.west) {$u_{i,3}$};
\node [lbl,anchor=135] at (q3.south east) {$u_{i,4}$};


\node [llbl] at (0,2.5) {$\lfloor \frac{\rho}{2} \rfloor - 1$};
\node [rlbl] at (1,2.5) {$\lfloor \frac{\rho}{2} \rfloor$};
\node [tlbl] at (l3) {$\lceil \frac{\rho}{2} \rceil - 1$};

\node [rlbl] at (l2) {$\lfloor \frac{\rho}{2} \rfloor$};
\node [llbl] at (l1) {$\lceil \frac{\rho}{2} \rceil - 1$};

\end{tikzpicture}

    \caption
    [%
        Illustration for the graph~$G^+_\rho$.
    ]
    {%
        Illustration for the graph~$G^+_\rho$.
        The graph shown is a subgraph of~$G^+_\rho$ as constructed for each $v_i$ in~$G$.
    }
    \label{fig:Gplus}
\end{figure}


\begin{lemma}
    \label{lem:Gplus}
\( \stb(G) = 1 \) if and only if \( \stb(G_\rho^+) = \rho \).
\end{lemma}

\begin{proof}
First, assume that $\stb(G) = 1$.
Then, there is a tree-decomposition~$T$ for~$G$ with strong breadth~$1$.
We construct a tree-decomposition~$T_\rho^+$ for~$G_\rho^+$ with strong breadth~$\rho$.
Make the middle, left, and right vertex of each extended $C_5$ center of a bag.
For each $v_i \in V$, if $v_i$ is center of a bag of~$T$, make $x_i$ a center of a bag of~$T_\rho^+$.
Otherwise, make $y_i$ center of a bag of~$T_\rho^+$.
%
The distance in~$G_\rho^+$ from $v_i$ to~$x_i$ is $\rho - 1$.
The distances from $v_i$ to~$y_i$, from $x_i$ to~$z_i$, and from $y_i$ to~$z_i$ are $\rho$.
Thus, $N_{G_\rho^+}^{\rho}[x_i] \cap V = N_G[v_i]$, $N_{G_\rho^+}^{\rho}[y_i] \cap V = \{ v_i \}$, and there is no conflict with Lemma~\ref{lem:C5centers}.
Therefore, the constructed $T_\rho^+$ is a valid tree-decomposition with strong breadth~$\rho$ for~$G_\rho^+$.

Next, assume that $\stb(G_\rho^+) = \rho$ and there is a tree-decomposition~$T_\rho^+$ with strong breadth~$\rho$ for~$G_\rho^+$.
By Lemma~\ref{lem:C5centers}, no vertex in distance less than $\rho$ to any $z_i$ can be a center of a bag in~$T_\rho^+$.
Therefore, because the distance from $v_i$ to~$z_i$ in~$G_\rho^+$ is $\rho - 1$, no $v_i \in V$ can be a center of a bag in~$T_\rho^+$.
The only vertices with a large enough distance to~$z_i$  to be a center of a bag are $x_i$ and~$y_i$.
Therefore, either $x_i$ or~$y_i$ is selected as center.
To construct a tree-decomposition~$T$ with strong breadth~$1$ for~$G$, select $v_i$ as center if and only if $x_i$ is a center of a bag in~$T_\rho^+$.
Because $N_{G_\rho^+}^{\rho}[x_i] \cap V = N_G[v_i]$ and $N_{G_\rho^+}^{\rho}[y_i] \cap V = \{ v_i \}$, the constructed $T$ is a valid tree-decomposition with strong breadth~$1$ for~$G$.
\end{proof}

Constructing $G'_\rho$ can be done in $\calO \big( n^2 \big)$ time and constructing $G_\rho^+$ can be done in $\calO(\rho \cdot n + m)$ time.
Thus, combining Lemma~\ref{lem:Gprime} and Lemma~\ref{lem:Gplus} allows us, for a given graph~$G$, some given $\rho$, and some given~$\rho'$, to construct a graph~$H$ in $\calO \big( \rho \cdot n^2 \big)$ time such that $\stb(G) \leq \rho$ if and only if $\stb(H) \leq \rho'$.
Additionally, by combining Theorem~\ref{theo:stbNPc} and Lemma~\ref{fig:Gplus}, we get:

\begin{theorem}
It is NP-complete to decide, for a graph~\( G \) and a given~\( \rho \), if \( \stb(G) = \rho \).
\end{theorem}


\subsection{Polynomial Time Cases}

In the previous section, we have shown that, in general, it is NP-complete to determine the strong tree-breadth of a graph.
In this section, we investigate cases for which a decomposition can be found in polynomial time.

Let $G$ be a graph with strong tree-breadth~$\rho$ and let $T$ be a corresponding tree-decomposition.
For a given vertex~$u$ in~$G$, we denote the set of connected components in $G - N_G^\rho[u]$ as $\calC_G[u]$.
We say that a vertex~$v$ is a \emph{potential partner} of~$u$ for some $C \in \calC_G[u]$ if $N_G^\rho[v] \supseteq N_G(C)$ and $N_G^\rho[v] \cap C \neq \emptyset$.

\begin{lemma}
    \label{lem:potPartProp}
Let \( C \) be a connected component in \( G - B_u \) for some \( B_u \subseteq N_G^\rho[u] \).
Also, let \( C \in \calC_G[u] \) and \( v \) be a potential partner of~\( u \) for~\( C \).
Then, for all connected components~\( C_v \) in \( G[C] - N_G^\rho[v] \), \( C_v \in \calC_G[v] \).
\end{lemma}

\begin{proof}
Consider a connected component~$C_v$ in $G[C] - N_G^\rho[v]$.
Clearly, $C_v \subseteq C$ and there is a connected component~$C' \in \calC_G[v]$ such that $C' \supseteq C_v$.

Let $x$ be an arbitrary vertex in~$C'$.
Then, there is a path~$P \subseteq C'$ from $x$ to~$C_v$.
Because $N_G(C) \subseteq B_u$ and $v$ is a potential partner of~$u$ for~$C$, $N_G(C) \subseteq B_u \cap N_G^\rho[v]$.
Also, $N_G(C)$ separates all vertices in~$C$ from all other vertices in~$G$.
Therefore, $x \in C$ and $C' \subseteq C$; otherwise, $P$ would intersect $N_G^\rho[v]$.
It follows that each vertex in~$P$ is in the same connected component of~$G[C] - N_G^\rho[v]$ and, thus, $C_v = C'$.
\end{proof}

From Lemma~\ref{lem:decoBorderBags} (page~\pageref{lem:decoBorderBags}), it directly follows:

\begin{corollary}
    \label{cor:potPartners}
If \( N_G^\rho[u] \) is a bag in~\( T \), then \( T \) contains a bag~\( N_G^\rho[v] \) for each \( C \in \calC_G[u] \) such that \( v \) is a potential partner of~\( u \) for~\( C \).
\end{corollary}

Because of Corollary~\ref{cor:potPartners}, there is a vertex set~$U$ such that each $u \in U$ has a potential partner~$v \in U$ for each connected component~$C \in \calC_G[u]$.
With such a set, we can construct a tree-decomposition for~$G$ with the following approach:
Pick a vertex~$u \in U$ and make it center of a bag~$B_u$.
For each connected component~$C \in \calC_G[u]$, $u$ has a potential partner~$v$.
$N_G^\rho[v]$ splits $C$ in more connected components and, because $v \in U$, $v$ has a potential partner~$w \in U$ for each of these components.
Hence, create a bag~$B_v = N_G^\rho[v] \cap (B_u \cup C)$ and continue this until the whole graph is covered.
Algorithm~\ref{algo:weakTB} below determines such a set of vertices with their potential partners (represented as a graph~$H$) and then constructs a decomposition as described above.

\begin{algorithm}
    [!htb]
    \caption
    {%
        Constructs, for a given graph~$G = (V, E)$ with strong tree-breadth~$\rho$, a tree-decomposition~$T$ with breadth~$\rho$.
    }
    \label{algo:weakTB}

Create an empty directed graph~$H = (V_H, E_H)$.
Let $\phi$ be a function that maps each edge~$(u, v) \in E_H$ to a connected component~$C \in \calC_G[u]$.
\label{line:weakTB_initH}

\ForEach
{%
    \( u, v \in V \) and all \( C \in \calC_G[u] \)
}
{%
    \If
    {%
        \( v \) is a potential parter of~\( u \) for~\( C \)%
        \label{line:weakTB_CheckUV}
    }
    {%
        Add the directed edge~$(u, v)$ to~$H$ and set $\phi(u, v) := C$.
        (Add $u$ and~$v$ to $H$ if necessary.)
        \label{line:weakTB_AddUV}
    }
}

\While
{%
    there is a vertex~\( u \in V_H \) and some \( C \in \calC_G[u] \) such that there is no \( (u, v) \in E_H \) with \( \phi(u, v) = C \)
    \label{line:weakTB_RemoveUloop}
}
{
    Remove $u$ from~$H$.
    \label{line:weakTB_RemoveU}
}

\If
{%
    \( H \) is empty
}
{%
    \keyword{Stop}.
    $\stb(G) > \rho$.
}

Create an empty tree-decomposition~$T$.
\label{line:weakTB_initT}

Let $G - T$ be the subgraph of~$G$ that is not covered by~$T$ and let $\psi$ be a function that maps each connected component in $G - T$ to a bag~$B_u \subseteq N_G^\rho[u]$.

Pick an arbitrary vertex~$u \in V_H$, add $B_u = N_G^\rho[u]$ as bag to~$T$, and set $\psi(C) := B_u$ for each connect component~$C$ in~$G - T$.
\label{line:weakTB_addFirstBag}

\While
{%
    \( G - T \) is non-empty
    \label{line:weakTB_Tloop}
}
{%
    Pick a connected component~$C$ in~$G - T$, determine the bag~$B_v := \psi(C)$ and find an edge~$(v, w) \in E_H$ with $\phi(v, w) = C$.
    \label{line:weakTB_PickNextBag}

    Add $B_w = N_G^\rho[w] \cap (B_v \cup C)$ to~$T$, and make $B_v$ and~$B_w$ adjacent in~$T$.
    \label{line:weakTB_AddBag}

    For each new connected component~$C'$ in~$G - T$ with $C' \subseteq C$, set $\psi(C_w) := B_w$.
    \label{line:weakTB_UpdateF}
}

Output~$T$.
\end{algorithm}

\begin{theorem}
Algorithm~\ref{algo:weakTB} constructs, for a given graph~\( G \) with strong tree-breadth~\( \rho \), a tree-decomposition~\( T \) with breadth~\( \rho \) in \( \calO(n^2m) \) time.
\end{theorem}

\begin{proof}
    [Correctness]
Algorithm~\ref{algo:weakTB} works in two parts.
First, it creates a graph~$H$ with potential centers (line~\ref{line:weakTB_initH} to line~\ref{line:weakTB_RemoveU}).
Second, it uses $H$ to create a tree-decomposition for~$G$ (line~\ref{line:weakTB_initT} to line~\ref{line:weakTB_UpdateF}).
To show the correctness of the algorithm, we show first that the centers of a tree-decomposition for~$G$ are vertices in~$H$ and, then, show that a tree-decomposition created based on~$H$ is a valid tree-decomposition for~$G$.

A vertex~$u$ is added to~$H$ (line~\ref{line:weakTB_AddUV}) if, for \emph{at least one} connected component~$C \in \calC_G[u]$, $u$ has a potential partner~$v$.
Later, $u$ is kept in~$H$ (line~\ref{line:weakTB_RemoveUloop} and line~\ref{line:weakTB_RemoveU}) if it has a potential partner~$v$ \emph{for all} connected components in~$C \in \calC_G[u]$.
By Corollary~\ref{cor:potPartners}, each center of a bag in a tree-decomposition~$T$ with strong breadth~$\rho$ satisfies these conditions.
Therefore, after line~\ref{line:weakTB_RemoveU}, $H$ contains all centers of bags in~$T$, \ie, if $G$ has strong tree-breadth~$\rho$, $H$ is non-empty.

Next, we show that $T$ created in the second part of the algorithm (line~\ref{line:weakTB_initT} to line~\ref{line:weakTB_UpdateF}) is a valid tree-decomposition for~$G$ with breadth~$\rho$.
To do so, we show the following invariant for the loop starting in line~\ref{line:weakTB_Tloop}:
\begin{enumerate*}[(i),mode=unboxed]
    \item
        \label{item:TvalidTB}
        $T$ is a valid tree-decomposition with breadth~$\rho$ for the subgraph covered by~$T$ and
    \item
        \label{item:conCompCcond}
        for each connected component~$C$ in~$G - T$, the bag~$B_v = \psi(C)$ is in~$T$, $N_G(C) \subseteq B_v$, and $C \in \calC_G[v]$.
\end{enumerate*}
After line~\ref{line:weakTB_addFirstBag}, the invariant clearly holds.
Assume by induction that the invariant holds each time line~\ref{line:weakTB_Tloop} is checked.
If $T$ covers the whole graph, the check fails and the algorithm outputs~$T$.
If $T$ does not cover $G$ completely, there is a connected component~$C$ in~$G - T$.
By condition~\ref{item:conCompCcond}, the bag~$B_v = \psi(C)$ is in~$T$, $N_G(C) \subseteq B_v$, and $C \in \calC_G[v]$.
Because of the way $H$ is constructed and $C \in \calC_G[v]$, there is an edge~$(v, w) \in E_H$ with $\phi(v, w) = C$, \ie, $w$ is a potential partner of~$v$ for~$C$.
Thus, line~\ref{line:weakTB_PickNextBag} is successful and the algorithm adds a new bag~$B_w = N_G^\rho[w] \cap (B_v \cup C)$ (line~\ref{line:weakTB_AddBag}).
Because $w$ is a potential partner of~$v$ for~$C$, \ie, $N_G(C) \subseteq N_G^\rho[w]$, and $N_G(C) \subseteq B_v$, $B_w \supseteq N_G(C)$.
Therefore, after adding $B_w$ to~$T$, $T$ still satisfies condition~\ref{item:TvalidTB}.
Additionally, $B_w$ splits $C$ in a set~$\calC'$ of connected components such that, for each $C' \in \calC'$, $N_G(C') \subseteq B_w$ and, by Lemma~\ref{lem:potPartProp}, $C' \in \calC_G[w]$.
Thus, condition~\ref{item:conCompCcond} is also satisfied.
\end{proof}

\begin{proof}
    [Complexity]
First, determine the pairwise distance of all vertices.
This can be done in $\calO(nm)$ time and allows to check the distance between vertices in constant time.

For a vertex~$u$, let $\calN[u] = \{ \, N_G(C) \mid C \in \calC_G[u] \, \}$.
Note that, for some $C \in \calC_G[u]$ and each vertex~$x \in N_G(C)$, there is an edge~$xy$ with $y \in C$.
Therefore, $|\calN[u]| := \sum_{C \in \calC_G[u]} |N_G(C)| \leq m$.
To determine, for some vertex~$u$, all its potential partners~$v$, first, compute $\calN[u]$.
This can be done in $\calO(m)$ time.
Then, check, for each vertex~$v$ and each $N_G(C) \in \calN[u]$, if $N_G(C) \subseteq N_G^\rho[v]$ and add the edge~$(u, v)$ to~$H$ if successful.
For a single vertex~$v$ this requires $\calO(m)$ time because $|\calN[u]| \leq m$ and distances can be determined in constant time.
Therefore, the total runtime for creating~$H$ (line~\ref{line:weakTB_initH} to line~\ref{line:weakTB_AddUV}) is $\calO \big( n (m + nm) \big) = \calO \big( n^2m \big)$.

Assume that, for each $\phi(u, v) = C$, $C$ is represented buy two values:
\begin{enumerate*}[(i),mode=unboxed]
    \item
        \label{item:charVert}
        a characteristic vertex~$x \in C$ (for example the vertex with the lowest index) and
    \item
        \label{item:indexC}
        the index of~$C$ in $\calC_G[u]$.
\end{enumerate*}
While creating~$H$, count and store, for each vertex~$u$ and each connected component~$C \in \calC_G[u]$, the number of edges~$(u, v) \in E_H$ with $\phi(u, v) = C$.
Note that there is a different counter for each $C \in \calC_G[u]$.
With this information, we can implement line~\ref{line:weakTB_RemoveUloop} and line~\ref{line:weakTB_RemoveU} as follows.
First check, for every vertex~$v$ in~$H$, if one of its counters is~$0$.
In this case, remove $v$ from~$H$ and update the counters for all vertices~$u$ with $(u, v) \in E_H$ using value~\ref{item:indexC} of~$\phi(u, v)$.
If this sets a counter for~$u$ to~$0$, add $u$ to a queue~$Q$ of vertices to process.
Continue this until each vertex is checked.
Then, for each vertex~$u$ in~$Q$, remove~$u$ form~$H$ and add its neighbours into~$Q$ if necessary until $Q$ is empty.
This way, a vertex is processed at most twice.
A single iteration runs in at most $\calO(n)$ time.
Therefore, line~\ref{line:weakTB_RemoveUloop} and line~\ref{line:weakTB_RemoveU} can be implemented in $\calO \big( n^2 \big)$ time.

Assume that $\psi$ uses the characteristic vertex~$x$ to represent a connected component, \ie, value~\ref{item:charVert} of $\phi$.
Then, finding an edge $(v, w) \in E_H$ (line~\ref{line:weakTB_PickNextBag}) can be done in $\calO(m)$~time.
Creating $B_w$ (line~\ref{line:weakTB_AddBag}), splitting $C$ into new connected components~$C'$, finding their characteristic vertex, and setting $\psi(C')$ (line~\ref{line:weakTB_UpdateF}) takes $\calO(m)$ time, too.
In each iteration, at least one more vertex of~$G$ is covered by~$T$.
Hence, there are at most $n$ iterations and, thus, the loop starting in line~\ref{line:weakTB_Tloop} runs in $\calO(mn)$ time.

Therefore, Algorithm~\ref{algo:weakTB} runs in total $\calO \big( n^2m \big)$ time.
\end{proof}

Algorithm~\ref{algo:weakTB} creates, for each graph~$G$ with $\stb(G) \leq \rho$, a tree-decomposition~$T$ with breadth~$\rho$.
Next, we invest a case where we can construct a tree-decomposition for~$G$ with strong breadth~$\rho$.

We say that two vertices $u$ and~$v$ are \emph{perfect partners} if
\begin{enumerate*}[(i),mode=unboxed]
    \item
        $u$ and~$v$ are potential partners of each other for some $C_u \in \calC_G[u]$ and some $C_v \in \calC_G[v]$,
    \item
        $C_u$ is the only connected component in~$\calC_G[u]$ which is intersected by~$N_G^\rho[v]$, and
    \item
        $C_v$ is the only connected component in~$\calC_G[v]$ which is intersected by~$N_G^\rho[u]$.
\end{enumerate*}
Accordingly, we say that a tree-decomposition~$T$ has \emph{perfect strong breadth~\( \rho \)} if it has strong breadth~$\rho$ and, for each center~$u$ of some bag and each connected component~$C \in \calC_G[u]$, there is a center~$v$ such that $v$ is a perfect partner of~$u$ for~$C$.

\begin{theorem}
A tree-decomposition with perfect strong breadth~\( \rho \) can be constructed in polynomial time.
\end{theorem}

\begin{proof}
To construct such a tree-decomposition, we can modify Algorithm~\ref{algo:weakTB}.
Instead of checking if $u$ has a potential partner~$v$ (line~\ref{line:weakTB_CheckUV}), check if $u$ and~$v$ are perfect partners.

Assume by induction that, for each bag~$B_v$ in~$T$, $B_v = N_G^\rho[v]$.
By definition of perfect partners $v$ and~$w$, $N_G^\rho[w]$ intersects only one~$C \in \calC_G[v]$, \ie, $N_G^\rho[w] \subseteq N_G^\rho[v] \cup C$.
Thus, when creating the bag~$B_w$ (line~\ref{line:weakTB_AddBag}), $B_w = N_G^\rho[w] \cap (B_v \cup C) = N_G^\rho[w] \cap \big( N_G^\rho[v] \cup C \big) = N_G^\rho[w]$.
Therefore, the created tree-decomposition~$T$ has perfect strong tree-breadth~$\rho$.
\end{proof}

We conjecture that there are weaker cases than perfect strong breadth which allow to construct a tree-decomposition with strong-breadth~$\rho$.
For example, if the centers of two adjacent bags are perfect partners, but a center~$u$ does not need to have a perfect partner for each $C \in \calC_G[u]$.
However, when using a similar approach as in Algorithm~\ref{algo:weakTB}, this would require a more complex way of constructing~$H$.




\section{Computing Decompositions for Special Graph Classes}
    \label{sec:SpecialGraphDecomp}

In this section, we show how to find good decompositions for some special graph classes.

\begin{theorem}
    \label{theo:chordalStb}
Chordal graphs have strong tree-breadth~\( 1 \).
An according decomposition can be computed in linear time.
\end{theorem}

\begin{proof}
Let $\sigma = \langle v_1, v_2, \ldots, v_n \rangle$ be an ordering for the vertices of a graph~$G$, $V_i = \{ v_1, v_2, \ldots, v_i \}$, $G_i$ denote the graph~$G[V_i]$, $N_i[v] = N_G[v] \cap V_i$, and $N_i(v) = N_G(v) \cap V_i$.
The reverse of such an ordering~$\sigma$ is called a \emph{perfect elimination ordering} for~$G$ if, for each~$i$, $N_i[v_i]$ induces a clique.
It is well known that a graph is chordal if and only if it admits a perfect elimination ordering~\cite{Dirac1961}.

Assume that we are given such an ordering~$\sigma$, \ie, the reverse of a perfect elimination ordering.
Additionally, assume by induction over~$i$ that $G_{i-1}$ admits a tree-decomposition~$T_{i-1}$ with strong breadth~$1$ such that centers of bags are pairwise non-adjacent.
This is clearly the case for~$G_1$.
We now show how to construct $T_i$ from~$T_{i-1}$.

First, consider the case that $v_i$ has a neighbour~$u$ in~$G_i$ which is center of some bag~$B$ in~$T_{i-1}$.
Because $u \in N_i[v_i]$ and $N_i[v_i]$ induces a clique, $N_i[v_i] \subseteq N_i[u]$ and, hence, $N_i[v_i]$ does not contain a center of any bag.
Thus, adding $v_i$ into~$B$ creates a valid tree-decomposition~$T_i$ for~$G_i$ with strong breadth~$1$ and pairwise non-adjacent centers.

Next, consider the case that $v_i$ has no neighbour~$u$ in~$G_i$ that is center of some bag in~$T_{i-1}$.
Note that $N_i(v_i)$ induces a clique in~$G_{i-1}$.
It is well known that, for each tree-decomposition~$T$ and for each clique~$K$ of graph, $T$ contains a bag~$B$ with~$K \subseteq B$.
Thus, there is a bag~$B$ in~$T_{i-1}$ with $N_i(v_i) \subseteq B$.
Therefore, we can create $T_i$ by adding the bag~$B' = N_i[v_i]$ to~$T_{i-1}$ and making $B'$ adjacent to~$B$.
This creates a valid tree-decomposition with strong breadth~$1$ and pairwise non-adjacent centers.

Based on the approach described above, we can construct a tree-decomposition with strong breadth~$1$ for a given chordal graph~$G$ as follows.
First, compute the reverse of a perfect elimination ordering of~$G$.
Such an ordering can be computed in linear time~\cite{RoseTarjLuek1976}.
Observe that we can simplify the approach above with the following rule:
If $v_i$ has no neighbour in~$G_i$ which is center of a bag, make $v_i$ center of a bag.
Otherwise, proceed with~$v_{i+1}$.
Therefore, by using a simple binary flag for each vertex, one can compute a tree-decomposition with strong breadth~$1$ for a given chordal graph~$G$ in linear time.
\end{proof}

\begin{theorem}
    \label{theo:DHistb}
Distance-hereditary graphs have strong tree-breadth~\( 1 \).
An according decomposition can be computed in linear time.
\end{theorem}

\begin{proof}
Recall that a graph is distance-hereditary if and only if it admits a pruning sequence~$\sigma$ (see Lemma~\ref{lem:punSeq}, page~\pageref{lem:punSeq}).
That is, a sequence $\sigma = \langle v_1, v_2, \ldots, v_n \rangle$ such that, for each vertex~$v_i$ and some $j < i$, $v_i$ is a pendant vertex, $v_i$ is a true twin of~$v_j$, or $v_i$ is a false twin of~$v_j$.

Assume that we are given such a pruning sequence.
Additionally, assume by induction over~$i$ that $G_i$ has a tree-decomposition~$T_i$ with strong breadth~$1$ where the centers of bags are pairwise non-adjacent.
Then, there are three cases:
\begin{enumerate}[(i)]
    \item
        \emph{\( v_{i+1} \) is a pendant vertex in~\( G_{i+1} \).}
        If the neighbour~$u$ of~$v_{i+1}$ is a center of a bag~$B_u$, add $v_{i+1}$ to~$B_u$.
        Thus, $T_{i+1}$ is a valid decomposition for~$G_{i+1}$.
        Otherwise, if $u$ is not a center, make $v_{i+1}$ center of a bag.
        Because $u$ is an articulation point, $T_{i+1} = T_i + N_G[v]$ is a valid decomposition for~$G_{i+1}$.
    \item
        \emph{\( v_{i+1} \) is a true twin of a vertex~\( u \) in~\( G_{i+1} \).}
        Simply add $v_{i+1}$ into any bag containing~$u$.
        The resulting decomposition is a valid decomposition for~$G_{i+1}$.
    \item
        \emph{\( v_{i+1} \) is a false twin of a vertex~\( u \) in~\( G_{i+1} \).}
        If $u$ is not center of a bag, add $v_{i+1}$ into any bag $u$ is in.
        Otherwise, make a new bag~$B_{i+1} = N_G[v_{i+1}]$ and make it adjacent to the bag~$N_G[u]$.
        Because no vertex in~$N_G(u)$ is center of a bag, the resulting decomposition is a valid decomposition for~$G_{i+1}$.
\end{enumerate}
Therefore, distance-hereditary graphs have strong tree-breadth~$1$.

Next, we show how to compute an according tree-decomposition in linear time.
The argument above already gives an algorithmic approach.
First, we compute a pruning sequence for~$G$.
This can be done in linear time with an algorithm by \name{Damiand} et al.\,\cite{DamiHabiPaul2001}.
Then, we determine which vertex becomes a center of a bag.
Note that we can simplify the three cases above with the following rule:
If $v_i$ has no neighbour in~$G_i$ which is center of a bag, make $v_i$ center of a bag.
Otherwise, proceed with~$v_{i+1}$.
This can be easily implemented in linear time with a binary flag for each vertex.
\end{proof}

Algorithm~\ref{algo:DHtb} formalizes the method described in the proof of Theorem~\ref{theo:DHistb}.

\begin{algorithm}
    [htb]
    \caption
    {%
        Computes, for a given distance-hereditary graph~$G$, a tree-decomposition~$T$ with strong breadth~$1$.
    }
    \label{algo:DHtb}

Compute a pruning sequence $\langle v_1, v_2, \ldots, v_n \rangle$ (see~\cite{DamiHabiPaul2001}).

Create a set~$C := \emptyset$.

\For
{%
    \( i := 1 \) \KwTo \( n \)
}
{%
    \If
    {%
        \( N_G[v_i] \cap V_i \cap C = \emptyset \)
    }
    {%
        Add $v_i$ to $C$.
    }
}

Create a tree-decomposition~$T$ with the vertices in~$C$ as centers of its bags.

\end{algorithm}

\begin{theorem}
    \label{theo:pbATfree}
If \( G \) is an AT-free graph, then \( \pl(G) \leq 2 \).
Furthermore, a path-decomposition of~\( G \) with length at most~\( 2 \) can be computed in \( \calO \big( n^2 \big) \) time.
\end{theorem}

\begin{proof}
Let $s$ be an arbitrary vertex of~$G$, let $x$ be the vertex last visited (numbered~$1$) by a LexBFS starting at~$s$, and let $\sigma$ be the ordering obtained by a LexBFS starting at~$x$.
Clearly, $\sigma$ can be generated in linear time.
Note that the LexBFS which computes~$\sigma$ also computes all distance layers~$L_i^{(x)}$ of~$G$ with $0 \leq i \leq \ecc(x)$.
For some vertex~$v \in L_i^{(x)}$, let $N_G^\downarrow(v) = N_G(v) \cap L_{i-1}^{(x)}$.

We can transform an AT-free graph~$G = (V, E)$ into an interval graph~$G^+ = \big( V, E^+ \big)$ by applying the following two operations:

\begin{enumerate}[(1)]
    \item
        \label{item:ATfreeMakeLayers}
        \emph{Make layers complete graphs.}
        In each layer~$L_i^{(x)}$, make every two vertices~$u, v \in L_i^{(x)}$ adjacent to each other in~$G^+$.
    \item
        \label{item:ATfreeMakeNeighs}
        \emph{Make down-neighbourhoods of adjacent vertices of a layer comparable.}
        For each~$i$ and every edge~$uv$ of~$G$ with $u, v \in L_i^{(x)}$ and $\sigma(v) < \sigma(u)$, make every $w \in N_G^{\downarrow}(v)$ adjacent to~$u$ in~$G^+$.
\end{enumerate}

\begin{claim}
    \label{cla:GplusGsquare}
\( G^+ \) is a subgraph of~\( G^2 \).
\end{claim}

\begin{proof}[Claim]
Clearly, for every edge~$uw$ of~$G^+$ added by operation~\ref{item:ATfreeMakeNeighs}, $d_G(u, w) \leq 2$ holds.
Also, for every edge~$uv$ of~$G^+$ added by operation~\ref{item:ATfreeMakeLayers}, $d_G(u, v) \leq 2$ holds by Lemma~\ref{lem:ATfreeDownNeigh} (page~\pageref{lem:ATfreeDownNeigh}).
\end{proof}

\begin{claim}
    \label{cla:GplusInterval}
\( G^+ \) is an interval graph.
\end{claim}

\begin{proof}
    [Claim]
It is known~\cite{Olariu1991} that a graph is an interval graph if and only if its vertices admit an \emph{interval ordering}.
That is, an ordering $\tau \colon V \rightarrow \{ 1, \ldots, n \}$ such that, for any three vertices $a$, $b$, and~$c$ with $\tau(a) < \tau(b) < \tau(c)$, $ac \in E$ implies $bc \in E$.
We show here that the LexBFS-ordering~$\sigma$ of~$G$ is an interval ordering of~$G^+$.
Recall that, for each~$v \in L_i^{(x)}$ and every~$u \in L_j^{(x)}$ with $i > j$, it holds that $\sigma(v) < \sigma(u)$ since $\sigma$ is a LexBFS-ordering.
Consider three arbitrary vertices $a$, $b$, and~$c$ of~$G$ and assume that $\sigma(a) < \sigma(b) < \sigma(c)$ and $ac \in E^+$.
Assume also that $a \in L_i^{(x)}$ for some~$i$.
If $c$ belongs to~$L_i^{(x)}$, then $b$ must be in $L_i^{(x)}$ as well.
Hence, $bc \in E^+$ due to operation~\ref{item:ATfreeMakeLayers}.
If both $b$ and~$c$ are in $L_{i-1}^{(x)}$, then again $bc \in E^+$ due to operation~\ref{item:ATfreeMakeLayers}.
Consider now the remaining case that $a, b \in L_i^{(x)}$ and $c \in L_{i-1}^{(x)}$.
If $ac \in E$, then $bc \in E^+$ because either $ab \in E$ and, thus, operation~\ref{item:ATfreeMakeNeighs} applies, or $ab \notin E$ and, thus, Lemma~\ref{lem:ATfreeDownNeigh} (page~\pageref{lem:ATfreeDownNeigh}) implies $bc \in E^+$.
If $ac \in E^+ \setminus E$ then, according to operation~\ref{item:ATfreeMakeNeighs}, edge~$ac$ was created in~$G^+$ because some vertex~$a' \in L_i^{(x)}$ existed such that $\sigma(a') < \sigma(a)$ and $a'a, a'c \in E$.
Since $a'c \in E$ and $\sigma(a') < \sigma(b) < \sigma(c)$, as before, $bc \in E^+$ must hold.
\end{proof}

To complete the proof, we recall that a graph is an interval graph if and only if it has a path-decomposition with each bag being a maximal clique (see Theorem~\ref{theo:intervalPL}, page~\pageref{theo:intervalPL}).
Furthermore, such a path-decomposition of an interval graph can easily be computed in linear time.
Let $\calP^+ = \big \{ X_1, \ldots, X_q \big \}$ be a path-decomposition of our interval graph~$G^+$.
Then, $\calP := \calP^+ = \big \{ X_1, \ldots, X_q \big \}$ is a path-decomposition of~$G$ with length at most~$2$ since, for every edge~$uv$ of $G^+$, the distance in $G$ between $u$ and $v$ is at most~$2$, as shown in Claim~\ref{cla:GplusGsquare}.
\end{proof}

Algorithm~\ref{algo:ATfreeDeco} formalizes the steps described in the
previous proof.

\begin{algorithm}
    [htb]
    \caption
    {%
        Computes a path-decomposition of length at most~$2$ for a given AT-free graph.
    }
    \label{algo:ATfreeDeco}

\KwIn
{%
    An AT-free graph $G = (V, E)$.
}

\KwOut
{%
    A path-decomposition of $G$.
}

Calculate a LexBFS-ordering~$\sigma$ of~$G$ with an arbitrary start vertex~$s \in V$.
Let $x$ be the last visited vertex, \ie, $\sigma(x) = 1$.

Calculate a LexBFS-ordering~$\sigma'$ of~$G$ starting at~$x$.

Set $E^+ := E$.

\ForEach
{%
    vertex pair $u, v$ with $d_G(x,u) = d_G(x,v)$ and $\sigma'(u) < \sigma'(v)$
}
{%
    Add $uv$ to~$E^+$.

    For each $w \in N_G(u)$ with $\sigma'(v) < \sigma'(w)$, add $vw$ to~$E^+$.%
}

Calculate a path-decomposition~$\calP$ of the interval graph~$G^+ = (V, E^+)$ by determining the maximal cliques of~$G^+$.

Output $\calP$.
\end{algorithm}

Because the class of cocomparability graphs is a proper subclass of AT-free graphs, we obtain the following corollary.

\begin{corollary}
    \label{cor:plCocompGr}
If \( G \) is a cocomparability graph, then \( \pl(G) \leq 2 \).
Furthermore, a path-decomposition of~\( G \) with length at most~\( 2 \) can be computed in \( \calO \big( n^2 \big) \) time.
\end{corollary}

Note that the complement of an induced cycle on six vertices has path-breadth~$2$ (see~\cite{DragKoehLeit2017} for details).
Thus, the bound~$2$ on the path-breadth of cocomparability graphs (and therefore, of AT-free graphs) is sharp.

% Other classes

Consider two parallel lines (upper and lower) in the plane.
Assume that each line contains $n$~points, labelled $1$ to~$n$.
Each two points with the same label define a segment with that label.
The intersection graph of such a set of segments between two parallel lines is called a \emph{permutation graph}.

Assume now that each of the two parallel lines contains $n$~intervals, labelled $1$ to~$n$, and each two intervals with the same label define a trapezoid with that label (a trapezoid can degenerate to a triangle or to a segment).
The intersection graph of such a set of trapezoids between two parallel lines is called a \emph{trapezoid graph}.
Clearly, every permutation graph is a trapezoid graph, but not vice versa.

\begin{theorem}
    \label{theo:pbPermutGr}
If \( G \) is a permutation graph, then \( \spb(G) = 1 \).
Furthermore, a path-decomposition of~\( G \) with optimal breadth can be computed in linear time.
\end{theorem}

\begin{proof}
We assume that a permutation model of~$G$ is given in advance (if not, we can compute one for~$G$ in linear time~\cite{McConnSpinra1997}).
That is, each vertex $v$ of $G$ is associated with a segment $s(v)$ such that $uv \in E$ if and only if segments $s(v)$ and $s(u)$ intersect.
In what follows, ``u.\,p.'' and ``l.\,p.'' refer to a vertex's point on the upper and lower, respectively, line of the permutation model.

First, we compute an (inclusion) maximal independent set~$M$ of~$G$ in linear time as follows.
Put in~$M$ (which is initially empty) a vertex~$x_1$ whose u.\,p.\ is leftmost.
For each $i \geq 2$, select a vertex~$x_i$ whose u.\,p.\ is leftmost among all vertices whose segments do not intersect $s(x_1), \ldots, s(x_{i - 1})$.
In fact, it is enough to check intersection with $s(x_{i - 1})$ only.
If such a vertex exists, put it in~$M$ and continue.
If no such vertex exists, $M = \{ x_1, \ldots, x_k \}$ has been constructed.

Now, we claim that $\big \{ N_G[x_1], \ldots, N_G[x_k] \big \}$ is a path-decomposition of~$G$ with strong breadth~$1$ and, hence, with length at most~$2$.
Clearly, each vertex of~$G$ is in some bag since every vertex not in~$M$ is adjacent to a vertex in~$M$, by the maximality of~$M$.
Consider an arbitrary edge~$uv$ of~$G$.
Assume that neither $u$ nor~$v$ is in~$M$ and that the u.\,p.\ of~$u$ is to the left of the u.\,p.\ of~$v$.
Necessarily, the l.\,p.\ of~$v$ is to the left of the l.\,p.\ of~$u$, since the segments $s(v)$ and $s(u)$ intersect.
Assume that the u.\,p.\ of~$u$ is between the u.\,p.s of $x_i$ and~$x_{i + 1}$.
From the construction of~$M$, $s(u)$ and~$s(x_i)$ must intersect, \ie, the l.\,p.\ of~$u$ is to the left of the l.\,p.\ of~$x_i$.
But then, since the l.\,p.\ of~$v$ is to the left of the l.\,p.\ of~$x_i$, segments $s(v)$ and $s(x_i)$ must intersect, too.
Thus, edge~$uv$ is in the bag~$N_G[x_i]$.

To show that all bags containing any particular vertex form a contiguous subsequence of the sequence~$\big \langle N_G[x_1], \ldots, N_G[x_k] \big \rangle$, consider an arbitrary vertex~$v$ of~$G$ and let $v \in N_G[x_i] \cap N_G[x_j]$ for $i < j$.
Consider an arbitrary bag~$N_G[x_l]$ with $i < l < j$.
We know that the vertices $x_i, x_l, x_j \in M$ are pairwise non-adjacent.
Furthermore, the segment~$s(v)$ intersects the segments $s(x_i)$ and~$s(x_j)$.
As segment~$s(x_l)$ is between $s(x_i)$ and~$s(x_j)$, necessarily, $s(v)$ intersects $s(x_l)$ as well.
\end{proof}

\begin{theorem}
    \label{theo:pbTrapGr}
If \( G \) is a trapezoid graph, then \( \pb(G) = 1 \).
Furthermore, a path-decomposition of~\( G \) with breadth~\( 1 \) can be computed in \( \calO(n^2) \) time.
\end{theorem}

\begin{proof}
We show that every trapezoid graph~$G$ is a minor of a permutation graph.

First, we compute in $\calO(n^2)$ time a trapezoid model for~$G$~\cite{MaSpinra1994}.
Then, we replace each trapezoid $\calT_i$ in this model with its two diagonals obtaining a permutation model with~$2 n$ vertices.
Let $H$ be the permutation graph of this permutation model.
It is easy to see that two trapezoids $\calT_1$ and~$\calT_2$ intersect if and only if a diagonal of~$\calT_1$ and a diagonal of~$\calT_2$ intersect.

Now, $G$ can be obtained back from~$H$ by a series of $n$~edge contractions.
For each trapezoid~$\calT_i$, contract the edge of~$H$ that corresponds to two diagonals of~$\calT_i$.

Since contracting edges does not increase the path-breadth (Lemma~\ref{lem:decoBorderBags}, page~\pageref{lem:decoBorderBags}), we get $\pb(G) = \pb(H) = 1$ by Theorem~\ref{theo:pbPermutGr}.
Any path-decomposition of~$H$ with breadth~$1$ is a path-decomposition of~$G$ with breadth~$1$.
\end{proof}

A bipartite graph is \emph{chordal bipartite} if each cycle of length at least~$6$ has a chord.
To the best of our knowledge, there is no linear time algorithm known to recognise chordal bipartite graphs.
However, in~\cite{DraganLomono2007}, \name{Dragan} and \name{Lomonosov} show that any chordal bipartite graph~$G = (X, Y, E)$ admits a tree-decomposition with the set of bags $\calB = \big \{ B_1, B_2 , \ldots, B_{|X|} \big \}$, where $B_i = N_G[x_i]$, $x_i \in X$.
Therefore, we can still compute a tree-decomposition in linear time with three steps.
First, compute a 2-colouring.
Second, select a colour and make the neighbourhood of all vertices with this colour bags.
Third, use the algorithm in~\cite{TarjanYannak1984} to check if the selected bags give a valid tree-decomposition.
Thus, it follows:

\begin{theorem}
    [\name{Dragan} and \name{Lomonosov}~\cite{DraganLomono2007}]
Each chordal bipartite graph has strong tree-breadth~\( 1 \).
An according tree-decomposition can be found in linear time.
\end{theorem}

