\chapter{Connected $r$-Domination}
    \label{cha:conDomination}
%

\DeclareRobustCommand{\rHrt}
{%
    \ifmmode
        \text{\boldmath (\raisebox{-0.25ex}{$\heartsuit$})}~%
    \else%
        {\boldmath (\raisebox{-0.25ex}{$\heartsuit$})}\xspace%
    \fi%
}

\DeclareRobustCommand{\rDmd}
{%
    \ifmmode%
        \text{\boldmath ($\diamondsuit$)}~%
    \else%
        {\boldmath ($\diamondsuit$)}\xspace%
    \fi%
}

The Domination problem is a classical problem in computer science.
It is a variant of the Set Covering problem, one of \name{Karp}'s 21 NP-complete problems~\cite{Karp1972}.
A generalisation of it is the $r$-Domination problem.
Assume that we are given a graph~$G = (V, E)$ and a function~$r \colon V \rightarrow \bbN$.
Then, we say a vertex set~$D$ is an \emph{\( r \)-dominating set} for~$G$ if, for each vertex~$u \in V$, $d(u, D) \leq r(u)$.
The \emph{\( r \)-Domination} problem asks for the smallest $r$-dominating set~$D$.

Clearly, the problem is NP-complete in general.
Although it can be solved for dually chordal graphs in linear time~\cite{BranChepDrag1998}, it remains NP-complete for chordal graphs~\cite{BoothJohnso1982}.
Even more, under reasonable assumptions, the $r$-Domination problem cannot be approximated within a factor of $(1-\varepsilon) \ln n$ in polynomial time for chordal graphs~\cite{ChlebiChlebi2008} and, thus, for graphs with bounded tree-breadth.

In this chapter, we use an approach presented by \name{Chepoi} and \name{Estellon}~\cite{ChepoiEstell2007}.
Assume that $D_r$ is a minimum $r$-dominating set for some graph~$G$.
Instead of finding an $r$-dominating set which is slightly larger than~$D_r$,
we investigate how to determine a set~$D$ such that, for some factor~$\phi \in \calO \big( \! \tb(G) \big)$, $D$ is an $(r + \phi)$-dominating set for~$G$ with $|D| \leq |D_r|$.
That is, for each vertex~$v$ of~$G$, $d(v, D) \leq r(v) + \phi$.

\name{Chepoi} and \name{Estellon}~\cite{ChepoiEstell2007} present a polynomial time algorithm to compute an $(r + 2 \delta)$-dominating set for $\delta$-hyperbolic graphs.
Due to Theorem~\ref{theo:hyperbolicTreeLength} (page~\ref{theo:hyperbolicTreeLength}), their algorithm gives an $(r + 2 \lambda)$-dominating set for graphs with tree-length~$\lambda$.
We slightly improve their result by constructing an $(r + \rho)$-dominating set for a graph~$G$ under the assumption that a tree-decomposition for~$G$ with breadth~$\rho$ is given.
Additionally, we present algorithms to compute connected $(r + \phi)$-dominating sets for different values of~$\phi$ and with different runtimes.

For this chapter, let $|T|$ denote the number of nodes and $\Lambda(T)$ denote the number of leaves of a tree~$T$.
If $T$ contains only one node, let $\Lambda(T) := 0$.
With $\alpha$, we denote the inverse \name{Ackermann} function (see, \eg,~\cite{CorLeiRivSte2009}).

\section{Using a Layering Partition}

For the remainder of this section, assume that we are given a graph~$G = (V, E)$ and a layering partition~$\calT$ of~$G$ for an arbitrary start vertex.
We denote the largest diameter of all clusters of~$\calT$ as~$\Delta$, \ie, $\Delta := \max \big \{ \, d_G(x, y) \mid \text{$x, y$ are in a cluster~$C$ of~$\calT$} \, \big \}$.

Theorem~\ref{theo:rDeltaDom} below shows that we can use the layering partition~$\calT$ to compute an $(r + \Delta)$-dominating set for~$G$ in linear time which is not larger than a minimum $r$-dominating set for~$G$.
This is done by finding a minimum $r$-dominating set of~$\calT$ where, for each cluster~$C$ of~$\calT$, $r(C)$ is defined as~$\min_{v \in C} r(v)$.

\begin{theorem}
    \label{theo:rDeltaDom}
Let \( D \) be a minimum \( r \)-dominating set for a given graph~\( G \).
An \( (r + \Delta) \)-dominating set~\( D' \) for~\( G \) with \( |D'| \leq |D| \) can be computed in linear time.
\end{theorem}

\begin{proof}
First, create a layering partition~$\calT$ of~$G$ and, for each cluster~$C$ of~$\calT$, set $r(C) := \min_{v \in C} r(v)$.
Second, find a minimum $r$-dominating set~$\calS$ for~$\calT$, \ie, a set~$\calS$ of clusters such that, for each cluster~$C$ of~$\calT$, $d_\calT(C, \calS) \leq r(C)$.
Third, create a set~$D'$ by picking an arbitrary vertex of~$G$ from each cluster in~$\calS$.
All three steps can be performed in linear time, including the computation of~$\calS$ (see~\cite{BranChepDrag1998}).

Next, we show that $D'$ is an $(r + \Delta)$-dominating set for~$G$.
By construction of~$\calS$, each cluster~$C$ of~$\calT$ has distance at most~$r(C)$ to~$\calS$ in~$\calT$.
Thus, for each vertex~$u$ of~$G$, $\calS$ contains a cluster~$C_\calS$ with $d_\calT(u, C_\calS) \leq r(u)$.
Additionally, by Lemma~\ref{lem:LayPartVertDist} (page~\pageref{lem:LayPartVertDist}), $d_G(u, v) \leq r(u) + \Delta$ for any vertex~$v \in C_\calS$.
Therefore, for any vertex~$u$, $d_G(u, D') \leq r(u) + \Delta$, \ie, $D'$ is an $(r + \Delta)$-dominating set for~$G$.

It remains to show that $|D'| \leq |D|$.
Let $\calD$ be the set of clusters of~$\calT$ that contain a vertex of~$D$.
Because $D$ is an $r$-dominating set for~$G$, it follows from Lemma~\ref{lem:LayPartVertDist} (page~\pageref{lem:LayPartVertDist}) that $\calD$ is an $r$-dominating set for~$\calT$.
Clearly, since clusters are pairwise disjoint, $|\calD| \leq |D|$.
By minimality of~$\calS$, $|\calS| \leq |\calD|$ and, by construction of~$D'$, $|D'| = |\calS|$.
Therefore, $|D'| \leq |D|$.
\end{proof}

We now show how to construct a connected $(r + 2 \Delta)$-dominating set for~$G$ using~$\calT$ in such a way that the set created is not larger than a minimum connected $r$-dominating set for~$G$.
For the remainder of this section, let $D_r$ be a minimum connected $r$-dominating set of~$G$ and let, for each cluster~$C$ of~$\calT$, $r(C)$ be defined as above.
Additionally, we say that a subtree~$T'$ of some tree~$T$ is an \emph{\( r \)-dominating subtree of~\( T \)} if the nodes (clusters in case of a layering partition) of~$T'$ form a connected $r$-dominating set for~$T$.

The first step of our approach is to construct a minimum $r$-dominating subtree~$T_r$ of~$\calT$.
Such a subtree~$T_r$ can be computed in linear time~\cite{Dragan1993}.
Lemma~\ref{lem:TrCardinality} below shows that $T_r$ gives a lower bound for the cardinality of~$D_r$.

\begin{lemma}
    \label{lem:TrCardinality}
If \( T_r \) contains more than one cluster, each connected \( r \)-dominating set of~\( G \) intersects all clusters of~\( T_r \).
Therefore, \( |T_r| \leq |D_r| \).
\end{lemma}

\begin{proof}
Let $D$ be an arbitrary connected $r$-dominating set of~$G$.
Assume that $T_r$ has a cluster~$C$ such that $C \cap D = \emptyset$.
Because $D$ is connected, the subtree of~$\calT$ induced by the clusters intersecting~$D$ is connected, too.
Thus, if $D$ intersects all leafs of~$T_r$, then it intersects all clusters of~$T_r$.
Hence, we can assume, without loss of generality, that $C$ is a leaf of~$T_r$.
Because $T_r$ has at least two clusters and by minimality of~$T_r$, $\calT$~contains a cluster~$C'$ such that $d_\calT(C', C) = d_\calT(C', T_r) = r(C')$.
Note that each path in~$G$ from a vertex in~$C'$ to a vertex in~$D$ intersects~$C$.
Therefore, by Lemma~\ref{lem:LayPartVertDist} (page~\pageref{lem:LayPartVertDist}), there is a vertex~$u \in C'$ with $r(u) = d_\calT(u, C) < d_\calT(u, D) \leq d_G(u, D)$.
That contradicts with $D$ being an $r$-dominating set.

Because any $r$-dominating set of~$G$ intersects each cluster of~$T_r$ and because these clusters are pairwise disjoint, it follows that $|T_r| \leq |D_r|$.
\end{proof}

As we show later in Corollary~\ref{cor:TdeltaDomSet}, each connected vertex set~$S \subseteq V$ that intersects each cluster of~$T_r$ gives an $(r + \Delta)$-dominating set for~$G$.
It follows from Lemma~\ref{lem:TrCardinality} that, if such a set~$S$ has minimum cardinality, $|S| \leq |D_r|$.
However, finding a minimum cardinality connected set intersecting each cluster of a layering partition (or of a subtree of it) is as hard as finding a minimum Steiner tree.

The main idea of our approach is to construct a minimum $(r + \delta)$-dominating subtree~$T_\delta$ of~$\calT$ for some integer~$\delta$.
We then compute a small enough connected set~$S_\delta$ that intersects all cluster of~$T_\delta$.
By trying different values of~$\delta$, we eventually construct a connected set~$S_\delta$ such that $|S_\delta| \leq |T_r|$ and, thus, $|S_\delta| \leq |D_r|$.
Additionally, we show that $S_\delta$ is a connected $(r + 2 \Delta)$-dominating set of~$G$.

For some non-negative integer~$\delta$, let $T_\delta$ be a minimum $(r + \delta)$-dominating subtree of~$\calT$.
Clearly, $T_0 = T_r$.
The following two lemmas set an upper bound for the maximum distance of a vertex of~$G$ to a vertex in a cluster of~$T_\delta$ and for the size of~$T_\delta$ compared to the size of~$T_r$.

\begin{lemma}
    \label{lem:TdeltaDist}
For each vertex~\( v \) of~\( G \), \( d_\calT(v, T_\delta) \leq r(v) + \delta \).
\end{lemma}

\begin{proof}
Let $C_v$ be the cluster of~$\calT$ containing~$v$ and let $C$ be the cluster of~$T_\delta$ closest to~$C_v$ in~$\calT$.
By construction of~$T_\delta$, $d_\calT(v, C) = d_\calT(C_v, C) \leq r(C_v) + \delta \leq r(v) + \delta$.
\end{proof}

Because the diameter of each cluster is at most~$\Delta$, Lemma~\ref{lem:LayPartVertDist} (page~\pageref{lem:LayPartVertDist}) and Lemma~\ref{lem:TdeltaDist} imply the following.

\begin{corollary}
    \label{cor:TdeltaDomSet}
If a vertex set intersects all clusters of~\( T_\delta \), it is an \( \big( r + (\delta + \Delta) \big) \)-dominating set of~\( G \).
\end{corollary}

\begin{lemma}
    \label{lem:TdeltaTrCardinality}
\( |T_{\delta}| \leq |T_r| - \delta \cdot \Lambda(T_\delta) \).
\end{lemma}

\begin{proof}
First, consider the case when $T_\delta$ contains only one cluster, \ie, $|T_\delta| = 1$.
Then, $\Lambda(T_\delta) = 0$ and, thus, the statement clearly holds.
Next, let $T_\delta$ contain more than one cluster, let $C_u$ be an arbitrary leaf of~$T_\delta$, and let $C_v$ be a cluster of~$T_r$ with maximum distance to~$C_u$ such that $C_u$ is the only cluster on the shortest path from $C_u$ to~$C_v$ in~$T_r$, \ie, $C_v$ is not in~$T_\delta$.
Due to the minimality of~$T_\delta$, $d_{T_r}(C_u, C_v) = \delta$.
Thus, the shortest path from $C_u$ to~$C_v$ in~$T_r$ contains $\delta$~clusters (including~$C_v$) which are not in~$T_\delta$.
Therefore, $|T_{\delta}| \leq |T_r| - \delta \cdot \Lambda(T_\delta)$.
\end{proof}

Now that we have constructed and analysed~$T_\delta$, we show how to construct~$S_\delta$.
First, we construct a set of shortest paths such that each cluster of~$T_\delta$ is intersected by exactly one path.
Second, we connect these paths with each other to from a connected set using an approach which is similar to \name{Kruskal}'s algorithm for minimum spanning trees.

Let $\calL = \big \{ C_1, C_2, \ldots, C_\lambda \big \}$ be the leaf clusters of~$T_\delta$ (excluding the root) with either $\lambda = \Lambda(T_\delta) - 1$ if the root of~$T_\delta$ is a leaf, or with  $\lambda = \Lambda(T_\delta)$ otherwise.
We construct a set~$\calP = \big \{ P_1, P_2, \ldots, P_\lambda \big \}$ of paths as follows.
Initially, $\calP$ is empty.
For each cluster~$C_i \in \calL$, in turn, find the ancestor~$C_i'$ of~$C_i$ which is closest to the root of~$T_\delta$ and does not intersect any path in~$\calP$ yet.
If we assume that the indices of the clusters in~$\calL$ represent the order in which they are processed, then $C_1'$ is the root of~$T_\delta$.
Then, select an arbitrary vertex~$v$ in~$\calC_i$ and find a shortest path~$P_i$ in~$G$ form~$v$ to~$C_i'$.
Add $P_i$ to~$\calP$ and continue with the next cluster in~$\calL$.
Figure~\ref{fig:LayPartPaths} gives an example.

\begin{figure}
    [htb]
    \centering
    \tikzsetnextfilename{fig_LayPartPaths}
\begin{tikzpicture}
    [
        xscale=1.2,
        yscale=-0.8660,
    ]
%

\node[nN] (n01) at (3,0) {};
\node[nN] (n02) at (4,0) {};
\node[nN] (n03) at (5,0) {};

\node[nN] (n11) at (2.5,1) {};
\node[nN] (n12) at (3.5,1) {};
\node[nN] (n13) at (4.5,1) {};
\node[nN] (n14) at (5.5,1) {};

\node[nN] (n21) at (1.5,2) {};
\node[nN] (n22) at (2.5,2) {};
\node[nN] (n23) at (3.5,2) {};
\node[nN] (n24) at (4.5,2) {};
\node[nN] (n25) at (5.5,2) {};
\node[nN] (n26) at (6.5,2) {};

\node[nN] (n31) at (1,3) {};
\node[nN] (n32) at (2,3) {};
\node[nN] (n33) at (3,3) {};
\node[nN] (n34) at (4,3) {};
\node[nN] (n35) at (5,3) {};
\node[nN] (n36) at (6,3) {};
\node[nN] (n37) at (7,3) {};

\begin{pgfonlayer}
    {background}

    \begin{scope}
        [
            every node/.style=
            {
                fill=clLight60Green!30,
                draw=clDark25Green,
                minimum width=0.75cm,
            },
            every fit/.append style=text badly centered
        ]

        \node[fit=(n01)(n02)(n03)] (c01) {};

        \node[fit=(n11)(n12)] (c11) {};
        \node[fit=(n13)(n14)] (c12) {};

        \node[fit=(n21)(n22)] (c21) {};
        \node[fit=(n23)(n24)] (c22) {};
        \node[fit=(n25)(n26)] (c23) {};

        \node[fit=(n31)(n32)] (c31) {};
        \node[fit=(n33)] (c32) {};
        \node[fit=(n34)] (c33) {};
        \node[fit=(n35)(n36)] (c34) {};
        \node[fit=(n37)] (c35) {};

    \end{scope}

    \draw (n01.center) -- (n02.center);
    \draw (n01.center) -- (n11.center);

    \draw (n02.center) -- (n03.center);
    \draw (n02.center) -- (n12.center);
    \draw (n02.center) -- (n13.center);

    \draw (n03.center) -- (n14.center);

    \draw (n11.center) -- (n21.center);
    \draw (n11.center) -- (n22.center);

    \draw (n12.center) -- (n22.center);

    \draw (n13.center) -- (n14.center);
    \draw (n13.center) -- (n23.center);
    \draw (n13.center) -- (n24.center);

    \draw (n14.center) -- (n25.center);
    \draw (n14.center) -- (n26.center);

    \draw (n21.center) -- (n31.center);
    \draw (n21.center) -- (n32.center);

    \draw (n22.center) -- (n32.center);
    \draw (n22.center) -- (n33.center);

    \draw (n23.center) -- (n34.center);

    \draw (n24.center) -- (n34.center);

    \draw (n25.center) -- (n35.center);

    \draw (n26.center) -- (n36.center);
    \draw (n26.center) -- (n37.center);

    \draw (n31.center) -- (n32.center);

    \draw (n35.center) -- (n36.center);

\end{pgfonlayer}

\def\pathCol{red!85!black}
\draw[very thick,\pathCol]
    (n31.center) --
    (n21.center) --
    (n11.center) --
    (n01.center);

\draw[very thick,\pathCol]
    (n34.center) --
    (n23.center) --
    (n13.center);

\draw[very thick,\pathCol]
    (n35.center) --
    (n25.center);

\begin{scope}
    [
        every node/.style=
        {
            nN,
            very thick,
            fill=\pathCol!30,
            draw=\pathCol,
        },
    ]
%

\node[] at (n01) {};

\node[] at (n11) {};
\node[] at (n13) {};

\node[] at (n21) {};
\node[] at (n23) {};
\node[] at (n25) {};

\node[] at (n31) {};
\node[] at (n33) {};
\node[] at (n34) {};
\node[] at (n35) {};
\node[] at (n37) {};

\end{scope}

\begin{pgfonlayer}
    {lowerBG}

\end{pgfonlayer}

\node [blbl] at (c31.south) {$C_1$};
\node [blbl] at (c32.south) {$C_2 = C_2'$};
\node [blbl] at (c33.south) {$C_3$};
\node [blbl] at (c34.south) {$C_4$};
\node [blbl] at (c35.south) {$C_5 = C_5'$};

\node [rlbl] at (c01.east) {$C_1'$};
\node [rlbl] at (c12.east) {$C_3'$};
\node [rlbl] at (c23.east) {$C_4'$};

\end{tikzpicture}


    \caption
    [%
        Example for the set~$\calP$ for a subtree of a layering partition.
    ]
    {%
        Example for the set~$\calP$ for a subtree of a layering partition.
        Paths are shown in red.
        Each path~$P_i$, with $1 \leq i \leq 5$, starts in the leaf~$C_i$ and ends in the cluster~$C_i'$.
        For $i = 2$ and $i = 5$, $P_i$ contains only one vertex.
    }
    \label{fig:LayPartPaths}
\end{figure}


\begin{lemma}
    \label{lem:clusterPath}
For each cluster~\( C \) of~\( T_\delta \), there is exactly one path~\( P_i \in \calP \) intersecting~\( C \).
Additionally, \( C \) and~\( P_i \) share exactly one vertex, \ie, \( |C \cap P_i| = 1 \).
\end{lemma}

\begin{proof}
Observe that, by construction of a layering partition, each vertex in a cluster~$C$ is adjacent to some vertex in the parent cluster of~$C$.
Therefore, a shortest path~$P$ in~$G$ from~$C$ to any of its ancestors~$C'$ only intersects clusters on the path from $C$ to~$C'$ in~$\calT$ and each cluster shares only one vertex with~$P$.
It remains to show that each cluster intersects exactly one path.

Without loss of generality, assume that the indices of clusters in~$\calL$ and paths in~$\calP$ represent the order in which they are processed and created, \ie, assume that the algorithms first creates $P_1$ which starts in~$C_1$, then $P_2$ which starts in~$C_2$, and so on.
Additionally, let $\calL_i = \{ C_1, C_2, \ldots, C_i \}$ and $\calP_i = \{ P_1, P_2, \ldots, P_i \}$.

To proof that each cluster intersects exactly one path, we show by induction over~$i$ that, if a cluster~$C_i$ of~$T_\delta$ satisfies the statement, then all ancestors of~$C_i$ satisfy it, too.
Thus, if $C_\lambda$ satisfies the statement, each cluster satisfies it.

First, consider $i = 1$.
Clearly, since $P_1$ is the first path, $P_1$ connects the leaf~$C_1$ with the root of~$T_\delta$ and no cluster intersects more than one path at this point.
Therefore, the statement is true for~$C_1$ and each of its ancestors.

Next, assume that $i > 1$ and that the statement is true for each cluster in~$\calL_{i - 1}$ and their respective ancestors.
Then, the algorithm creates $P_i$ which connects the leaf~$C_i$ with the cluster~$C'_i$.
Assume that there is a cluster~$C$ on the path from $C_i$ to~$C_i'$ in~$\calT$ such that $C$ intersects a path~$P_j$ with $j < i$.
Clearly, $C_i'$ is an ancestor of~$C$.
Thus, by induction hypothesis, $C_i'$ is also intersected by some path~$P \neq P_i$.
This contradicts with the way $C_i'$ is selected by the algorithm.
Therefore, each cluster on the path from $C_i$ to~$C_i'$ in~$\calT$ only intersects $P_i$ and $P_i$ does not intersect any other clusters.

Because $i > 1$, $C'_i$ has a parent cluster~$C''$ in~$T_\delta$ that is intersected by a path~$P_j$ with $j < i$.
By induction hypothesis, each ancestor of~$C''$ is intersected by a path in~$\calP_{i - 1}$.
Therefore, each ancestor of~$C_i$ is intersected by exactly one path in~$\calP_i$.
\end{proof}

Next, we use the paths in~$\calP$ to create the set~$S_\delta$.
As first step, let $S_\delta := \bigcup_{P_i \in \calP} P_i$.
Later, we add more vertices into~$S_\delta$ to ensure it is a connected set.

Now, create a partition~$\calV = \big \{ V_1, V_2, \ldots, V_\lambda \big \}$ of~$V$ such that, for each~$i$, $P_i \subseteq V_i$, $V_i$ is connected, and $d_G(v, P_i) = \min_{P \in \calP} d_G(v, P)$ for each vertex~$v \in V_i$.
That is, $V_i$ contains the vertices of~$G$ which are not more distant to~$P_i$ in~$G$ than to any other path in~$\calP$.
Additionally, for each vertex~$v \in V$, set $P(v) := P_i$ if and only if $v \in V_i$ (\ie, $P(v)$ is the path in~$\calP$ which is closest to~$v$) and set~$d(v) := d_G \big( v, P(v) \big)$.
Such a partition as well as $P(v)$ and~$d(v)$ can be computed by performing a BFS on~$G$ starting at all paths~$P_i \in \calP$ simultaneously.
Later, the BFS also allows us to easily determine the shortest path from $v$ to~$P(v)$ for each vertex~$v$.

To manage the subsets of~$\calV$, we use a Union-Find data structure such that, for two vertices $u$ and~$v$, $\mathrm{Find}(u) = \mathrm{Find}(v)$ if and only if $u$ and~$v$ are in the same set of~$\calV$.
A Union-Find data structure additionally allows us to easily join two set of~$\calV$ into one by performing a single $\mathrm{Union}$ operation.
Note that, whenever we join two sets of~$\calV$ into one, $P(v)$ and~$d(v)$ remain unchanged for each vertex~$v$.

Next, create an edge set~$E' = \{ \, uv \mid \mathrm{Find}(u) \neq \mathrm{Find}(v) \, \}$, \ie, the set of edges~$uv$ such that $u$ and~$v$ are in different sets of~$\calV$.
Sort $E'$ in such a way that an edge~$uv$ precedes an edge~$xy$ only if $d(u) + d(v) \leq d(x) + d(y)$.

The last step to create $S_\delta$ is similar to \name{Kruskal}'s minimum spanning tree algorithm.
Iterate over the edges in~$E'$ in increasing order.
If, for an edge~$uv$, $\mathrm{Find}(u) \neq \mathrm{Find}(v)$, \ie, if $u$ and~$v$ are in different sets of~$\calV$, then join these sets into one by performing $\mathrm{Union}(u, v)$, add the vertices on the shortest path from $u$ to~$P(u)$ to~$S_\delta$, and add the vertices on the shortest path from $v$ to~$P(v)$ to~$S_\delta$.
Repeat this, until $\calV$ contains only one set, \ie, until $\calV = \{ V \}$.

Algorithm~\ref{algo:Tdelta} below summarises the steps to create a set~$S_\delta$ for a given subtree of a layering partition subtree~$T_\delta$.

\begin{algorithm}
    [!htb]
    \caption
    {%
        Computes a connected vertex set that intersects each cluster of a given layering partition.
    }
    \label{algo:Tdelta}

\KwIn
{%
    A graph~$G = (V, E)$ and a subtree~$T_\delta$ of some layering partition of~$G$.
}

\KwOut
{%
    A connected set~$S_\delta \subseteq V$ that intersects each cluster of~$T_\delta$ and contains at most $|T_\delta| + \big( \Lambda(T_\delta) - 1 \big) \cdot \Delta$ vertices.
}

Let $\calL = \big \{ C_1, C_2, \ldots, C_\lambda \big \}$ be the set of clusters excluding the root that are leaves of~$T_\delta$.

Create an empty set~$\calP$.
\label{line:createEmptyP}

\ForEach
{%
    cluster~\( C_i \in \calL \)
}
{%
    Select an arbitrary vertex~$v \in C_i$.

    Find the highest ancestor~$C_i'$ of~$C_i$ (\ie, the ancestor which is closest to the root of~$T_\delta$) that is not flagged.

    Find a shortest path~$P_i$ from~$v$ to an ancestor of~$v$ in~$C_i'$ (\ie, a shortest path from $C_i$ to~$C_i'$ in~$G$ that contains exactly one vertex of each cluster of the corresponding path in~$T_\delta$).

    Add $P_i$ to~$\calP$.

    Flag each cluster intersected by~$P_i$.
    \label{line:flagPiClusters}
}

Create a set $S_\delta := \bigcup_{P_i \in \calP} P_i$.
\label{line:addPtoS}

Perform a BFS on~$G$ starting at all paths~$P_i \in \calP$ simultaneously.
This results in a partition $\calV = \big \{ V_1, V_2, \ldots, V_\lambda \big \}$ of~$V$ with $P_i \subseteq V_i$ for each $P_i \in \calP$.
For each vertex~$v$, set $P(v) := P_i$ if and only if $v \in V_i$ and let $d(v) := d_G(v, P(v))$.
\label{line:performBFS}

Create a Union-Find data structure and add all vertices of~$G$ such that $\mathrm{Find}(v) = i$ if and only if $v \in V_i$.
\label{line:initUnionFind}

Determine the edge set~$E' = \{ \, uv \mid \mathrm{Find}(u) \neq \mathrm{Find}(v) \, \}$.
\label{line:determineEprime}

Sort $E'$ such that $uv \leq xy$ if and only if $d(u) + d(v) \leq d(x) + d(y)$.
Let $\langle e_1, e_2, \ldots, e_{|E'|} \rangle$ be the resulting sequence.
\label{line:sortEprime}

\For
{%
    \( i := 1 \) \KwTo \( |E'| \)%
    \label{line:EprimeLoop}
}
{%
    Let $uv = e_i$.

    \If
    {%
        \( \mathrm{Find}(u) \neq \mathrm{Find}(v) \)
    }
    {%
        Add the shortest path from $u$ to~$P(u)$ to~$S_\delta$.
        \label{line:addPuToS}

        Add the shortest path from $v$ to~$P(v)$ to~$S_\delta$.
        \label{line:addPvToS}

        $\mathrm{Union}(u, v)$
        \label{line:unionSets}
    }
}

Output~$S_\delta$.
\end{algorithm}

\begin{lemma}
    \label{lem:SdeltaCardinality}
For a given graph~\( G \) and a given subtree~\( T_\delta \) of some layering partition of~\( G \), Algorithm~\ref{algo:Tdelta} constructs, in \( \calO \big( m \, \alpha(n) \big) \)~time, a connected set~\( S_\delta \) with \( |S_\delta| \leq |T_\delta| + \Delta \cdot \Lambda(T_\delta) \) which intersects each cluster of~\( T_\delta \).
\end{lemma}

\begin{proof}
    [Correctness]
First, we show that $S_\delta$ is connected at the end of the algorithm.
To do so, we show by induction that, at any time, $S_\delta \cap V'$ is a connected set for each set~$V' \in \calV$.
Clearly, when $\calV$ is created, for each set~$V_i \in \calV$, $S_\delta \cap V_i = P_i$.
Now, assume that the algorithm joins the set $V_u$ and~$V_v$ in~$\calV$ into one set based on the edge~$uv$ with $u \in V_u$ and $v \in V_v$.
Let $S_u = S_\delta \cap V_u$ and $S_v = S_\delta \cap V_v$.
Note that $P(u) \subseteq S_u$ and $P(v) \subseteq S_v$.
The algorithm now adds all vertices to~$S_\delta$ which are on a path from $P(u)$ to~$P(v)$.
Therefore, $S_\delta \cap (V_u \cup V_v)$ is a connected set.
Because $\calV = \{ V \}$ at the end of the algorithm, $S_\delta$ is connected eventually.
Additionally, since $P_i \subseteq S_\delta$ for each $P_i \in \calP$, it follows that $S_\delta$ intersects each cluster of~$T_\delta$.

Next, we show that the cardinality of $S_\delta$ is at most $|T_\delta| + \Delta \cdot \Lambda(T_\delta)$.
When first created, the set~$S_\delta$ contains all vertices of all paths in~$\calP$.
Therefore, by Lemma~\ref{lem:clusterPath}, $|S_\delta| = \sum_{P_i \in \calP} |P_i| = |T_\delta|$.
Then, each time two sets of~$\calV$ are joined into one set based on an edge~$uv$, $S_\delta$ is extended by the vertices on the shortest paths from $u$ to~$P(u)$ and from $v$ to~$P(v)$.
Therefore, the size of~$S_\delta$ increases by $d(u) + d(v)$, \ie, $|S_\delta| := |S_\delta| + d(u) + d(v)$.
Let $X$ denote the set of all edges used to join two sets of~$\calV$ into one at some point during the algorithm.
Note that $|X| = |\calP| - 1 \leq \Lambda(T_\delta)$.
Therefore, at the end of the algorithm,
\[
    |S_\delta|
        = \sum_{\mathclap{P_i \in \calP}} |P_i| + \sum_{\mathclap{uv \in X}} \big( d(u) + d(v) \big)
        \leq |T_\delta| + \Lambda(T_\delta) \cdot \max_{\mathclap{uv \in X}} \big( d(u) + d(v) \big).
\]

\begin{claim}
For each edge~\( uv \in X \), \( d(u) + d(v) \leq \Delta \).
\end{claim}

\begin{proof}[Claim]
To represent the relations between paths in~$\calP$ and vertex sets in~$\calV$, we define a function~$f \colon \calP \rightarrow \calV$ such that $f(P_i) = V_j$ if and only if $P_i \subseteq V_j$.
Directly after constructing~$\calV$, $f$ is a bijection with $f(P_i) = V_i$.
At the end of the algorithm, after all sets of~$\calV$ are joined into one, $f(P_i) = V$ for all $P_i \in \calP$.

Recall the construction of~$\calP$ and assume that the indices of the paths in~$\calP$ represent the order in which they are created.
Assume that $i > 1$.
By construction, the path~$P_i \in \calP$ connects the leaf~$C_i$ with the cluster~$C'_i$ in~$T_\delta$.
Because $i > 1$, $C'_i$ has a parent cluster in~$T_\delta$ that is intersected by a path~$P_j \in \calP$ with $j < i$.
We define~$P_j$ as the \emph{parent} of~$P_i$.
By Lemma~\ref{lem:clusterPath}, this parent~$P_j$ is unique for each $P_i \in \calP$ with~$i > 1$.
Based on this relation between paths in~$\calP$, we can construct a rooted tree~$\bbT$ with the node set~$\{ \, x_i \mid P_i \in \calP \, \}$ such that each node~$x_i$ represents the path~$P_i$ and $x_j$ is the parent of~$x_i$ if and only if $P_j$ is the parent of~$P_i$.

Because each node of~$\bbT$ represents a path in~$\calP$, $f$~defines a colouring for the nodes of~$\bbT$ such that $x_i$ and~$x_j$ have different colours if and only if $f(P_i) \neq f(P_j)$.
As long as $|\calV| > 1$, $\bbT$ contains two adjacent nodes with different colours.
Let $x_i$ and~$x_j$ be these nodes with $j < i$ and let $P_i$ and~$P_j$ be the corresponding paths in~$\calP$.
Note that $x_j$ is the parent of~$x_i$ in~$\bbT$ and, hence, $P_j$ is the parent of~$P_i$.
Therefore, $P_i$ ends in a cluster~$C'_i$ which has a parent cluster~$C$ that intersects~$P_j$.
By properties of layering partitions, it follows that $d_G(P_i, P_j) \leq \Delta + 1$.
Recall that, by construction, $d(v) = \min_{P \in \calP} d_G(v, P)$ for each vertex~$v$.
Thus, for each edge~$uv$ on a shortest path from $P_i$ to~$P_j$ in~$G$ (with $u$ being closer to~$P_i$ than to~$P_j$), $d(u) + d(v) \leq d_G(u, P_i) + d_G(v, P_j) \leq \Delta$.
Therefore, because $f(P_i) \neq f(P_j)$, there is an edge~$uv$ on a shortest path from $P_i$ to~$P_j$ such that $f \big( P(u) \big) \neq f \big( P(v) \big)$ and $d(u) + d(v) \leq \Delta$.
\end{proof}

From the claim above, it follows that, as long as $\calV$ contains multiple sets, there is an edge~$uv \in E'$ such that $d(u) + d(v) \leq \Delta$ and $\mathrm{Find}(u) \neq \mathrm{Find}(v)$.
Therefore, $\max_{uv \in X} \big( d(u) + d(v) \big) \leq \Delta$ and, hence, $|S_\delta| \leq |T_\delta| + \Delta \cdot \Lambda(T_\delta)$.
\end{proof}

\begin{proof}
    [Complexity]
First, the algorithm computes~$\calP$ (line~\ref{line:createEmptyP} to line~\ref{line:flagPiClusters}).
If the parent of each vertex from the original BFS that was used to construct~$\calT$ is still known, $\calP$ can be constructed in $\calO(n)$ total time.
After picking a vertex~$v$ in $C_i$, simply follow the parent pointers until a vertex in~$C_i'$ is reached.
Computing $\calV$ as well as $P(v)$ and $d(v)$ for each vertex~$v$ of~$G$ (line~\ref{line:performBFS}) can be done with single BFS and, thus, requires at most $\calO(n + m)$ time.

Recall that, for a Union-Find data structure storing $n$~elements, each operation requires at most $\calO \big( \alpha(n) \big)$ amortised time.
Therefore, initialising such a data structure to store all vertices (line~\ref{line:initUnionFind}) and computing $E'$ (line~\ref{line:determineEprime}) requires at most $\calO \big( m \, \alpha(n) \big)$ time.
Note that, for each vertex~$v$, $d(v) \leq |V|$.
Thus, sorting~$E'$ (line~\ref{line:sortEprime}) can be done in linear time using counting sort.
When iterating over $E'$ (line~\ref{line:EprimeLoop} to line~\ref{line:unionSets}), for each edge~$uv \in E'$, the $\mathrm{Find}$-operation is called twice and the $\mathrm{Union}$-operation is called at most once.
Thus, the total runtime for all these operations is at most $\calO \big( m \, \alpha(n) \big)$.

Let $P_u = \{ u, \ldots, x, y, \ldots, p \}$ be the shortest path in~$G$ from a vertex~$u$ to~$P(u)$.
Assume that $y$ has been added to~$S_\delta$ in a previous iteration.
Thus, $\{ y, \ldots, p \} \subseteq S_\delta$ and, when adding $P_u$ to~$S_\delta$, the algorithm only needs to add $\{ u, \ldots, x \}$.
Therefore, by using a simple binary flag to determine if a vertex is contained in~$S_\delta$, constructing~$S_\delta$ (line~\ref{line:addPtoS}, line~\ref{line:addPuToS}, and line~\ref{line:addPvToS}) requires at most $\calO(n)$ time.

In total, Algorithm~\ref{algo:Tdelta} runs in $\calO \big( m \, \alpha(n) \big)$ time.
\end{proof}


Because, for each integer~$\delta \geq 0$, $|S_\delta| \leq |T_\delta| + \Delta \cdot \Lambda(T_\delta)$ (Lemma~\ref{lem:SdeltaCardinality}) and $|T_{\delta}| \leq |T_r| - \delta \cdot \Lambda(T_\delta)$ (Lemma~\ref{lem:TdeltaTrCardinality}), we have the following.

\begin{corollary}
    \label{cor:SdeltaCardinality}
For each \( \delta \geq \Delta \), \( |S_\delta| \leq |T_r| \) and, thus, \( |S_\delta| \leq |D_r| \).
\end{corollary}

To the best of our knowledge, there is no algorithm known that computes $\Delta$ in less than $\calO(nm)$ time.
Additionally, under reasonable assumptions, computing the diameter or radius of a general graph requires~$\Omega \big( n^2 \big)$ time~\cite{AbboWillWang2016}.
We conjecture that the runtime for computing $\Delta$ for a given graph has a similar lower bound.

To avoid the runtime required for computing~$\Delta$, we use the following approach shown in Algorithm~\ref{algo:con2DeltaDom} below.
First, compute a layering partition~$\calT$ and the subtree~$T_r$.
Second, for a certain value of~$\delta$, compute~$T_\delta$ and perform Algorithm~\ref{algo:Tdelta} on it.
If the resulting set~$S_\delta$ is larger than~$T_r$ (\ie, $|S_\delta| > |T_r|$), increase~$\delta$; otherwise, if $|S_\delta| \leq |T_r|$, decrease~$\delta$.
Repeat the second step with the new value of~$\delta$.

One strategy to select values for~$\delta$ is a classical binary search over the number of vertices of~$G$.
In this case, Algorithm~\ref{algo:Tdelta} is called up-to $\calO(\log n)$ times.
Empirical analysis~\cite{AbuAtaDragan2016}, however, have shown that $\Delta$ is usually very small.
Therefore, we use a one-sided binary search instead.

Because of Corollary~\ref{cor:SdeltaCardinality}, using a one-sided binary search allows us to find a value~$\delta \leq \Delta$ for which $|S_\delta| \leq |T_r|$ by calling Algorithm~\ref{algo:Tdelta} at most $\calO(\log \Delta)$ times.
Algorithm~\ref{algo:con2DeltaDom} below implements this approach.

\SetKwFor{OSBS}{One-Sided Binary Search}{}{}

\begin{algorithm}
    [htb]
    \caption
    {%
        Computes a connected $(r + 2 \Delta)$-dominating set for a given graph~$G$.
    }
    \label{algo:con2DeltaDom}

\KwIn
{%
    A graph~$G = (V, E)$ and a function~$r \colon V \rightarrow \bbN$.
}

\KwOut
{%
    A connected $(r + 2 \Delta)$-dominating set~$D$ for~$G$ with $|D| \leq |D_r|$.
}

Create a layering partition~$\calT$ of~$G$.

For each cluster~$C$ of~$\calT$, set $r(C) := \min_{v \in C} r(v)$.

Compute a minimum $r$-dominating subtree~$T_r$ for~$\calT$ (see~\cite{Dragan1993}).

\OSBS
{%
    over~\( \delta \), starting with \( \delta = 0 \)
}
{%
    Create a minimum $\delta$-dominating subtree~$T_\delta$ of~$T_r$ (\ie, $T_\delta$ is a minimum $(r + \delta)$-dominating subtree for~$\calT$).

    Run Algorithm~\ref{algo:Tdelta} on~$T_\delta$ and let the set~$S_\delta$ be the corresponding output.

    \If
    {%
        \( |S_\delta| \leq |T_r| \)
    }
    {%
        Decrease~$\delta$.
    }
    \Else
    {%
        Increase~$\delta$.
    }
}

Output $S_\delta$ with the smallest~$\delta$ for which $|S_\delta| \leq |T_r|$.
\end{algorithm}

\begin{theorem}
    \label{theo:rDeltaConDom}
For a given graph~\( G \), Algorithm~\ref{algo:con2DeltaDom} computes a connected \( (r + 2 \Delta) \)-dominating set~\( D \) with \( |D| \leq |D_r| \) in \( \calO \big( m \, \alpha(n) \log \Delta \big) \) time.
\end{theorem}

\begin{proof}
Clearly, the set~$D$ is connected because $D = S_\delta$ for some~$\delta$ and, by Lemma~\ref{lem:SdeltaCardinality}, the set~$S_\delta$ is connected.
By Corollary~\ref{cor:SdeltaCardinality}, for each $\delta \geq \Delta$, $|S_\delta| \leq |T_r|$.
Thus, for each $\delta \geq \Delta$, the binary search decreases~$\delta$ and, eventually, finds some~$\delta$ such that $\delta \leq \Delta$ and $|S_\delta| \leq |T_r|$.
Therefore, the algorithm finds a set~$D$ with $|D| \leq |D_r|$.
Note that, because $D = S_\delta$ for some $\delta \leq \Delta$ and because $S_\delta$ intersects each cluster of~$T_\delta$ (Lemma~\ref{lem:SdeltaCardinality}), it follows from Lemma~\ref{lem:TdeltaDist} that, for each vertex~$v$ of~$G$, $d_\calT(v, D) \leq r(v) + \Delta$ and, by Lemma~\ref{lem:LayPartVertDist}, $d_G(v, D) \leq r(v) + 2 \Delta$.
Thus, $D$ is an $(r + 2 \Delta)$-dominating set for~$G$.

Creating a layering partition for a given graph and computing a minimum connected $r$-dominating set of a tree can be done in linear time~\cite{Dragan1993}.
The one-sided binary search over~$\delta$ has at most $\calO(\log \Delta)$ iterations.
Each iteration of the binary search requires at most linear time to compute~$T_\delta$, $\calO \big( m \, \alpha(n) \big)$ time to compute~$S_\delta$ (Lemma~\ref{lem:SdeltaCardinality}), and constant time to decide whether to increase or decrease~$\delta$.
Therefore, Algorithm~\ref{algo:con2DeltaDom} runs in $\calO \big( m \, \alpha(n) \log \Delta \big)$ total time.
\end{proof}


\section{Using a Tree-Decomposition}

Theorem~\ref{theo:rDeltaDom} and Theorem~\ref{theo:rDeltaConDom} respectively show how to compute an $(r + \Delta)$-dominating set in linear time and a connected $(r + 2 \Delta)$-dominating set in $\calO \big( m \, \alpha(n) \log \Delta \big)$ time.
It is known that the maximum diameter~$\Delta$ of clusters of any layering partition of a graph approximates the tree-breadth and tree-length of this graph.
Indeed, as shown in Lemma~\ref{lem:LayPartDist} (page~\pageref{lem:LayPartDist}), for a graph~$G$ with $\tl(G) = \lambda$, $\Delta \leq 3 \lambda$.

\begin{corollary}
Let \( D \) be a minimum \( r \)-dominating set for a given graph~\( G \) with \( \tl(G) = \lambda \).
An \( (r + 3 \lambda) \)-dominating set~\( D' \) for~\( G \) with \( |D'| \leq |D| \) can be computed in linear time.
\end{corollary}

\begin{corollary}
Let \( D \) be a minimum connected \( r \)-dominating set for a given graph~\( G \) with \( \tl(G) = \lambda \).
A connected \( (r + 6 \lambda) \)-dominating set~\( D' \) for~\( G \) with \( |D'| \leq |D| \) can be computed in \( \calO \big( m \, \alpha(n) \log \lambda \big) \) time.
\end{corollary}

In this section, we consider the case when we are given a tree-decomposition with breadth~$\rho$ and length~$\lambda$.
We present algorithms to compute an $(r + \rho)$-dominating set as well as a connected $\big(r + \min (3 \lambda, 5 \rho) \big)$-dominating set in $\calO(nm)$~time.

For the remainder of this section, assume that we are given a graph~$G = (V, E)$ and a tree-decomposition~$\calT$ of~$G$ with breadth~$\rho$ and length~$\lambda$.
We assume that $\rho$ and~$\lambda$ are known and that, for each bag~$B$ of~$\calT$, we know a vertex~$c(B)$ with $B \subseteq N_G^\rho[c(B)]$.
Let $\calT$ be minimal, \ie, $B \nsubseteq B'$ for any two bags $B$ and~$B'$.
Thus, the number of bags is not exceeding the number vertices of~$G$.
Additionally, let each vertex of~$G$ store a list of bags containing it and let each bag of~$T$ store a list of vertices it contains.
One can see this as a bipartite graph where one subset of vertices are the vertices of~$G$ and the other subset are the bags of~$\calT$.
Therefore, the total input size is in $\calO(n + m + M)$ where $M \leq n^2$ is the sum of the cardinality of all bags of~$\calT$.


\subsection{Preprocessing}

Before approaching the (Connected) $r$-Domination problem, we compute a subtree~$T_r$ of~$\calT$ such that, for each vertex~$v$ of~$G$, $T_r$ contains a bag~$B$ with $d_G(v, B) \leq r(v)$.
We call such a (not necessarily minimal) subtree an \emph{\( r \)-covering subtree of~\( \calT \)}.

We do not know how to compute a minimum $r$-covering subtree~$T_r$ directly.
However, if we are given a bag~$B$ of~$\calT$, we can compute the smallest $r$-covering subtree~$T_B$ which contains~$B$.
Then, we can identify a bag~$B'$ in~$T_B$ for which we know it is a bag of~$T_r$.
Thus, we can compute $T_r$ by computing the smallest $r$-covering subtree which contains~$B'$.

The idea for computing~$T_B$ is to determine, for each vertex~$v$ of~$G$, the bag~$B_v$ of~$\calT$ for which $d_G(v, B_v) \leq r(v)$ and which is closet to~$B$.
Then, let $T_B$ be the smallest tree that contains all these bags~$B_v$.
Algorithm~\ref{algo:T_rWithBag} below implements this approach.

Additionally to computing the tree~$T_B$, we make it a rooted tree with $B$ as the root, give each vertex~$v$ a pointer~$\beta(v)$ to a bag of~$T_B$, and give each bag~$B'$ a counter~$\sigma(B')$.
The pointer~$\beta(v)$ identifies the bag~$B_v$ which is closest to~$B$ in~$T_B$ and intersects the $r$-neighbourhood of~$v$.
The counter~$\sigma(B')$ states the number of vertices~$v$ with $\beta(v) = B'$.
Even though setting $\beta$ and~$\sigma$ as well as rooting the tree are not necessary for computing~$T_B$, we use it when computing an $(r + \rho)$-dominating set later.

\begin{algorithm}
    [htb]
    \caption
    {%
        Computes the smallest $r$-covering subtree~$T_B$ of a given tree-decomposition~$\calT$ that contains a given bag~$B$ of~$\calT$.
    }
    \label{algo:T_rWithBag}

Make $\calT$ a rooted tree with the bag~$B$ as the root.

Create a set $\calB$ of bags and initialise it with $\calB := \{ B \}$.

For each bag~$B'$ of~$\calT$, set $\sigma(B') := 0$ and determine $d_\calT(B', B)$.
\label{line:bagDist}

For each vertex~$u$, determine the bag~$B(u)$ which contains~$u$ and has minimal distance to~$B$.
\label{line:determineBWithU}

\ForEach
{%
    \( u \in V \)%
    \label{line:loopBv}
}
{%
    Determine a vertex~$v$ such that $d_G(u, v) \leq r(u)$ and $d_\calT \big( B(v), B \big)$ is minimal and let $B_u := B(v)$.
    \label{line:determineBu}

    Add $B_u$ to $\calB$, set $\beta(u) := B_u$, and increase $\sigma(B_u)$ by~$1$.
    \label{line:addBvToB}
}

Output the smallest subtree~$T_B$ of~$\calT$ that contains all bags in~$\calB$.
\label{line:ConstructTB}
\end{algorithm}

\begin{lemma}
    \label{lem:TBAlgo}
For a given tree-decomposition~\( \calT \) and a given bag~\( B \) of~\( \calT \), Algorithm~\ref{algo:T_rWithBag} computes an \( r \)-covering subtree~\( T_B \) in \( \calO(nm) \) time such that \( T_B \) contains~\( B \) and has a minimal number of bags.
\end{lemma}

\begin{proof}
    [Correctness]
Note that, by construction of the set~$\calB$ (line~\ref{line:loopBv} to line~\ref{line:addBvToB}), $\calB$ contains a bag~$B_u$ for each vertex~$u$ of~$G$ such that $d_G(u, B_u) \leq r(u)$.
Thus, each subtree of~$\calT$ which contains all bags of~$\calB$ is an $r$-covering subtree.
To show the correctness of the algorithm, it remains to show that the smallest $r$-covering subtree of~$\calT$ which contains~$B$ has to contain each bag from the set~$\calB$.
Then, the subtree~$T_B$ constructed in line~\ref{line:ConstructTB} is the desired subtree.

Clearly, by properties of tree-decompositions, the set of bags which intersect the $r$-neighbourhood of some vertex~$u$ induces a subtree~$T_u$ of~$\calT$.
That is, $T_u$ contains exactly the bags~$B'$ with $d_G(u, B') \leq r(u)$.
Note that $\calT$ is a rooted tree with $B$ as the root.
Clearly, the bag~$B_u \in \calB$ (determined in line~\ref{line:determineBu}) is the root of~$T_u$ since it is the bag closest to~$B$.
Hence, each bag~$B'$ with $d_G(u, B') \leq r(u)$ is a descendant of~$B_u$.
Therefore, if a subtree of~$\calT$ contains~$B$ and does not contain~$B_u$, then it also cannot contain any descendant of~$B_u$ and, thus, contains no bag intersecting the $r$-neighbourhood of~$u$.
\end{proof}

\begin{proof}
    [Complexity]
Recall that $\calT$ has at most $n$~bags and that the sum of the cardinality of all bags of~$\calT$ is~$M \leq n^2$.
Thus, line~\ref{line:bagDist} and line~\ref{line:determineBWithU} require at most $\calO(M)$~time.
Using a BFS, it takes at most $\calO(m)$~time, for a given vertex~$u$, to determine a vertex~$v$ such that $d_G(u, v) \leq r(u)$ and $d_\calT \big( B(v), B \big)$ is minimal (line~\ref{line:determineBu}).
Therefore, the loop starting in line~\ref{line:loopBv} and, thus, Algorithm~\ref{algo:T_rWithBag} run in at most $\calO(nm)$ total time.
\end{proof}

Lemma~\ref{lem:rDiskLeaf} and Lemma~\ref{lem:TBleavesInTr} below show that each leaf~$B' \neq B$ of~$T_B$ is a bag of a minimum $r$-covering subtree~$T_r$ of~$\calT$.
Note that both lemmas only apply if $T_B$ has at least two bags.
If $T_B$ contains only one bag, it is clearly a minimum $r$-covering subtree.

\begin{lemma}
    \label{lem:rDiskLeaf}
For each leaf~\( B' \neq B \) of~\( T_B \), there is a vertex~\( v \) in~\( G \) such that \( B' \) is the only bag of~\( T_B \) with \( d_G(v, B') \leq r(v) \).
\end{lemma}

\begin{proof}
Assume that Lemma~\ref{lem:rDiskLeaf} is false.
Then, there is a leaf~$B'$ such that, for each vertex~$v$ with $d_G(v, B') \leq r(v)$, $T_B$ contains a bag~$B'' \neq B'$ with $d_G(v, B'') \leq r(v)$.
Thus, for each vertex~$v$, the $r$-neighbourhood of~$v$ is intersected by a bag of the tree-decomposition~$T_B - B'$.
This contradicts with the minimality of~$T_B$.
\end{proof}

\begin{lemma}
    \label{lem:TBleavesInTr}
For each leaf~\( B' \neq B \) of~\( T_B \), there is a minimum $r$-covering subtree~\( T_r \) of~\( \calT \) which contains~\( B'\).
\end{lemma}

\begin{proof}
Assume that $T_r$ is a minimum $r$-covering subtree which does not contain~$B'$.
Because of Lemma~\ref{lem:rDiskLeaf}, there is a vertex~$v$ of~$G$ such that $B'$ is the only bag of~$T_B$ which intersects the $r$-neighbourhood of~$v$.
Therefore, $T_r$ contains only bags which are descendants of~$B'$.
Partition the vertices of~$G$ into the sets $\Vup$ and~$\Vdown$ such that $\Vdown$ contains the vertices of~$G$ which are contained in~$B'$ or in a descendant of~$B'$.
Because $T_r$ is an $r$-covering subtree and because $T_r$ only contains descendants of~$B'$, it follows from properties of tree-decompositions that, for each vertex~$v \in \Vup$, there is a path of length at most~$r(v)$ from~$v$ to a bag of~$T_r$ passing through~$B'$ and, thus, $d_G(v, B') \leq r(v)$.
Similarly, since $T_B$ is an $r$-covering subtree, it follows that, for each vertex~$v \in \Vdown$, $d_G(v, B') \leq r(v)$.
Therefore, for each vertex~$v$ of~$G$, $d_G(v, B') \leq r(v)$ and, thus, $B'$ induces an $r$-covering subtree~$T_r$ of~$\calT$ with $|T_r| = 1$.
\end{proof}

Algorithm~\ref{algo:T_rDeco} below uses Lemma~\ref{lem:TBleavesInTr} to compute a minimum $r$-covering subtree~$T_r$ of~$\calT$.

\begin{algorithm}
    [htb]
    \caption
    {%
        Computes a minimum $r$-covering subtree~$T_r$ of a given tree-decomposition~$\calT$.
    }
    \label{algo:T_rDeco}

Pick an arbitrary bag~$B$ of~$\calT$.

Determine the subtree~$T_B$ of~$\calT$ using Algorithm~\ref{algo:T_rWithBag}.

\If
{%
    \( |T_B| = 1 \)
}
{%
    Output~$T_r := T_B$.
}
\Else
{%
    Select an arbitrary leaf~$B' \neq B$ of~$T_B$.

    Determine the subtree~$T_{B'}$ of~$\calT$ using Algorithm~\ref{algo:T_rWithBag}.

    Output~$T_r := T_{B'}$.
}
\end{algorithm}

\begin{lemma}
    \label{lem:computeTr}
Algorithm~\ref{algo:T_rDeco} computes a minimum \( r \)-covering subtree~\( T_r \) of~\( \calT \) in \( \calO(nm) \) time.
\end{lemma}

\begin{proof}
Algorithm~\ref{algo:T_rDeco} first picks an arbitrary bag~$B$ and then uses Algorithm~\ref{algo:T_rWithBag} to compute the smallest $r$-covering subtree~$T_B$ of~$\calT$ which contains~$B$.
By Lemma~\ref{lem:TBleavesInTr}, for each leaf~$B'$ of~$T_B$, there is a minimum $r$-covering subtree~$T_r$ which contains~$B'$.
Thus, performing Algorithm~\ref{algo:T_rWithBag} again with $B'$ as input creates such a subtree~$T_r$.

Clearly, with exception of calling Algorithm~\ref{algo:T_rWithBag}, all steps of Algorithm~\ref{algo:T_rDeco} require only constant time.
Because Algorithm~\ref{algo:T_rWithBag} requires at most $\calO(nm)$ time (see Lemma~\ref{lem:TBAlgo}) and is called at most two times, Algorithm~\ref{algo:T_rDeco} runs in at most $\calO(nm)$ total time.
\end{proof}

Algorithm~\ref{algo:T_rDeco} computes~$T_r$ by, first, computing~$T_B$ for some bag~$B$ and, second, computing~$T_{B'} = T_r$ for some leaf~$B'$ of~$T_B$.
Note that, because both trees are computed using Algorithm~\ref{algo:T_rWithBag}, Lemma~\ref{lem:rDiskLeaf} applies to $T_B$ and~$T_{B'}$.
Therefore, we can slightly generalise Lemma~\ref{lem:rDiskLeaf} as follows.

\begin{corollary}
    \label{cor:rDiskLeaf}
For each leaf~\( B \) of~\( T_r \), there is a vertex~\( v \) in~\( G \) such that \( B \) is the only bag of~\( T_r \) with \( d_G(v, B) \leq r(v) \).
\end{corollary}


\subsection{$r$-Domination}

In this subsection, we use the minimum $r$-covering subtree~$T_r$ to determine an $(r + \rho)$-dominating set~$S$ in $\calO(nm)$ time using the following approach.
First, compute~$T_r$.
Second, pick a leaf~$B$ of~$T_r$.
If there is a vertex~$v$ such that $v$ is not dominated and $B$ is the only bag intersecting the $r$-neighbourhood of~$v$, then add the center of~$B$ into~$S$, flag all vertices~$u$ with $d_G(u, B) \leq r(u)$ as dominated, and remove $B$ from~$T_r$.
Repeat the second step until $T_r$ contains no more bags and each vertex is flagged as dominated.
Algorithm~\ref{algo:rhoDomTreeDeco} below implements this approach.
Note that, instead of removing bags from~$T_r$, we use a reversed BFS-order of the bags to ensure the algorithm processes bags in the correct order.

\begin{algorithm}
    [htb]
    \caption
    {%
        Computes an $(r + \rho)$-dominating set~$S$ for a given graph~$G$ with a given tree-decomposition~$\calT$ with breadth~$\rho$.
    }
    \label{algo:rhoDomTreeDeco}

Compute a minimum $r$-covering subtree~$T_r$ of~$\calT$ using Algorithm~\ref{algo:T_rDeco}.
\label{line:computeTr}

Give each vertex~$v$ a binary flag indicating if $v$ is \emph{dominated}.
Initially, no vertex is dominated.

Create an empty vertex set~$S_0$.

Let $\langle B_1, B_2, \ldots, B_k \rangle$ be the reverse of a BFS-order of~$T_r$ starting at its root.
\label{line:computeBFSofTr}

\For
{%
    \( i = 1 \) \KwTo \( k \)
}
{%
    \If
    {%
        \( \sigma(B_i) > 0 \)
    }
    {%
        Determine all vertices~$u$ such that $u$ has not been flagged as dominated and that $d_G(u, B_i) \leq r(u)$.
        Add all these vertices into a new set~$X_i$.
        \label{line:determineXi}

        Let $S_i = S_{i - 1} \cup \big \{ c(B_i) \big \}$.

        For each vertex~$u \in X_i$, flag $u$ as dominated, and decrease $\sigma \big( \beta(u) \big)$ by~$1$.
        \label{line:flagVertex}
    }
    \Else
    {%
        Let $S_i = S_{i - 1}$.
        \label{line:setSi}
    }
}
Output $S := S_k$.
\end{algorithm}

\begin{theorem}
Let \( D \) be a minimum \( r \)-dominating set for a given graph~\( G \).
Given a tree-decomposition with breadth~\( \rho \) for~\( G \), Algorithm~\ref{algo:rhoDomTreeDeco} computes an \( (r + \rho) \)-dominating set~\( S \) with \( |S| \leq |D| \) in \( \calO(nm) \) time.
\end{theorem}

\begin{proof}
    [Correctness]
First, we show that $S$ is an $(r + \rho)$-dominating set for~$G$.
Note that a vertex~$v$ is flagged as dominated only if $S_i$ contains a vertex~$c(B_j)$ with $d_G(v, B_j) \leq r(v)$ (see line~\ref{line:determineXi} to line~\ref{line:flagVertex}).
Thus, $v$ is flagged as dominated only if $d_G(v, S_i) \leq d_G \big(v, c(B_j) \big) \leq r(v) + \rho$.
Additionally, by construction of~$T_r$ (see Algorithm~\ref{algo:T_rWithBag}), for each vertex~$v$, $T_r$~contains a bag~$B$ with $\beta(v) = B$, $\sigma(B)$ states the number of vertices~$v$ with $\beta(v) = B$, and $\sigma(B)$ is decreased by~$1$ only if such a vertex~$v$ is flagged as dominated (see line~\ref{line:flagVertex}).
Therefore, if $G$ contains a vertex~$v$ with $d_G(v, S_i) > r(v) + \rho$, then $v$ is not flagged as dominated and $T_r$ contains a bag~$B_i$ with $\beta(v) = B_i$ and $\sigma(B_i) > 0$.
Thus, when $B_i$ is processed by the algorithm, $c(B_i)$ will be added to~$S_i$ and, hence, $d_G(v, S_i) \leq r(v) + \rho$.

Let $V^S_i = \{ \, u \mid d_G(u, B_j) \leq r(u), c(B_j) \in S_i \, \}$ be the set of vertices which are flagged as dominated after the algorithm processed~$B_i$, \ie, each vertex in $V^S_i$ is $(r + \rho)$-dominated by~$S_i$.
Similarly, for some set~$D_i \subseteq D$, let $V^D_i = \{ \, u \mid d_G(u, D_i) \leq r(u) \, \}$ be the set of vertices dominated by~$D_i$.
To show that $|S| \leq |D|$, we show by induction over~$i$ that, for each~$i$, (i)~there is a set~$D_i \subseteq D$ such that $V^D_i \subseteq V^S_i$, (ii)~$|S_i| = |D_i|$, and (iii)~if, for some vertex~$v$, $\beta(v) = B_j$ with $j \leq i$, then $v \in V^S_i$.

For the base case, let $S_0 = D_0 = \emptyset$.
Then, $V^S_0 = V^D_0 = \emptyset$ and all three statements are satisfied.
For the inductive step, first, consider the case when $\sigma(B_i) = 0$.
Because $\sigma(B_i) = 0$, each vertex~$v$ with $\beta(v) = B_i$ is flagged as dominated, \ie, $v \in V^S_{i-1}$.
Thus, by setting $S_i = S_{i - 1}$ (line~\ref{line:setSi}) and $D_i = D_{i - 1}$, all three statements are satisfied for~$i$.
Next, consider the case when $\sigma(B_i) > 0$.
Therefore, $G$ contains a vertex~$u$ with $\beta(u) = B_i$ and $u \notin V^S_{i-1}$.
Then, the algorithm sets $S_i = S_{i-1} \cup \big \{ c(B_i) \big \}$ and flags all such $u$ as dominated (see line~\ref{line:determineXi} to line~\ref{line:flagVertex}).
Thus, $u \in V^S_i$ and statement~(iii) is satisfied.
%
Let $d_u$ be a vertex in $D$ with minimal distance to~$u$.
Thus, $d_G(d_u, u) \leq r(u)$, \ie, $d_u$ is in the $r$-neighbourhood of~$u$.
Note that, because $u \notin V^S_{i-1}$ and $V^D_{i-1} \subseteq V^S_{i-1}$, $d_u \notin D_{i-1}$.
Therefore, by setting $D_i = D_{i-1} \cup \{ d_u \}$, $|S_i| = |S_{i-1}| + 1 = |D_{i-1}| + 1 = |D_i|$ and statement~(ii) is satisfied.
%
Recall that $\beta(u)$ points to the bag closest to the root of~$T_r$ which intersects the $r$-neighbourhood of~$u$.
Thus, because $\beta(u) = B_i$, each bag~$B \neq B_i$ with $d_G(u, B) \leq r(u)$ is a descendant of~$B_i$.
Therefore, $d_u$ is in~$B_i$ or in a descendant of~$B_i$.
Let $v$ be an arbitrary vertex of~$G$ such that $v \notin V^S_{i-1}$ and $d_G(v, d_u) \leq r(v)$, \ie, $v$ is dominated by~$d_u$ but not by~$S_{i-1}$.
Due to statement~(iii) of the induction hypothesis, $\beta(v) = B_j$ with $j \geq i$, \ie, $B_j$ cannot be a descendant of~$B_i$.
Partition the vertices of~$G$ into the sets $\Vup_i$ and~$\Vdown_i$ such that $\Vdown_i$ contains the vertices which are contained in~$B_i$ or in a descendant of~$B_i$.
If $v \in \Vdown_i$, then there is a path of length at most~$r(v)$ from $v$ to~$B_j$ passing through~$B_i$.
If $v \in \Vup_i$, then, because $d_u \in \Vdown_i$, there is a path of length at most~$r(v)$ from $v$ to~$d_u$ passing through~$B_i$.
Therefore, $d_G(v, B_i) \leq r(v)$.
That is, each vertex $r$-dominated by~$d_u$, is $(r + \rho)$-dominated by some~$c(B_j) \in S_i$.
Therefore, because $S_i = S_{i-1} \cup \big \{ c(B_i) \big \}$ and $D_i = D_{i-1} \cup \{ d_u \}$, $v \in V^S_i \cap V^D_i $ and, thus, statement~(i) is satisfied.
\end{proof}

\begin{proof}
    [Complexity]
Computing~$T_r$ (line~\ref{line:computeTr}) takes at most~$\calO(nm)$ time (see Lemma~\ref{lem:computeTr}).
Because $T_r$ has at most $n$ bags, computing a BFS-order of $T_r$ (line~\ref{line:computeBFSofTr}) takes at most $\calO(n)$ time.
For some bag~$B_i$, determining all vertices~$u$ with $d_G(u, B_i) \leq r(u)$, flagging $u$ as dominated, and decreasing $\sigma \big( \beta(u) \big)$ (line~\ref{line:determineXi} to line~\ref{line:flagVertex}) can be done in $\calO(m)$ time by performing a BFS starting at all vertices of~$B_i$ simultaneously.
Therefore, because $T_r$ has at most $n$ bags, Algorithm~\ref{algo:rhoDomTreeDeco} requires at most $\calO(nm)$ total time.
\end{proof}


\subsection{Connected $r$-Domination}

In this subsection, we show how to compute a connected $(r + 5 \rho)$-dominating set and a connected $(r + 3 \lambda)$-dominating set for~$G$.
For both results, we use almost the same algorithm.
To identify and emphasise the differences, we use the label~\rHrt for parts which are only relevant to determine a connected $(r + 5 \rho)$-dominating set and use the label~\rDmd for parts which are only relevant to determine a connected $(r + 3 \lambda)$-dominating set.

For the remainder of this subsection, let $D_r$ be a minimum connected $r$-dominating set of~$G$.
For \rHrt~$\phi = 3 \rho$ or \rDmd~$\phi = 2 \lambda$, let $T_\phi$ be a minimum $(r + \phi)$-covering subtree of~$\calT$ as computed by Algorithm~\ref{algo:T_rDeco}.

The idea of our algorithm is to, first, compute~$T_\phi$ and, second, compute a small enough connected set~$C_\phi$ such that $C_\phi$ intersects each bag of~$T_\phi$.
Lemma~\ref{lem:rPhiDomSet} below shows that such a set~$C_\phi$ is an $\big( r + (\phi + \lambda) \big)$-dominating set.

\begin{lemma}
    \label{lem:rPhiDomSet}
Let \( C_\phi \) be a connected set that contains at least one vertex of each leaf of~\( T_\phi \).
Then, \( C_\phi \) is an \( \big( r + (\phi + \lambda) \big) \)-dominating set.
\end{lemma}

\begin{proof}
Clearly, since $C_\phi$ is connected and contains a vertex of each leaf of~$T_\phi$, $C_\phi$ contains a vertex of every bag of~$T_\phi$.
By construction of~$T_\phi$, for each vertex~$v$ of~$G$, $T_\phi$ contains a bag~$B$ such that $d_G(v, B) \leq r(v) + \phi$.
Therefore, $d_G(v, C_\phi) \leq r(v) + \phi + \lambda$, \ie, $C_\phi$ is an $\big( r + (\phi + \lambda) \big)$-dominating set.
\end{proof}

To compute a connected set~$C_\phi$ which intersects all leaves of~$T_\phi$, we first consider the case when $T_\rho$ contains only one bag~$B$.
In this case, we can construct~$C_\phi$ by simply picking an arbitrary vertex~$v \in B$ and setting $C_\phi = \{ v \}$.
Similarly, if $T_\rho$ contains exactly two bags $B$ and~$B'$, pick a vertex~$v \in B \cap B'$ and set $C_\phi = \{ v \}$.
In both cases, due to Lemma~\ref{lem:rPhiDomSet}, $C_\phi$ is clearly an $\big( r + (\phi + \lambda) \big)$-dominating set with $|C_\phi| \leq |D_r|$.

Now, consider the case when $T_\phi$ contains at least three bags.
Additionally, assume that $T_\phi$ is a rooted tree such that its root~$R$ is a leaf.

\subsubsection{Notation.}
Based on its degree in~$T_\phi$, we refer to each bag~$B$ of~$T_\phi$ either as leaf, as \emph{path bag} if $B$ has degree~$2$, or as \emph{branching bag} if $B$ has a degree larger than~$2$.
Additionally, we call a maximal connected set of path bags a \emph{path segment} of~$T_\phi$.
Let $\bbL$ denote the set of leaves, $\bbP$ denote the set of path segments, and $\bbB$ denote the set of branching bags of~$T_\phi$.
Clearly, for any given tree~$T$, the sets $\bbL$, $\bbP$, and~$\bbB$ are pairwise disjoint and can be computed in linear time.

Let $B$ and~$B'$ be two adjacent bags of~$T_\phi$ such that $B$ is the parent of~$B'$.
We call $S = B \cap B'$ the \emph{up-separator} of~$B'$, denoted as~$\Sup(B')$, and a \emph{down-separator} of~$B$, denoted as $\Sdown(B)$, \ie, $S = \Sup(B') = \Sdown(B)$.
Note that a branching bag has multiple down-separators and that (with exception of~$R$) each bag has exactly one up-separator.
For each branching bag~$B$, let~$\calSdown(B)$ be the set of down-separators of~$B$.
Accordingly, for a path segment~$P \in \bbP$, $\Sup(P)$ is the up-separator of the bag in~$P$ closest to the root and $\Sdown(P)$ is the down separator of the bag in~$P$ furthest from the root.
Let $\nu$ be a function that assigns a vertex of~$G$ to a given separator.
Initially, $\nu(S)$ is undefined for each separator~$S$.


\subsubsection{Algorithm.}
Now, we show how to compute~$C_\phi$.
We, first, split $T_\phi$ into the sets $\bbL$, $\bbP$, and~$\bbB$.
Second, for each~$P \in \bbP$, we create a small connected set~$C_P$, and, third, for each~$B \in \bbB$, we create a small connected set~$C_B$.
If this is done properly, the union~$C_\phi$ of all these sets forms a connect set which intersects each bag of~$T_\phi$.

Note that, due to properties of tree-decompositions, it can be the case that there are two bags $B$ and~$B'$ which have a common vertex~$v$, even if $B$ and~$B'$ are non-adjacent in~$T_\phi$.
In such a case, either $v \in \Sdown(B) \cap \Sup(B')$ if $B$ is an ancestor of~$B'$, or $v \in \Sup(B) \cap \Sup(B')$ if neither is ancestor of the other.
To avoid problems caused by this phenomena and to avoid counting vertices multiple times, we consider any vertex in an up-separator as part of the bag above.
That is, whenever we process some segment or bag~$X \in \bbL \cup \bbP \cup \bbB$, even though we add a vertex~$v \in \Sup(X)$ to~$C_\phi$, $v$ is not contained in~$C_X$.


\paragraph{Processing Path Segments.}
First, after splitting~$T_\phi$, we create a set~$C_{P}$ for each path segment~$P \in \bbP$ as follows.
We determine $\Sup(P)$ and~$\Sdown(P)$ and then find a shortest path~$Q_P$ from $\Sup(P)$ to~$\Sdown(P)$.
Note that $Q_P$ contains exactly one vertex from each separator.
Let $x \in \Sup(P)$ and $y \in \Sdown(P)$ be these vertices.
Then, we set $\nu \big( \Sup(P) \big) = x$ and $\nu \big( \Sdown(P) \big) = y$.
Last, we add the vertices of~$Q_P$ into~$C_\phi$ and define $C_P$ as $Q_P \setminus \Sup(P)$.
Let $C_\bbP$ be the union of all sets~$C_P$, \ie, $C_\bbP = \bigcup_{P \in \bbP} C_P$.

\begin{lemma}
    \label{lem:CPcardinality}
\( |C_\bbP| \leq |D_r| - \phi \cdot \Lambda \big( T_\phi \big) \).
\end{lemma}

\begin{proof}
Recall that $T_\phi$ is a minimum $(r + \phi)$-covering subtree of~$\calT$.
Thus, by Corollary~\ref{cor:rDiskLeaf}, for each leaf~$B \in \bbL$ of~$T_\phi$, there is a vertex~$v$ in~$G$ such that $B$ is the only bag of~$T_\phi$ with $d_G(v, B) \leq r(v) + \phi$.
Because $D_r$ is a connected $r$-dominating set, $D_r$ intersects the $r$-neighbourhood of each of these vertices~$v$.
Thus, by properties of tree-decompositions, $D_r$ intersects each bag of~$T_\phi$.
Additionally, for each such~$v$, $D_r$ contains a path~$D_v$ with $|D_v| \geq \phi$ such that $D_v$ intersects the $r$-neighbourhood of~$v$, intersects the corresponding leaf~$B$ of~$T_\phi$, and does not intersect $\Sup(B)$ ($\Sdown(B)$ if $B = R$).
Let $D_\bbL$ be the union of all such sets~$D_v$.
Therefore, $|D_\bbL| \geq \phi \cdot \Lambda \big( T_\phi \big)$.

Because $D_r$ intersects each bag of~$T_\phi$, $D_r$ also intersects the up- and down-separators of each path segment.
For a path segment~$P \in \bbP$, let $x$ and~$y$ be two vertices of~$D_r$ such that $x \in \Sup(P)$, $y \in \Sdown(P)$, and for which the distance in~$G[D_r]$ is minimal.
Let $D_P$ be the set of vertices on the shortest path in~$G[D_r]$ from $x$ to~$y$ without~$x$, \ie, $x \notin D_P$.
Note that, by construction, for each~$P \in \bbP$, $D_P$ contains exactly one vertex in $\Sdown(P)$ and no vertex in~$\Sup(P)$.
Thus, for all~$P, P' \in \bbP$, $D_P \cap D_{P'} = \emptyset$.
Let $D_\bbP$ be the union of all such sets~$D_P$, \ie, $D_\bbP = \bigcup_{P \in \bbP} D_P$.
By construction, $|D_\bbP| = \sum_{P \in \bbP} |D_P|$ and $D_\bbL \cap D_\bbP = \emptyset$.
Therefore, $|D_r| \geq |D_\bbP| + |D_\bbL|$ and, hence,
\[
    \sum_{P \in \bbP} |D_P| \leq |D_r| - |D_\bbL| \leq |D_r| - \phi \cdot \Lambda \big( T_\phi \big).
\]

Recall that, for each $P \in \bbP$, the sets $C_P$ and~$D_P$ are constructed based on a path from $\Sup(P)$ to~$\Sdown(P)$.
Since $C_P$ is based on a shortest path in~$G$, it follows that $|C_P| = d_G \big( \Sup(P), \Sdown(P) \big) \leq |D_P|$.
Therefore,
\[
    |C_\bbP| \leq \sum_{P \in \bbP} |C_P| \leq \sum_{P \in \bbP} |D_P| \leq |D_r| - \phi \cdot \Lambda \big( T_\phi \big).
    \qedhere
\]
\end{proof}


\paragraph{Processing Branching Bags.}
After processing path segments, we process the branching bags of~$T_\phi$.
Similar to path segments, we have to ensure that all separators are connected.
Branching bags, however, have multiple down-separators.
To connect all separators of some bag~$B$, we pick a vertex~$s$ in each separator~$S \in \calSdown(B) \cup \big \{ \Sup(B) \big \}$.
If $\nu(S)$ is defined, we set $s = \nu(S)$.
Otherwise, we pick an arbitrary~$s \in S$ and set $\nu(S) = s$.
Let $\calSdown(B) = \{ S_1, S_2, \ldots \}$, $s_i = \nu(S_i)$, and $t = \nu \big( \Sup(B) \big)$.
We then connect these vertices as follows.
(See Figure~\ref{fig:branchingBagConstr} for an illustration.)
\begin{enumerate}[\rHrt]
    \item[\rHrt]
        Connect each vertex~$s_i$ via a shortest path~$Q_i$ (of length at most~$\rho$) with the center~$c(B)$ of~$B$.
        Additionally, connect~$c(B)$ via a shortest path~$Q_t$ (of length at most~$\rho$) with~$t$.
        Add all vertices from the paths~$Q_i$ and from the path~$Q_t$ into $C_\phi$ and let $C_B$ be the union of these paths without~$t$.
    \item[\rDmd]
        Connect each vertex~$s_i$ via a shortest path~$Q_i$ (of length at most~$\lambda$) with~$t$.
        Add all vertices from the paths~$Q_i$ into $C_\phi$ and let $C_B$ be the union of these paths without~$t$.
\end{enumerate}
Let $C_\bbB$ be the union of all created sets~$C_B$, \ie, $C_\bbB = \bigcup_{B \in \bbB} C_B$.

\begin{figure}
    [htb]
    \centering
    \renewcommand\thesubfigure{}
    \captionsetup[subfigure]{labelformat=simple,labelsep=none}
    \begin{subfigure}[b]{0.45\textwidth}
        \centering
        \tikzsetnextfilename{fig_conD_breadthConstr}
\begin{tikzpicture}

\draw (0,0) circle (1cm);

\foreach \rot/\lbl in {90/Top,225/Left,-45/Right}
{
    \begin{scope}[rotate=\rot]
        \coordinate (c\lbl) at (1.4,0);
        \draw (1.4,0.9) -- ++(0,-1.8);
        \node[sN] (n\lbl) at (c\lbl) {};
    \end{scope}
}

\node [rlbl] at (0.9,1.4) {$\Sup$};
\node [lbl,anchor=45] at (nLeft.225) {$\Sdown_i$};
\node [lbl,anchor=135] at (nRight.-45) {$\Sdown_j$};

\node [nN,rectangle] (nCentre) at (0,0) {};

\begin{pgfonlayer}{background}

\draw
    [-latex,shorten >=1pt]
    (cLeft) --
    node [pos=0.6,lbl,inner sep=2pt,anchor=-45] {$\rho$}
    (nCentre);

\draw
    [
        -latex,
        shorten >=1pt,
    ]
    (cRight) --
    node
        [
            pos=0.6,
            lbl,
            inner sep=2pt,
            anchor=-135,
        ]
        {$\rho$}
    (nCentre);

\draw
    [
        -latex,
        shorten >=1pt,
    ]
    (0,0) --
    node
        [
            pos=0.4,
            inner sep=2pt,
            rlbl,
        ]
        {$\rho$}
    (nTop);

\end{pgfonlayer}

\end{tikzpicture}

        \caption{\rHrt}
    \end{subfigure}
    \hfil
    \begin{subfigure}[b]{0.45\textwidth}
        \centering
        \tikzsetnextfilename{fig_conD_lengthConstr}
\begin{tikzpicture}

\draw (0,0) circle (1cm);

\foreach \rot/\lbl in {90/Top,225/Left,-45/Right}
{
    \begin{scope}[rotate=\rot]
        \coordinate (c\lbl) at (1.4,0);
        \draw (1.4,0.9) -- ++(0,-1.8);
        \node[sN] (n\lbl) at (c\lbl) {};
    \end{scope}
}

\node [rlbl] at (0.9,1.4) {$\Sup$};
\node [lbl,anchor=45] at (nLeft.225) {$\Sdown_i$};
\node [lbl,anchor=135] at (nRight.-45) {$\Sdown_j$};

\begin{pgfonlayer}{background}

\draw
    [
        -latex,
        shorten >=1pt,
    ]
    (cLeft)
    .. controls (225:0.0) and (90:0.2) ..
    node [midway,llbl] {$\lambda$}
    (nTop.south);

\draw
    [
        -latex,
        shorten >=1pt,
    ]
    (cRight)
    .. controls (-45:0.0) and (90:0.2) ..
    node [midway,rlbl] {$\lambda$}
    (nTop.south);

\end{pgfonlayer}

\end{tikzpicture}

        \caption{\rDmd}
    \end{subfigure}%

    \caption
    {%
        Construction of the set~$C_B$ for a branching bag~$B$.
    }
    \label{fig:branchingBagConstr}
\end{figure}

Before analysing the cardinality of~$C_\bbB$ in Lemma~\ref{lem:CBcardinality} below, we need an axillary lemma.

\begin{lemma}
    \label{lem:treeLeafsBranch}
For a tree~\( T \) which is rooted in one of its leaves, let \( b \) denote the number of branching nodes, \( c \) denote the total number of children of branching nodes, and \( l \) denote the number of leaves.
Then, \( c + b \leq 3 l - 1 \) and \( c \leq 2 l - 1 \).
\end{lemma}

\begin{proof}
Assume that we construct $T$ by starting with only the root and then step by step adding leaves to it.
Let $T_i$ be the subtree of~$T$ with $i$~nodes during this construction.
We define $b_i$, $c_i$, and~$l_i$ accordingly.
Now, assume by induction over~$i$ that Lemma~\ref{lem:treeLeafsBranch} is true for~$T_i$.
Let $v$ be the leaf we add to construct~$T_{i+1}$ and let $u$ be its neighbour.

First, consider the case when $u$ is a leaf of~$T_i$.
Then, $u$ becomes a path node of~$T_{i+1}$.
Therefore, $b_{i+1} = b_i$, $c_{i+1} = c_i$, and $l_{i+1} = l_i$.
%
Next, assume that $u$ is path node of~$T_i$.
Then, $u$ is a branch node of~$T_{i+1}$.
Thus, $b_{i+1} = b_i + 1$, $c_{i+1} = c_i + 2$, and $l_{i+1} = l_i + 1$.
Therefore, $c_{i+1} + b_{i+1} = c_i + b_i + 3 \leq 3 (l_i + 1) - 1 = 3 l_{i+1} - 1$ and $c_{i+1} = c_i + 2 \leq 2 (l_i + 1) - 1 = 2 l_{i+1} - 1$.
%
It remains to check the case when $u$ is a branch node of~$T_i$.
Then, $b_{i+1} = b_i$, $c_{i+1} = c_i + 1$, and $l_{i+1} = l_i + 1$.
Thus, $c_{i+1} + b_{i+1} = c_i + b_i + 1 \leq 3 l_i - 1 + 1 \leq 3 l_{i+1} - 1$ and $c_{i+1} = c_i + 1 \leq 2 l_i - 1 + 1 \leq 2 l_{i+1} - 1$.
Therefore, in all three cases, Lemma~\ref{lem:treeLeafsBranch} is true for~$T_{i+1}$.
\end{proof}

\begin{lemma}
    \label{lem:CBcardinality}
\( |C_\bbB| \leq \phi \cdot \Lambda \big( T_\phi \big) \).
\end{lemma}

\begin{proof}
For some branching bag~$B \in \bbB$, the set~$C_B$ contains \rHrt~a path of length at most~$\rho$ for each~$S_i \in \calSdown(B)$ and a path of length at most~$\rho$ to~$\Sup(B)$, or \rDmd~a path of length at most~$\lambda$ for each~$S_i \in \calSdown(B)$.
Thus, \rHrt~$|C_B| \leq \rho \cdot \big| \calSdown(B) \big| + \rho$ or \rDmd~$|C_B| \leq \lambda \cdot \big| \calSdown(B) \big|$.
Recall that $\calSdown(B)$ contains exactly one down-separator for each child of~$B$ in~$T_\phi$ and that $C_\bbB$ is the union of all sets~$C_B$.
Therefore, Lemma~\ref{lem:treeLeafsBranch} implies the following.
\begin{alignat*}{2}
    |C_\bbB|
        & \leq \sum_{B \in \bbB} |C_B| \\
    \rHrt \enspace
        &  \leq \rho \cdot \! \sum_{B \in \bbB} \big| \calSdown(B) \big| + \rho \cdot |\bbB|
        && \leq 3 \rho \cdot \Lambda \big( T_\phi \big) - 1 \\
    \rDmd \enspace
        &  \leq \lambda \cdot \! \sum_{B \in \bbB} \big| \calSdown(B) \big|
        && \leq 2 \lambda \cdot \Lambda \big( T_\phi \big) - 1 \\
        &  \leq \phi \cdot \Lambda \big( T_\phi \big) - 1.
        \qedhere
\end{alignat*}
\end{proof}

\paragraph{Properties of \( C_\phi \).}
We now analyse the created set~$C_\phi$ with the result that $C_\phi$ is a connected $(r + \phi)$-dominating set for~$G$.

\begin{lemma}
    \label{lem:CphiBagInt}
\( C_\phi \) contains a vertex in each bag of~\( T_\phi \).
\end{lemma}

\begin{proof}
Clearly, by construction, $C_\phi$ contains a vertex in each path bag and in each branching bag.
Now, consider a leaf~$L$ of~$T_\phi$.
$L$ is adjacent to a path segment or branching bag~$X \in \bbP \cap \bbB$.
Whenever such an $X$ is processed, the algorithm ensures that all separators of~$X$ contain a vertex of~$C_\phi$.
Since one of these separators is also the separator of~$L$, it follows that each leaf~$L$ and, thus, each bag of~$T_\phi$ contains a vertex of~$C_\phi$.
\end{proof}

\begin{lemma}
    \label{lem:CphiCardinality}
\( |C_\phi| \leq |D_r| \).
\end{lemma}

\begin{proof}
Note that, for each vertex~$u$ we add to~$C_\phi$, we also add $u$ to a unique set~$C_X$ for some $X \in \bbP \cap \bbB$.
The exception is the vertex~$v$ in~$\Sdown(R)$ which is added to no such set~$C_X$.
It follows from our construction of the sets~$C_X$ that there is only one such vertex~$v$ and that $v = \nu \big( \Sdown(R) \big)$.
Thus, $|C_\phi| = |C_\bbP| + |C_\bbB| + 1$.
Now, it follows from Lemma~\ref{lem:CPcardinality} and Lemma~\ref{lem:CBcardinality} that
\[
    |C_\phi| \leq |D_r| - \phi \cdot \Lambda \big( T_\phi \big) + \phi \cdot \Lambda \big( T_\phi \big) - 1 + 1 \leq |D_r|.
    \qedhere
\]
\end{proof}

\begin{lemma}
    \label{lem:CphiConnected}
\( C_\phi \) is connected.
\end{lemma}

\begin{proof}
First, note that, by maximality, two path segments of~$T_\phi$ cannot share a common separator.
Also, note that, when processing a branching bag~$B$, the algorithm first checks if, for any separator~$S$ of~$B$, $\nu(S)$ is already defined; if this is the case, it will not be overwritten.
Therefore, for each separator~$S$ in~$T_\phi$, $\nu(S)$ is defined and never overwritten.

Next, consider a path segment or branching bag~$X \in \bbP \cup \bbB$ and let $S$ and~$S'$ be two separators of~$X$.
Whenever such an $X$ is processed, our approach ensures that $C_\phi$ connects $\nu(S)$ with~$\nu(S')$.
Additionally, observe that, when processing~$X$, each vertex added to~$C_\phi$ is connected via~$C_\phi$ with~$\nu(S)$ for some separator~$S$ of~$X$.

Thus, for any two separators $S$ and~$S'$ in~$T_\phi$, $C_\phi$ connects $\nu(S)$ with~$\nu(S')$ and, additionally, each vertex~$v \in C_\phi$ is connected via~$C_\phi$ with~$\nu(S)$ for some separator~$S$ in~$T_\phi$.
Therefore, $C_\phi$ is connected.
\end{proof}

From Lemma~\ref{lem:CphiBagInt}, Lemma~\ref{lem:CphiCardinality}, Lemma~\ref{lem:CphiConnected}, and from applying Lemma~\ref{lem:rPhiDomSet} it follows:

\begin{corollary}
    \label{cor:CphiConDomSet}
\( C_\phi \) is a connected \( \big( r + (\phi + \lambda) \big) \)-dominating set for~\( G \) with \( |C_\phi| \leq |D_r| \).
\end{corollary}


\paragraph{Implementation.}
Algorithm~\ref{algo:conPhiDomination} below implements our approach described above.
This also includes the case when~$T_\phi$ contains at most two bags.

\SetKw{KwStop}{stop}

\begin{algorithm}
    [!htb]
    \caption
    {%
        Computes \rHrt~a connected $(r + 5 \rho)$-dominating set or \rDmd~a connected $(r + 3 \lambda)$-dominating set for a given graph~$G$ with a given tree-decomposition~$\calT$ with breadth~$\rho$ and length~$\lambda$.
    }
    \label{algo:conPhiDomination}

\parbox[t]{\hsize}
{%
    \rHrt Set $\phi := 3 \rho$. \\
    \rDmd Set $\phi := 2 \lambda$.
}

Compute a minimum $(r + \phi)$-covering subtree~$T_\phi$ of~$\calT$ using Algorithm~\ref{algo:T_rDeco}.
\label{line:compTphi}

\If
{%
    \( T_\phi \) contains only one bag~\( B \)%
    \label{line:ifTphiOneBag}
}
{%
    Pick an arbitrary vertex~$u \in B$, output~$C_\phi := \{ u \}$, and \KwStop.
}

\If
{%
    \( T_\phi \) contains exactly two bags \( B \) and~\( B' \)
}
{%
    Pick an arbitrary vertex~$u \in B \cap B'$, output~$C_\phi := \{ u \}$, and \KwStop.
    \label{line:TphiTwoBags}
}

Pick a leaf of~$T_\phi$ and make it the root of~$T_\phi$.
\label{line:rootTphi}

Split $T_\phi$ into a set~$\bbL$ of leaves, a set~$\bbP$ of path segments, and a set~$\bbB$ of branching bags.
\label{line:splitTphi}

Create an empty set~$C_\phi$.

\ForEach
{%
    \( P \in \bbP \)
}
{%
    Find a shortest path~$Q_P$ from $\Sup(P)$ to~$\Sdown(P)$ and add its vertices into~$C_\phi$.
    \label{line:pathSegFindQ}

    Let $x \in \Sup(P)$ be the start vertex and $y \in \Sdown(P)$ be the end vertex of~$Q_P$.
    Set $\nu \big( \Sup(P) \big) := x$ and $\nu \big( \Sdown(P) \big) := y$.
    \label{line:pathSegSep}
}

\ForEach
{%
    \( B \in \bbB \)%
    \label{line:loopBranchBag}
}
{%
    If $\nu \big( \Sup(B) \big)$ is defined, let $u := \nu \big( \Sup(B) \big)$.
    Otherwise, let $u$ be an arbitrary vertex in~$\Sup(B)$ and set $\nu \big( \Sup(B) \big) := u$.

    \parbox[t]{\hsize}
    {%
        \rHrt Let $v := c(B)$ be the center of~$B$. \\
        \rDmd Let $v := u$.
    }
    \label{line:defineV}

    Find a shortest path from $u$ to~$v$ and add its vertices into~$C_\phi$.

    \ForEach
    {%
        \( S_i \in \calSdown(B) \)
    }
    {%
        If $\nu(S_i)$ is defined, let $w_i := \nu(S_i)$.
        Otherwise, let $w_i$ be an arbitrary vertex in~$S_i$ and set $\nu(S_i) := w_i$.

        Find a shortest path from $w_i$ to~$v$ and add the vertices of this path into~$C_\phi$.
        \label{line:conBagSepar}
    }
}

Output $C_\phi$.
\end{algorithm}

\begin{theorem}
Algorithm~\ref{algo:conPhiDomination} computes a connected \( \big( r + (\phi + \lambda) \big) \)-dominating set~\( C_\phi \) with \( |C_\phi| \leq |D_r| \) in \( \calO(nm) \) time.
\end{theorem}

\begin{proof}
Since Algorithm~\ref{algo:conPhiDomination} constructs a set~$C_\phi$ as described above, its correctness follows from Corollary~\ref{cor:CphiConDomSet}.
It remains to show that the algorithm runs in $\calO(nm)$ time.

Computing $T_\phi$ (line~\ref{line:compTphi}) can be done in $\calO(nm)$ time (see Lemma~\ref{lem:computeTr}).
Picking a vertex~$u$ in the case when $T_\phi$ contains at most two bags (line~\ref{line:ifTphiOneBag} to line~\ref{line:TphiTwoBags}) can be easily done in $\calO(n)$ time.
Recall that $T_\phi$ has at most $n$~bags.
Thus, splitting $T_\phi$ in the sets $\bbL$, $\bbP$, and~$\bbB$ can be done in $\calO(n)$ time.

Determining all up-separators in~$T_\phi$ can be done in $\calO(M)$ time as follows.
Process all bags of~$T_\phi$ in an order such that a bag is processed before its descendants, \eg, use a preorder or BFS-order.
Whenever a bag~$B$ is processed, determine a set~$S \subseteq B$ of flagged vertices, store $S$ as up-separator of~$B$, and, afterwards, flag all vertices in~$B$.
Clearly, $S$~is empty for the root.
Because a bag~$B$ is processed before its descendants, all flagged vertices in~$B$ also belong to its parent.
Thus, by properties of tree-decompositions, these vertices are exactly the vertices in~$\Sup(B)$.
Clearly, processing a single bag~$B$ takes at most $\calO(|B|)$ time.
Thus, processing all bags takes at most $\calO(M)$ time.
Note that it is not necessary to determine the down-separators of a (branching) bag.
They can easily be accessed via the children of a bag.

Processing a single path segment (line~\ref{line:pathSegFindQ} and line~\ref{line:pathSegSep}) can be easily done in $\calO(m)$ time.
Processing a branching bag~$B$ (line~\ref{line:loopBranchBag} to line~\ref{line:conBagSepar}) can be implemented to run in $\calO(m)$ time by, first, determining $\nu(S)$ for each separator~$S$ of~$B$ and, second, running a BFS starting at~$v$ (defined in line~\ref{line:defineV}) to connect $v$ with each vertex~$\nu(S)$.
Because $T_\phi$ has at most $n$~bags, it takes at most $\calO(nm)$ time to process all path segments and branching bags of~$T_\phi$.
Therefore, Algorithm~\ref{algo:conPhiDomination} runs in \( \calO(nm) \) total time.
\end{proof}

\section{Implications for the $p$-Center Problem}

The \emph{\( p \)-Center} problems asks for a vertex set~$C$ such that $|C| \leq p$ and the eccentricity of~$C$ is minimal.
It is known (see, \eg,~\cite{BranChepDrag1998}) that the $p$-Center problem and $r$-Domination problem are closely related.
Indeed, one can solve each of these problems by solving the other problem a logarithmic number of times.
Lemma~\ref{lem:rDomToPCenterApprox} below generalises this observation.
Informally, it states that we are able to find a $+ \phi$-approximation for the $p$-Center problem if we can find a good $(r + \phi)$-dominating set.

\begin{lemma}
    \label{lem:rDomToPCenterApprox}
For a given graph~\( G \), let \( D_r \) be an optimal (connected) \( r \)-dominating set and \( C_p \) be an optimal (connected) \( p \)-center.
If, for some non-negative integer~\( \phi \), there is an algorithm to compute a (connected) \( (r + \phi) \)-dominating set~\( D \) with \( |D| \leq |D_r| \) in \( \calO \big( T(G) \big) \) time, then there is an algorithm to compute a (connected) \( p \)-center~\( C \) with \( \ecc(C) \leq \ecc(C_p) + \phi \) in \( \calO \big( T(G) \log n \big) \) time.
\end{lemma}

\begin{proof}
Let $\calA$ be an algorithm which computes a (connected) $(r + \phi)$ dominating set~$D = \calA(G, r)$ for~$G$ with $|D| \leq |D_r|$ in $\calO \big( T(G) \big)$ time.
Then, we can compute a (connected) $p$-center for~$G$ as follows.
Make a binary search over the integers~$i \in [0, n]$.
In each iteration, set $r_i(u) = i$ for each vertex~$u$ of~$G$ and compute the set~$D_i = \calA(G, r_i)$.
Then, increase~$i$ if $|D_i| > p$ and decrease~$i$ otherwise.
Note that, by construction, $\ecc(D_i) \leq i + \phi$.
Let $D$ be the resulting set, \ie, out of all computed sets~$D_i$, $D$ is the set with minimal~$i$ for which~$|D_i| \leq p$.
It is easy to see that finding~$D$ requires at most $\calO \big( T(G) \log n \big)$ time.

Clearly, $C_p$ is a (connected) $r$-dominating set for~$G$ when setting $r(u) = \ecc(C_p)$ for each vertex~$u$ of~$G$.
Thus, for each~$i \geq \ecc(C_p)$, $|D_i| \leq |C_p| \leq p$ and, hence, the binary search decreases~$i$ for next iteration.
Therefore, there is an $i \leq \ecc(C_p)$ such that $D = D_i$.
Hence, $|D| \leq |C_p|$ and $\ecc(D) \leq \ecc(C_p) + \phi$.
\end{proof}

From Lemma~\ref{lem:rDomToPCenterApprox}, the results in Table~\ref{tbl:pCenterResults} and Table~\ref{tbl:ConPCenterResults} follow immediately.

\begin{table}
    [htb]
    \centering
    \caption
    {%
        Implications of our results for the $p$-Center problem.
    }
    \label{tbl:pCenterResults}
%
    \def\arraystretch{1.25}
    \begin{tabular}{lcc}
        \hline
        Approach
            & Approx.
            & Time \\
        \hline
        Layering Partition
            & $+ \Delta$
            & $\calO(m \log n)$ \\
%         \hline
        Tree-Decomposition
            & $+ \rho$
            & $\calO(nm \log n)$ \\
        \hline
    \end{tabular}
\end{table}

\begin{table}
    [htb]
    \centering
    \caption
    {%
        Implications of our results for the Connected $p$-Center problem.
    }
    \label{tbl:ConPCenterResults}
%
    \def\arraystretch{1.25}
    \begin{tabular}{lcc}
        \hline
        Approach
            & Approx.
            & Time \\
        \hline
        Layering Partition
            & $+ 2 \Delta$
            & $\calO(m \, \alpha(n) \log \Delta \log n)$ \\
%         \hline
        Tree-Decomposition
            & $+ \min (5 \rho, 3 \lambda)$
            & $\calO(nm \log n)$ \\
        \hline
    \end{tabular}
\end{table}

Theorem~\ref{theo:pCenterLayPart} below shows that we can slightly improve the result for the $p$-Center problem when using a layering partition.

\begin{theorem}
    \label{theo:pCenterLayPart}
For a given graph~\( G \), a \( + \Delta \)-approximation for the \( p \)-Center problem can be computed in linear time.
\end{theorem}

\begin{proof}
First, create a layering partition~$\calT$ of~$G$.
Second, find an optimal $p$-center~$\calS$ for~$\calT$.
Third, create a set~$S$ by picking an arbitrary vertex of~$G$ from each cluster in~$\calS$.
All three steps can be performed in linear time, including the computation of~$\calS$ (see~\cite{Frederickson1991}).

Let $C$ be an optimal $p$-center for~$G$.
Note that, by Lemma~\ref{lem:LayPartVertDist} (page~\pageref{lem:LayPartVertDist}), $C$ also induces a $p$-center for~$\calT$.
Therefore, because $S$ induces an optimal $p$-center for~$\calT$, Lemma~\ref{lem:LayPartVertDist} (page~\pageref{lem:LayPartVertDist}) implies that, for each vertex~$u$ of~$G$,
\[
    d_G(u, C) \leq d_G(u, S) \leq d_\calT(u, \calS) + \Delta \leq d_\calT(u, C) + \Delta \leq d_G(u, C) + \Delta.
    \qedhere
\]
\end{proof}

