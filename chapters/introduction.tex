\chapter{Introduction}
    \label{cha:intro}
%

Decomposing a graph into a tree is an old concept.
It is introduced already by \textsc{Halin}~\cite{Halin1976}.
However, a more popular introduction is given by \name{Robertson} and \name{Seymour}~\cite{RobertSeymou1983,RobertSeymou1984}.
The idea is to decompose a graph into multiple induced subgraphs, usually called \emph{bags}, where each vertex can be in multiple bags.
These bags are combined to a tree or path in such a way that the following requirements are fulfilled:
Each vertex is in at least one bag, each edge is in at least one bag, and, for each vertex, the bags containing it induce a subtree.
We give formal definitions in Section~\ref{sec:treeDecoDef} (page~\pageref{sec:treeDecoDef}).

For a given graph, there can be up-to exponentially many different tree- or path-decompositions.
The easiest is to have only one bag containing the whole graph.
To make the concept more interesting, it is necessary to add additional restrictions.
The most known is called \emph{tree-width}.
A decomposition has \emph{width}~$\omega$ if each bag contains at most~$\omega + 1$ vertices.
Then, a graph~$G$ has tree-width~$\omega$ (written as $\tw(G) = \omega$) if there is a tree-decomposition for~$G$ which has width~$\omega$ and there is no tree-decomposition with smaller width.

Tree-width is well studied.
Determining the minimal width~$\omega$ for a given graph~$G$ is NP-complete~\cite{ArnbCornPros1987}.
However, if $\omega$ is fixed, it can be checked in linear time if $\tw(G) \leq \omega$~\cite{Bodlae1996}.
The algorithm in~\cite{Bodlae1996} also creates a tree decomposition for~$G$.
For a graph class with bounded tree-with, many NP-complete problems can be solved in polynomial or even linear time.

In the last years, a new perspective on tree-decompositions was invested.
Instead of limiting the number of vertices in each bag, the distance between vertices inside a bag is limited.
There are two major variants: \emph{breadth} and \emph{length}.

The \emph{breadth} of a decomposition is~$\rho$, if, for each bag~$B$, there is a vertex~$v$ such that each vertex in~$B$ has distance at most~$\rho$ to~$v$.
Accordingly, we say the \emph{tree-breadth} of a graph~$G$ is $\rho$ (written as $\tb(G) = \rho$) if there is a tree-decomposition for~$G$ with breadth~$\rho$ and there is no tree-decomposition with smaller breadth.

The \emph{length} of a decomposition is $\lambda$, if the maximal distance of two vertices in each bag is at most~$\lambda$.
Accordingly, we say the \emph{tree-length} of a graph~$G$ is $\lambda$ (written as $\tl(G) = \lambda$) if there is a tree-decomposition for~$G$ with length~$\lambda$ and there is no tree-decomposition with smaller length.

\section{Existing and Recent Results}

The first results on tree-breadth and -length were motivated by tree spanners.
A \emph{tree spanner}~$T$ of a graph~$G$ is a spanning tree for~$G$ such that the distance of two vertices in~$T$ approximates their distance in~$G$.
A natural applications for tree spanners is routing in networks.
Routing messages directly over a tree can be done very efficiently~\cite{ThorupZwick2001}.
Another option is to use the tree as orientation as shown in~\cite{DraganMatama2011}.
Next to routing, tree spanners are used for a protocol locating mobile objects in networks~\cite{PelegReshef2001}.
Other applications can be found for example in biology~\cite{BandelDress1986} and in approximating bandwidth~\cite{VenRotMadMak1997}.

To approximate tree spanners for general graphs, \name{Dragan} and \name{Köhler} introduce tree-breadth in~\cite{DraganKohler2014}.
As additional result, they show a simple algorithm for approximating tree-breadth.
Later, \name{Abu-Ata} and \name{Dragan}~\cite{DraganAbuAta2014} use tree-breadth to construct collective tree spanners for general graphs.
In~\cite{AbuAtaDragan2016}, they also analyse the tree-breadth of real world graph showing that many real world graphs have a small tree-breadth.
The hardness of computing the tree-breadth of a graph is investigated by \name{Ducoffe} et al.\,\cite{DucoLegaNiss2016}.
They show that it is NP-complete to determine, for a given graph~$G$ and any given integer~$\rho \geq 1$, if $\tb(G) \leq \rho$.
Additionally, they present a polynomial-time algorithms to decide if $\tb(G) = 1$ for the case that $G$ is a bipartite or a planar graph.

Tree-length is first introduced by \name{Dourisboure} and \name{Gavoille}~\cite{DourisGavoil2007}.
They investigate the connection to $k$-chordal graphs and present first approximation results.
Later, \name{Dourisboure}~et~al.\,\cite{DouDraGavYan2007} use tree-length to investigate additive spanners.
Additionally, \name{Dourisboure}~\cite{Dourisboure2005} shows how to compute efficient routing schemes for graphs with bounded tree-length.
An other application of tree-length is the Metric Dimension problem.
It asks for the smallest set of vertices such that each vertex in a given graph can be identified by its distances to the vertices in such a set.
\name{Belmonte} et al.\,\cite{BelFomGolRam2016} show that the Metric Dimension problem is Fixed Parameter Tractable for graphs with bounded tree-length.
In~\cite{Lokshtanov2010}, \name{Lokshtanov} investigates the hardness of computing the tree-length of a given graph and shows that it is NP-complete to do so, even for length~$2$.
A connection between tree-width and tree-length is presented by \name{Coudert} et al.\,\cite{CoudDucoNiss2016}.
They show that, for a given graph~$G$, $\tl(G) \leq \lfloor \ell(G) / 2 \rfloor \cdot (\tw(G) - 1)$ where $\ell(G)$ is the length of a longest isometric cycle in~$G$.

\section{Outline}

In this dissertation, we further investigate tree-breadth.
In Chapter~\ref{cha:compDeco}, we present approaches to compute a tree-decomposition with small breadth for a given graph.
This includes approximation algorithms for general graphs as well as optimal solutions for special graph classes.
We also introduce a variant of tree-breadth called \emph{strong tree-breadth}.
Next, in Chapter~\ref{cha:conDomination}, we present algorithms to approach the (Connected) $r$-Domination problem for graphs with bounded tree-breadth.
Our results include (almost) linear time algorithms for the case that no tree-decomposition is given and polynomial time algorithms (with a better solution quality) for the case that a tree-decomposition is given.
In Chapter~\ref{cha:bandwidthLineDist}, we use graphs with bounded path-breadth and -length to construct efficient constant-factor approximation algorithms for the bandwidth and line-distortion problems.
Motivated by these results, we introduce and investigate the new Minimum Eccentricity Shortest Path problem in Chapter~\ref{cha:mesp}.
We analyse the hardness of the problem, show algorithms to compute an optimal solution, and present various approximation algorithms differing in quality and runtime.
This is done for general graphs as well as for special graph classes.

\subsubsection{Note.}
Some of the results in this dissertation were obtained in collaborative work with \name{Feodor F.\ Dragan}\footnote{Kent State University, USA} and \name{Ekkehard Köhler}\footnote{Brandenburgische Technische Universität Cottbus, Germany}.
Results have been published partially
\begin{itemize*}[label={},afterlabel={},mode=unboxed]
    \item
        at~\emph{SWAT}~2014, Copenhagen, Denmark~\cite{DragKoehLeit2014},
    \item
        at~\emph{WADS}~2015, Victoria, Canada~\cite{DraganLeiter2015},
    \item
        at~\emph{WG}~2015, Munich, Germany~\cite{DraganLeiter2015a},
    \item
        at~\emph{COCOA}~2016, Hong Kong, China~\cite{LeiterDragan2016},
    \item
        in the \emph{Journal of Graph Algorithms and Applications}~\cite{DraganLeiter2016}, and
    \item
        in~\emph{Algorithmica}~\cite{DragKoehLeit2017}.
\end{itemize*}
