\chapter{Preliminaries}
    \label{cha:prelim}
%

\section{General Definitions}
If not stated or constructed otherwise, all graphs occurring in this dissertation are connected, finite, unweighted, undirected, without loops, and without multiple edges.
For a graph~$G = (V, E)$, we use $n = |V|$ and $m = |E|$ to denote the cardinality of the vertex set and the edge set of~$G$.

The \emph{length} of a path from a vertex~$v$ to a vertex~$u$ is the number of edges in the path.
The \emph{distance}~$d_G(u, v)$ in a graph~$G$ of two vertices $u$ and~$v$ is the length of a shortest path connecting $u$ and~$v$.
The distance between a vertex~$v$ and a set~$S \subseteq V$ is defined as $d_G(v, S) = \min_{u \in S}  d_G(u, v)$.
Let $x$ and~$y$ be two vertices in a graph~$G$ such that $x$ is most distant from some arbitrary vertex and $y$ is most distant from~$x$.
Such a vertex pair~$x, y$ is called \emph{spread pair} and a shortest path from $x$ to~$y$ is called \emph{spread path}.

The \emph{eccentricity}~$\ecc_G(v)$ of a vertex~$v$ is $\max_{u \in V} d_G(u, v)$.
For a set~$S \subseteq V$, its eccentricity is $\ecc_G(S) = \max_{u \in V} d_G(u, S)$.
For a vertex pair $s,t$, a shortest $(s, t)$-path~$P$ has \emph{minimal eccentricity}, if there is no shortest $(s,t)$-path~$Q$ with $\ecc_G(Q) < \ecc_G(P)$.
Two vertices $x$ and $y$ are called \emph{mutually furthest} if $d_G(x,y) = \ecc_G(x) = \ecc_G(y)$.
A vertex~$u$ is \emph{$k$-dominated} by a vertex~$v$ (by a set~$S \subseteq V$), if $d_G(u, v) \leq k$ ($d_G(u, S) \leq k$, respectively).

The \emph{diameter} of a graph~$G$ is $\diam(G) = \max_{u,v \in V} d_G(u, v)$.
The diameter~$\diam_G(S)$ of a set~$S \subseteq V$ is defined as $\max_{u,v \in S} d_G(u, v)$.
The \emph{radius} of a set~$S \subseteq V$ is defined as $\min_{u \in V} \max_{v \in S} d_G(u, v)$.
A pair of vertices $x,y$ of $G$ is called a \emph{diametral  pair} if $d_G(x, y) = \diam(G)$.
In this case, every shortest path connecting $x$ and~$y$ is called a \emph{diametral path}.

For a vertex~$v$ of~$G$, $N_G(v) = \{ \, u \in V \mid uv \in E \, \}$ is called the \emph{open neighbourhood} of~$v$ and $N_G[v] = N_G(v) \cap \{ v \}$ is called the \emph{closed neighbourhood} of~$v$.
Similarly, for a set~$S \subseteq V$, we define $N_G(S) = \big \{ \, u \in V \mid d_G(u, S) = 1 \, \big \}$.
The \emph{\( l \)-neighbourhood} of a vertex~$v$ in~$G$ is $N_G^l[v] = \big \{ \, u \mid d_G(u, v) \leq l \, \big \}$.
The $l$-neighbourhood of a vertex~$v$ is also called \emph{\( l \)-disk} of~$v$.
Two vertices $u$ and~$v$ are \emph{true twins} if $N_G[u] = N_G[v]$ and are \emph{false twins} if they are non-adjacent and $N_G(u) = N_G(v)$.

The \emph{degree} of a vertex~$v$ is the number of vertices adjacent to it.
If a vertex has degree~$1$, it is called a \emph{pendant} vertex.
A graph~$G$ is called \emph{sparse}, if the sum of all vertex degrees is in~$\calO(n)$.

For some vertex~$v$, $L_i^{(v)} = \big \{ \, u \in V \mid d_G(u,v) = i \, \big \}$ denotes the vertices with distance~$i$ from~$v$.
We will also refer to $L_i^{(v)}$ as the \emph{\( i \)-th layer}.
For two vertices $u$ and~$v$, $I_G(u,v) = \{ \, w \mid d_G(u,v) = d_G(u,w) + d_G(w,v) \, \}$ is the \emph{interval} between $u$ and~$v$.
The set~$S_i(s,t) = L_i^{(s)} \cap I_G(u,v)$ is called a \emph{slice} of the interval from $u$ to~$v$.
For any set~$S \subseteq V$ and a vertex~$v$, $\Pr_G(v, S) = \big \{ \, u \in S \mid d_G(u, v) = d_G(v, S) \, \big \}$ denotes the \emph{projection} of $v$ on~$S$ in~$G$.

For a vertex set~$S$, let $G[S]$ denote the subgraph of~$G$ induced by~$S$.
With $G - S$, we denote the graph $G[V \setminus S]$.
A vertex set~$S$ is a \emph{separator} for two vertices $u$ and~$v$ in~$G$ if each path from $u$ to~$v$ contains a vertex~$s \in S$; in this case we say $S$ \emph{separates} $u$ from~$v$.
If a separator~$S$ contains only one vertex~$s$, \ie, $S = \{ s \}$, then $s$ is an \emph{articulation point}.
A \emph{block} is a maximal subgraph without articulation points.

A \emph{chord} in a cycle is an edge connecting two non-consecutive vertices of the cycle.
A cycle is called \emph{induced} if it has no chords.
For each $k \geq 3$, an induced cycle of length~$k$ is called as~$C_k$.
A subgraph is called \emph{clique} if all its vertices are pairwise adjacent.
A \emph{maximal clique} is a clique that cannot be extended by including any additional vertex.

If clear from context, graph identifying indices of notations may be omitted.
For example, we may write $N[v]$ instead of~$N_G[v]$.


\subsubsection{One-Sided Binary Search.}
Consider a sorted sequence~$\langle x_1, x_2, \ldots, x_{n} \rangle$ in which we search for a value~$x_p$.
We say the value~$x_i$ is at position~$i$.
For a one-sided binary search, instead of starting in the middle at position~$n/2$, we start at position~$1$.
We then processes position~$2$, then position~$4$, then position~$8$, and so on until we reach position~$j = 2^i$ and, next, position~$k = 2^{i+1}$ with $x_j < x_p \leq x_k$.
Then, we perform a classical binary search on the sequence~$\langle x_{j + 1}, \ldots, x_k \rangle$.
Note that, because $x_j < x_p \leq x_k$, $2^i < p \leq 2^{i+1}$ and, hence, $j < p \leq k < 2p$.
Therefore, a one-sided binary search requires at most $\calO(\log p)$ iterations to find~$x_p$.

\section{Tree-Decompositions}
    \label{sec:treeDecoDef}

A \emph{tree-decomposition} for a graph~$G = (V, E)$ is a family~$\calT = \{ B_1, B_2, \ldots \}$ of subsets of~$V$, called \emph{bags}, such that $\calT$ forms a tree with the bags in~$\calT$ as nodes which satisfies the following conditions:
\begin{enumerate}[(i),mode=unboxed]
    \item
        Each vertex is contained in a bag, \ie, $V = \bigcup_{B \in \calB} B$,
    \item
        for each edge~$uv \in E$, $\calT$ contains a bag~$B$ with $u, v \in B$, and
    \item
        for each vertex~$v \in V$, the bags containing $v$ induce a subtree of~$\calT$.
\end{enumerate}
A \emph{path-decomposition} of graph is a tree-decomposition with the restriction that the bags form a path instead of a tree with multiple branches.

\begin{lemma}
    [\name{Tarjan} and \name{Yannakakis}~\cite{TarjanYannak1984}]
    \label{lem:decoValidation}
Let \( \calF \) be a set of bags for some graph~\( G \).
Determining if \( \calF \) forms a tree-decomposition for~\( G \) and, if this is the case, computing the corresponding tree can be done in linear time.
\end{lemma}

It follows from Lemma~\ref{lem:decoValidation} that we can consider a tree-decomposition interchangeably either as family of bags or as structured tree of bags with defined edges and neighbourhoods. 

\begin{lemma}
    [\name{Diestel}~\cite{Diestel1997}]
    \label{lem:cliqueBag}
Let \( K \) be a maximal clique in some graph~\( G \).
Then, each tree-decomposition for~\( G \) contains a bag~\( B \) such that \( K \subseteq B \).
\end{lemma}

\begin{lemma}
    \label{lem:decoBorderBags}
Let \( B \) be a bag of a tree-decomposition~\( T \) for a graph~\( G \) and let \( C \) be a connected component in~\( G - B \).
Then, \( T \) contains a bag~\( B_C \) with \( B_C \supseteq N_G(C) \) and \( B_C \cap C \neq \emptyset \).
\end{lemma}

\begin{proof}
Let $B_C$ be the bag in~$T$ for which $B_C \cap C \neq \emptyset$ and the distance between $B$ and~$B_C$ in~$T$ is minimal.
Additionally, let $B'$ be the bag in~$T$ adjacent to~$B_C$ which is closest to~$B$ and let $S = B_C \cap B'$.
Note that $S \cap C = \emptyset$ and, by properties of tree-decompositions, $S$ separates $C$ from all vertices in~$B \setminus S$.
Assume that there is a vertex~$u \in N_G(C) \setminus S$.
Because $u \in N_G(C)$, there is a vertex~$v \in C$ which is adjacent to~$u$.
This contradicts with $S$ being a separator for $u$ and~$v$.
Therefore, $N_G(C) \subseteq S \subseteq B_C$.
\end{proof}

A tree-decomposition~$T$ of~$G$ has \emph{breadth~\( \rho \)} if, for each bag~$B$ of~$T$, there is a vertex~$v$ in~$G$ with $B \subseteq N_G^\rho[v]$.
The \emph{tree-breadth} of a graph~$G$ is $\rho$, written as $\tb(G) = \rho$, if $\rho$ is the minimal breadth of all tree-decomposition for~$G$.
Similarly, \emph{path-breadth} of a graph~$G$ is $\rho$, written as $\pb(G) = \rho$, if $\rho$ is the minimal breadth of all path-decomposition for~$G$.

A tree-decomposition~$T$ of~$G$ has \emph{length~\( \lambda \)} if, for each bag~$B$ of~$T$, the diameter of~$B$ is at most~$\lambda$, \ie, for all vertices $u, v \in B$, $d_G(u, v) \leq \lambda$.
Accordingly, the \emph{tree-length} of a graph~$G$ is $\lambda$, written as $\tl(G) = \lambda$, if $\lambda$ is the minimal breadth of all tree-decomposition for~$G$.
Similarly, \emph{path-length} of a graph~$G$ is $\lambda$, written as $\pl(G) = \lambda$, if $\lambda$ is the minimal breadth of all path-decomposition for~$G$.

Clearly, it follows from their definitions that, for any graph~$G$, $\tb(G) \leq \tl(G) \leq 2 \tb(G)$, $\pb(G) \leq \pl(G) \leq 2 \pb(G)$, $\tb(G) \leq \pb(G)$, and $\tl(G) \leq \pl(G)$.
Additionally, as shown in Lemma~\ref{lem:contractingEdge} below, non of these parameter increases when an edge of a graph is contracted.

\begin{lemma}
    [\name{Dragan} and \name{Köhler}~\cite{DraganKohler2014}]
    \label{lem:contractingEdge}
Let \( G \) be a graph and \( H \) be a graph created by contracting an edge of~\( G \).
Then, for each~\( \phi \in \{ \tb, \tl, \pb, \pl \} \), \( \phi(H) \leq \phi(G) \).
\end{lemma}


\section{Layering Partitions}

\name{Brandstädt} et al.\,\cite{BranChepDrag1999} and \name{Chepoi}, \name{Dragan} \cite{ChepoiDragan2000} introduced a method called \emph{layering partition}.
The idea is the following.
First, partition the vertices of a given graph~$G$ in distance layers for a given vertex~$s$.
Second, partition each layer~$L_i^{(s)}$ into \emph{clusters} in such a way that two vertices $u$ and~$v$ are in the same cluster if and only if they are connected by a path only using vertices in the same or upper layers.
That is, $u$ and~$v$ are in the same cluster if and only if, for some~$i$, $\{ u, v \} \subseteq L_i^{(s)}$ and there is a path~$P$ from~$u$ to~$v$ in~$G$ such that, for all $j < i$, $P \cap L_j^{(s)} = \emptyset$.
Note that each cluster~$C$ is a set of vertices of~$G$, \ie, $C \subseteq V$, and all clusters are pairwise disjoint.
The created clusters form a rooted tree~$\calT$ with the cluster~$\{ s \}$ as root where each cluster is a node of~$\calT$ and two clusters $C$ and~$C'$ are adjacent in~$\calT$ if and only if $G$ contains an edge~$uv$ with $u \in C$ and~$v \in C'$.
Figure~\ref{fig:LayPartEx} gives an example for such a partition.

\begin{figure}
    [htb]
    \centering

    \begin{subfigure}[b]{0.45\textwidth}
        \centering
        % -----------------------------
% Example for layering partition.

\tikzsetnextfilename{fig_prel_LayPartExG}
\begin{tikzpicture}
    [scale=1.2]

\node[nN] at (0,0) {};
\node[nN] at (0,1) {};
\node[nN] at (1,0) {};
\node[nN] at (1,1) {};
\node[nN] at (2,0) {};
\node[nN] at (2,1) {};
\node[nN] at (2,2) {};
\node[nN] (s) at ($(2,2) + (120:1)$) {};
\node[llbl] at (s.west) {$s$};
\node[nN] (b) at ($(2,0) + (-60:1)$) {};
\node[nN] (t) at ($(2,2) + (60:1)$) {};
\node[nN] at (3,0) {};
\node[nN] at (3,1) {};
\node[nN] at (3,2) {};
\node[nN] at (4,0) {};
\node[nN] at (4,1) {};

\begin{pgfonlayer}
    {background}
    \draw 
        (0,0) -- (1,0) -- (2,0) -- (b.center) -- (3,0) -- (4,0) -- (4,1) -- (3,1) -- (3,2) -- (t.center) -- (s.center) -- (2,2) -- (2,1) -- (1,1) -- (0,1) -- cycle;
    \draw 
        (2,0) -- (2,1) 
        (2,2) -- (t.center) 
        (3,0) -- (3,1);
\end{pgfonlayer}

\end{tikzpicture}

        \caption
        {%
            A graph $G$.
        }
        \label{fig:LayPartExG}
    \end{subfigure}
    \hfil
    \begin{subfigure}[b]{0.45\textwidth}
        \centering
        % -----------------------------
% Example for layering partition.

\tikzsetnextfilename{fig_prel_LayPartExL}
\begin{tikzpicture}
    [
        x=1.2cm,
        y=1.2cm,
    ]

\def\hei{-0.8660}

\node[nN] (s) at (0,0) {}; 
\node[llbl] at ($(s.west)+(-3pt,0)$) {$s$};

\node[nN] (n11) at (-0.5,\hei) {};
\node[nN] (n12) at (0.5,\hei) {};

\node[nN] (n21) at (-1,2*\hei) {};
\node[nN] (n22) at (1,2*\hei) {};

\node[nN] (n31) at (-1.5,3*\hei) {};
\node[nN] (n32) at (-0.5,3*\hei) {};
\node[nN] (n33) at (1.5,3*\hei) {};

\node[nN] (n41) at (-2,4*\hei) {};
\node[nN] (n42) at (-1,4*\hei) {};
\node[nN] (n43) at (0,4*\hei) {};
\node[nN] (n44) at (1,4*\hei) {};
\node[nN] (n45) at (2,4*\hei) {};

\node[nN] (n51) at (-1.5,5*\hei) {};
\node[nN] (n52) at (1.5,5*\hei) {};


\begin{pgfonlayer}
    {lowerBG}
    
    \begin{scope}
        [
            every node/.style=
            {
                fill=clLight60Green!30,
                draw=clDark25Green,
            },
            every fit/.append style=text badly centered
        ]

        \node[fit=(s)] {};
        \node[fit=(n11)(n12)] {};
        \node[fit=(n21)(n22)] {};
        \node[fit=(n31)(n32)(n33)] {};
        \node[fit=(n41)(n42)] {};
        \node[fit=(n43)(n44)(n45)] {};
        \node[fit=(n51)] {};
        \node[fit=(n52)] {};

    \end{scope}
\end{pgfonlayer}

\begin{pgfonlayer}
    {background}

    \draw
        (s.center) -- (2,4*\hei) -- (1.5,5*\hei) -- (1,4*\hei) -- (0,4*\hei) -- (-0.5,3*\hei)
        (s.center) -- (-2,4*\hei) -- (-1.5,5*\hei) -- (-1,4*\hei) -- (-0.5,3*\hei)
        (1,4*\hei) -- (1.5,3*\hei)
        (-1,2*\hei) -- (-0.5,3*\hei)
        (-0.5,\hei) -- (0.5,\hei);
\end{pgfonlayer}

\end{tikzpicture}


        \caption
        {%
            A layering partition~$\calT$ for~$G$.
        }
        \label{fig:LayPartExL}
    \end{subfigure}%

    \caption
    [%
        Example of a layering partition.
    ]
    {%
        Example of a layering partition.
        A given graph~$G$~\subref{fig:LayPartExG} and the layering partition of~$G$ generated when starting at vertex~$s$~\subref{fig:LayPartExL}.
        Example taken from~\cite{ChepoiDragan2000}.
    }
    \label{fig:LayPartEx}
\end{figure}

\begin{lemma}
    [\name{Chepoi} and \name{Dragan}~\cite{ChepoiDragan2000}]
    \label{lem:LayPartLinearTime}
A layering partition of a given graph can be computed in linear time.
\end{lemma}


For a given graph~$G = (V, E)$, let $\calT$ be a layering partition of~$G$ and let $\Delta$ be the maximum cluster diameter of~$\calT$.
For two vertices $u$ and~$v$ of~$G$ contained in the clusters $C_u$ and~$C_v$ of~$\calT$, respectively, we define $d_\calT(u, v) := d_\calT(C_u, C_v)$.

\begin{lemma}
    \label{lem:LayPartVertDist}
For all vertices $u$ and~$v$ of~$G$, $d_\calT(u, v) \leq d_G(u, v) \leq d_\calT(u, v) + \Delta$.
\end{lemma}

\begin{proof}
Clearly, by construction of a layering partition, $d_{\calT}(u, v) \leq d_G(u, v)$ for all vertices $u$ and~$v$ of~$G$.

Next, let $C_u$ and~$C_v$ be the clusters containing $u$ and~$v$, respectively.
Note that $\calT$ is a rooted tree.
Let $C'$ be the lowest common ancestor of $C_u$ and~$C_v$.
Therefore, $d_\calT(u, v) = d_\calT(u, C') + d_\calT(C', v)$.
By construction of a layering partition, $C'$ contains a vertex~$u'$ and vertex~$v'$ such that $d_G(u, u') = d_\calT(u, u')$ and $d_G(v, v') = d_\calT(v, v')$.
Since the diameter of each cluster is at most~$\Delta$, $d_G(u, v) \leq d_\calT(u, u') + \Delta + d_\calT(v, v') = d_{\calT}(u, v) + \Delta$.
\end{proof}

\section{Special Graph Classes}

\subsubsection{Chordal Graphs.}
A graph is \emph{chordal} if every cycle with at least four vertices has a chord.
The class of chordal graphs is a well known class which can be recognised in linear time~\cite{TarjanYannak1984}.
Due to the strong tree structure of chordal graphs, they have the following property
known as $m$-convexity:

\begin{lemma}
     [\name{Faber} and \name{Jamison}~\cite{FaberJamison1986}]
     \label{lem:ChordalMConvex}
Let \( G \) be a chordal graph.
If, for two distinct vertices~\( u,v \) in a disk~\( N^r[x] \), there is a path~\( P \) connecting them with \( P \cap N^r[x] = \{ u, v \} \), then \( u \) and~\( v \) are adjacent.
\end{lemma}

It is well known~\cite{Gavril1974} that a graph~$G$ is chordal if it admits a so-called \emph{clique tree}.
That is, $G$ admits a tree-decomposition~$T$ such that each bag of~$T$ induces a clique.
Such a tree-decomposition can be found in linear time~\cite{Spinrad2003}.

\begin{theorem}
    [\name{Gavril}~\cite{Gavril1974}]
    \label{theo:chordalLength}
A graph~\( G \) is chordal if and only if \( \tl(G) \leq 1 \).
\end{theorem}

A subclass of chordal graphs are \emph{interval graph}.
A graph is an interval graph if it is the intersection graphs of intervals on a line.
It is known~\cite{GilmorHoffma1964} that a graph~$G$ is an interval graph if and only if $G$ admits a path-decomposition~$\calP$ such that each bag of~$\calP$ induces a clique.
Such a decomposition~$\calP$ can be computed in linear time.

\begin{theorem}
    [\name{Gilmore} and \name{Hoffman}~\cite{GilmorHoffma1964}]
    \label{theo:intervalPL}
A graph~\( G \) is an interval graph if and only if \( \pl(G) \leq 1 \).
\end{theorem}


\subsubsection{Dually Chordal Graphs.}
A graph is \emph{dually chordal} if it is the intersection graph of maximal cliques of a chordal graph.
In~\cite{BraDraCheVol1998}, \name{Brandstädt} et al.\ introduce dually chordal graphs and show that they can be recognised in linear time.
It follows from Theorem~\ref{theo:duallyChordDeco} below that dually chordal graphs have tree-breadth~$1$.

\begin{theorem}
    [\name{Brandstädt} et al.\,\cite{BraDraCheVol1998}]
    \label{theo:duallyChordDeco}
A graph~\( G = (V, E) \) is dually chordal if and only if the set~\( \big \{ \, N[v] \mid v \in V \, \big \} \) forms a valid tree-decomposition for~\( G \).
\end{theorem}

\subsubsection{Distance-Hereditary Graphs.}
A graph~$G$ is \emph{distance-hereditary} if the distances in any connected induced subgraph of~$G$ are the same as in~$G$.

\begin{lemma}
    [\name{Bandelt} and \name{Mulder}~\cite{BandelMulder1986}]
    \label{lem:prelDistHere}
Let \( G \) be a distance-hereditary graph, let \( x \) be an arbitrary vertex of~\( G \), and let \( u, v \in L_k^{(x)} \) be in the same connected component of the graph~\( G - L_{k-1}^{(x)} \).
Then, \( N(v) \cap L_{k-1}^{(x)} = N(u) \cap L_{k-1}^{(x)} \).
\end{lemma}

Let $\sigma = \langle v_1, v_2, \ldots, v_n \rangle$ be an ordering for the vertices of some graph~$G$, let $V_i = \{ v_1, v_2, \ldots, v_i \}$, and let $G_i$ denote the graph~$G[V_i]$.
An ordering~$\sigma$ is called a \emph{pruning sequence} for~$G$ if, for $1 \leq j < i \leq n$, each $v_i$ satisfies one of the following conditions in~$G_i$:
\begin{enumerate}[(i)]
    \item
        $v_i$ is a pendant vertex,
    \item
        $v_i$ is a true twin of some vertex~$v_j$, or
    \item
        $v_i$ is a false twin of some vertex~$v_j$.
\end{enumerate}

\begin{lemma}
    [\name{Bandelt} and \name{Mulder}~\cite{BandelMulder1986}]
    \label{lem:punSeq}
A graph~\( G \) is distance-hereditary if and only if there is a pruning sequence for~\( G \).
\end{lemma}

\subsubsection{\texorpdfstring{$\delta$}{δ}-Hyperbolic Graphs.}

A graph has \emph{hyperbolicity~\( \delta \)} if, for any four vertices $u$, $v$, $w$, and~$x$, the two larger of the sums $d(u, v) + d(w, x)$, $d(u, w) + d(v, x)$, and $d(u, x) + d(v, w)$ differ by at most~$2 \delta$.

\begin{lemma}
    [\name{Chepoi} et al.\,\cite{CheDraEstHab2008}]
    \label{lem:deltaHyper}
Let \( u \), \( v \), \( w \), and \( x \) be four vertices in a \( \delta \)-hyperbolic graph.
If \( d(u, w) > \max \big \{ d(u, v), d(v, w) \big \} + 2 \delta \), then \( d(v, x) < \max \big \{ d(x, u), d(x, w) \big \} \).
\end{lemma}

\begin{theorem}
    [\name{Chepoi} et al.\,\cite{CheDraEstHab2008}]
    \label{theo:hyperbolicTreeLength}
If a graph~\( G \) has hyperbolicity~\( \delta \), then \( \delta \leq \tl(G) \) and \( \tl(G) \in \calO(\delta \log n) \).
\end{theorem}


\subsubsection{AT-Free Graphs.}

An \emph{asteroidal triple} (AT for short) in a graph is a set of three vertices where every two of them are connected by a path avoiding the neighbourhood of the third.
A graph is called \emph{AT-free} if it does not contain an asteroidal triple.

A \emph{Lexicographic Breadth-First Search} (LexBFS for short) is a refinement of a standard BFS which produces a vertex ordering~$\sigma \colon V \rightarrow \{ 1, \ldots, n \}$.
A LexBFS starts at some start vertex~$s$, orders the vertices of a graph~$G$ by assigning numbers from $n$ to~$1$ to the vertices in the order as they are discovered by the following search process.
Each vertex~$v$ has a label consisting of a (reverse) ordered list of the numbers of those neighbours of~$v$ that were already visited by the LexBFS; initially this label is empty.
The LexBFS starts with some vertex~$s$, assigns the number~$n$ to~$s$, and adds this number to the end of the label of all unnumbered neighbours of~$s$.
Then, in each step, the LexBFS selects the unnumbered vertex~$v$ with the lexicographically largest label, assigns the next available number~$k$ to~$v$, and adds this number to the end of the labels of all unnumbered neighbours of~$v$.
An ordering~$\sigma$ of the vertex set of a graph generated by a LexBFS is called \emph{LexBFS-ordering}.
Note that the closer a vertex is to the start vertex~$s$ in~$G$, the larger its number is in~$\sigma$.
It is known that a LexBFS-ordering of an arbitrary graph can be generated in linear time~\cite{RoseTarjLuek1976}.

For the two lemmas below, assume that we are given an AT-free graph~$G = (V, E)$, let $s$ be an arbitrary vertex of~$G$, let $x$ be the vertex last visited (numbered~$1$) by a LexBFS starting at~$s$, and let $\sigma$ be the ordering obtained by a LexBFS starting at~$x$.
Additionally, for some vertex~$v \in L_i^{(x)}$, let $N_G^\downarrow(v) = N_G(v) \cap L_{i-1}^{(x)}$.

\begin{lemma}
    [\name{Corneil} et al.\,\cite{CornOlarStew1999}]
    \label{lem:ATfreeDomPair}
Let \( y \) be a vertex of~\( G \) and let \( P \) be a shortest path from~\( x \) to~\( y \).
For each vertex~\( z \) with \( \sigma(y) \leq \sigma(z) \leq \sigma(x) = n \), \( d_G(z, P) \leq 1 \).
\end{lemma}

\begin{lemma}
    \label{lem:ATfreeDownNeigh}
For every integer~\( i \geq 1 \) and every two non-adjacent vertices~\( u, v \in L_i^{(x)} \) of~\( G \), \( \sigma(v) < \sigma(u) \) implies \( N_G^{\downarrow}(v) \subseteq N_G^{\downarrow}(u) \).
In particular, \( d_G(v, u) \leq 2 \) holds for every \( u, v \in L_i^{(x)} \) and every~\( i \).
\end{lemma}

\begin{proof}
Consider an arbitrary neighbour~$w \in L_{i-1}^{(x)}$ of~$v$ and a shortest path~$P$ from $v$ to~$x$ in~$G$ containing~$w$.
Since $\sigma(v) < \sigma(u)$, by Lemma~\ref{lem:ATfreeDomPair}, path~$P$ must dominate vertex~$u$.
Since $u$ and~$v$ are not adjacent, $u$ is in $L_i$, and all vertices of~$P \setminus \{ v, w \}$ belong to layers $L_j^{(x)}$ with $j < i - 1$, vertex~$u$ must be adjacent to~$w$.
Otherwise, $u$, $v$, and~$x$ form an asteroidal triple.
\end{proof}

