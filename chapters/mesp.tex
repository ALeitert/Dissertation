\chapter
[%
    The Minimum Eccentricity Shortest Path Problem%
]
{%
    The Minimum Eccentricity Shortest Path Problem%
    \chapterNote
    {
        Results from this chapter have been published partially
        at~\emph{WADS}~2015, Victoria, Canada~\cite{DraganLeiter2015},
        at~\emph{WG}~2015, Munich, Germany~\cite{DraganLeiter2015a}, and
        in the \emph{Journal of Graph Algorithms and Applications}~\cite{DraganLeiter2016}.
    }%
}
    \label{cha:mesp}
%

In Section~\ref{sec:bw-vs-pl} (page~\pageref{sec:bw-vs-pl}) and Section~\ref{sec:ld-vs-pl} (page~\pageref{sec:ld-vs-pl}), we use the path-length of a graph to obtain a shortest path~$P$ with eccentricity~$\lambda$ and then use $P$ to calculate approximations for the Bandwidth and Line-Distortion problems.
However, after determining~$P$, we do not use the bounded path-length again.
This leads to the question if we can directly determine such a path.
We call this the \emph{Minimum Eccentricity Shortest Path} problem, \emph{MESP} for short.

\begin{definition}
    \label{def:MESP}
For a given a graph~\( G \), the \emph{Minimum Eccentricity Shortest Path} problem asks to find a shortest path~\( P \) such that, for each shortest path~\( Q \), \( \ecc(P) \leq \ecc(Q) \).
\end{definition}

Note that our Minimum Eccentricity Shortest Path problem is close but different from the \emph{Central Path} problem in graphs introduced in~\cite{Slater1982}.
It asks, for a given graph~$G$, to find a not necessarily shortest path~$P$ such that any other path of~$G$ has eccentricity at least~$\ecc(P)$.
The Central Path problem generalizes the Hamiltonian Path problem and, therefore, is NP-complete even for chordal graphs~\cite{Muller1996}.
Our problem, however, is polynomial time solvable for chordal graphs (see Corollary~\ref{cor:mespChordal}).

In this chapter, we investigate the Minimum Eccentricity Shortest Path problem.
We analyse the hardness of the problem, show algorithms to compute an optimal solution, and present approximation algorithms differing in quality and runtime.
This is done for general graphs as well as for special graph classes.
Additionally, we show that, if a shortest path with eccentricity~$k$ is given, a $k$-dominating set can be found in pseudo-polynomial time.

\section{Hardness}
    \label{sec:mesp_Hardness}

In this section, we show that finding a minimum eccentricity shortest path is NP-hard, even if restricted to planar bipartite graphs with maximum vertex-degree~$3$.
Additionally, we show that the problem is W[2]-hard for sparse graphs.
To do that, we define the decision version of the Minimum Eccentricity Shortest Path problem, named \emph{\( k \)-ESP}, as follows:
Given a graph~$G$ and an integer~$k$, does $G$ contain a shortest path~$P$ with eccentricity at most~$k$?

\begin{theorem}
    \label{theo:kDomShortNPc}
The decision version of the Minimum Eccentricity Shortest Path problem is NP-complete.
\end{theorem}

\begin{proof}
To prove Theorem~\ref{theo:kDomShortNPc}, we use a version of 3-SAT which is called \emph{Planar Monotone 3-SAT}.
It was introduced by \name{de Berg} and \name{Khosravi} in~\cite{deBergKhosra2010}.
Consider an instance of 3-SAT given in CNF with the variables~$\calP = \{ p_1, \ldots, p_n \}$ and the clauses~$\calC = \{ c_1, \ldots, c_m \}$.
A clauses is called \emph{positive} if it consists only of positive variables (\ie, $p_a \lor p_b \lor p_c$) and is called \emph{negative} if it consists only of negative variables (\ie, $\neg p_a \lor \neg p_b \lor \neg p_c$).
Consider the bipartite graph~$\calG = (\calP, \calC, \calE)$ where $p_ic_j \in \calE$ if and only if $c_j$ contains $p_i$ or $\neg p_i$.
An instance of 3-SAT is \emph{planar monotone} if each clause is either positive or negative and there is a planar embedding for~$\calG$ such that all variables are on a (horizontal) line~$L$, all positive clauses are above~$L$, all negative clauses are below~$L$, and no edge is crossing~$L$.
Planar Monotone 3-SAT is NP-complete~\cite{deBergKhosra2010}.

Now, assume that we are given an instance~$\calI$ of Planar Monotone 3-SAT with the variables~$\calP = \{ p_1, \ldots, p_n \}$ and the clauses~$\calC = \{ c_1, \ldots, c_m \}$.
Also, let $k = \max \{ n, m \}$.
We create a graph~$G$ as shown in Figure~\ref{fig:kDomShortNPc}.
For each variable~$p_i$ create two vertices, one representing~$p_i$ and one representing~$\neg p_i$.
Create one vertex~$c_i$ for every clause~$c_i$.
Additionally, create two vertices $u_0$, $u_n$ and, for each $i$ with $0 \leq i \leq n$, a vertex~$v_i$.
Connect each variable vertex~$p_i$ and $\neg p_i$ with $v_{i-1}$ and $v_i$ directly with an edge.
Connect each clause with the variables contained in it with a path of length~$k$.
Also connect $v_0$ with $u_0$ and $v_n$ with $u_n$ with a path of length~$k$.

Recall that, by definition of~$\calI$, the corresponding bipartite graph~$\calG$ has a planar embedding where all variables are on a line.
Therefore, we can clearly achieve a planar embedding for~$G$ when placing its vertices as shown in Figure~\ref{fig:kDomShortNPc}.

\begin{figure}
    [htb]
    \centering
    \tikzsetnextfilename{fig_mesp_NPc}
\begin{tikzpicture}
[xscale=0.8660]

\node[sN] (u0) at (-1.5,0) {};
\node[sN] (un) at (8.5,0) {};

\node[sN] (v0) at (0,0) {};
\node[sN] (v1) at (2,0) {};
\node[sN] (vn1) at (5,0) {};
\node[sN] (vn) at (7,0) {};

\node[sN] (p1t) at (1,0.5) {};
\node[sN] (p1b) at (1,-0.5) {};
\node[sN] (pnt) at (6,0.5) {};
\node[sN] (pnb) at (6,-0.5) {};

\node[sN] (c1) at (2,1.5) {};
\node[sN] (ci) at (5,1.5) {};
\node[sN] (cj) at (2,-1.5) {};
\node[sN] (cm) at (5,-1.5) {};


\node[lbl] at (3.5,0) {$\hbox{.}\,\hbox{.}\,\hbox{.}$};


\node[blbl] at (u0.south) {$u_0$};
\node[blbl] at (un.south) {$u_n$};

\node[blbl] at (v0.south) {$v_0$};
\node[blbl] at (v1.south) {$v_1$};
\node[blbl,yshift=-2pt] at (vn1.south) {$v_{n-1}$};
\node[blbl] at (vn.south) {$v_n$};

\node[blbl] at (p1t.south) {$p_1$};
\node[blbl,xshift=-5pt] at (p1b.south) {$\neg p_1$};
\node[blbl] at (pnt.south) {$p_n$};
\node[blbl,xshift=5pt] at (pnb.south) {$\neg p_n$};

\node[tlbl] at (c1.north) {$c_1$};
\node[tlbl] at (ci.north) {$c_i$};
\node[blbl] at (cj.south) {$c_j$};
\node[blbl] at (cm.south) {$c_m$};

% ---------------------

\draw[decorate,decoration={brace,amplitude=3pt,raise=2pt}] (u0.north) -- (v0.north) node[midway,tlbl,yshift=4pt]{$k$};
\draw[decorate,decoration={brace,amplitude=3pt,raise=2pt}] (vn.north) -- (un.north) node[midway,tlbl,yshift=4pt]{$k$};

\begin{pgfonlayer}{background}

\draw (u0.center) -- ++(0.4,0);
\draw (v0.center) -- ++(-0.4,0);
\draw[dashed] (u0) -- (v0);

\draw (v0.center) -- (p1t.center);
\draw (v0.center) -- (p1b.center);

\draw (v1.center) -- (p1t.center);
\draw (v1.center) -- (p1b.center);

\draw (v1.center) -- ++(30:1);
\draw (v1.center) -- ++(-30:1);

\draw (vn1.center) -- ++(150:1);
\draw (vn1.center) -- ++(-150:1);

\draw (vn1.center) -- (pnt.center);
\draw (vn1.center) -- (pnb.center);

\draw (vn.center) -- (pnt.center);
\draw (vn.center) -- (pnb.center);

\draw (vn.center) -- ++(0.4,0);
\draw (un.center) -- ++(-0.4,0);
\draw[dashed] (vn) -- (un);



% \draw[thin,dashed] ($(current bounding box.center)+(1.5pt,1)$) -- (-1.5,1);
% \draw[thin,dashed] ($(current bounding box.center)+(-1.5pt,1)$) -- (8.5,1);
%\draw[thin,grey] (-1.75,0) -- (8.75,0);

% \node[lbl,anchor=south west] at ($(current bounding box.west)+(0,1.5)$) {positive clauses};
% \node[lbl,anchor=north west] at ($(current bounding box.west)+(0,-1.5)$) {negative clauses};

% \node[tlbl] at (current bounding box.north) {positive clauses};
% \node[blbl] at (current bounding box.south) {negative clauses};


\foreach \c/\p in {1/1t,i/nt,j/1b,m/nb}
{
    \draw[white,line width=4pt] (p\p.center) -- (c\c.center);
    \draw[dashed] (p\p.center) -- (c\c.center);
    \draw[] (p\p.center) -- ($(p\p.center)!0.3cm!(c\c.center)$);
    \draw[] (c\c.center) -- ($(c\c.center)!0.3cm!(p\p.center)$);
}


\draw[] (c1.center) -- ++(-30:0.5);
\draw[dashed] (c1.center) -- ++(-30:1);
\draw[] (c1.center) -- ++(-45:0.5);
\draw[dashed] (c1.center) -- ++(-45:1.025);

\draw[] (ci.center) -- ++(210:0.5);
\draw[dashed] (ci.center) -- ++(210:1);
\draw[] (ci.center) -- ++(225:0.5);
\draw[dashed] (ci.center) -- ++(225:1.025);

\draw[] (cj.center) -- ++(30:0.5);
\draw[dashed] (cj.center) -- ++(30:1);
\draw[] (cj.center) -- ++(45:0.5);
\draw[dashed] (cj.center) -- ++(45:1.025);

\draw[] (cm.center) -- ++(-210:0.5);
\draw[dashed] (cm.center) -- ++(-210:1);
\draw[] (cm.center) -- ++(-225:0.5);
\draw[dashed] (cm.center) -- ++(-225:1.025);


\end{pgfonlayer}

\end{tikzpicture}

    \caption
    [%
        Reduction from Planar Monotone 3-SAT to $k$-ESP.
    ]
    {
        Reduction from Planar Monotone 3-SAT to $k$-ESP.
        Illustration to the proof of Theorem~\ref{theo:kDomShortNPc}.
    }
    \label{fig:kDomShortNPc}
\end{figure}

Note that every shortest path in~$G$ not containing $v_0$ and~$v_n$ has an eccentricity larger than~$k$.
Also, a shortest path from $v_0$ to~$v_n$ has length~$2n$ ($d(v_{i-1}, v_i) = 2$, passing through $p_i$ or~$\neg p_i$).
Since $k \geq n$, no shortest path from $v_0$ to~$v_n$ is passing through a vertex~$c_i$; in this case the minimal length would be $2k + 2$.
Additionally, note that, for all vertices in~$G$ except the vertices which represent clauses, the distance to a vertex~$v_i$ with $0 \leq i \leq n$ is at most~$k$.

We now show that $\calI$ is satisfiable if and only if $G$ has a shortest path with eccentricity~$k$.

First, assume $\calI$ is satisfiable.
Let $f\colon \calP \rightarrow \{ T, F \}$ be a satisfying assignment for the variables.
As shortest path~$P$, we choose a shortest path from $v_0$ to $v_n$.
Thus, we have to chose between $p_i$ and $\neg p_i$.
We chose $p_i$ if and only if $f(p_i) = T$.
Because $\calI$ is satisfiable, there is a $p_i$ for each $c_j$ such that either $f(p_i) = T$ and $d(c_j,p_i) = k$, or $f(p_i) = F$ and $d(c_j, \neg p_i) = k$.
Thus, $P$ has eccentricity~$k$.

Next, consider a shortest path~$P$ in~$G$ of eccentricity~$k$.
As mentioned above, $P$ contains either $p_i$ or~$\neg p_i$.
Now, we define $f\colon \calP \rightarrow \{ T, F \}$ as follows:
\[
    f(p_i) =
        \begin{cases}
            T & \text{if $p_i \in P$,} \\
            F & \text{else, \ie $\neg p_i \in P$.} \\
        \end{cases}
\]
Because $P$ has eccentricity~$k$ and only vertices representing a variable in the clause~$c_j$ are at distance~$k$ to vertex~$c_j$, $f$~is a satisfying assignment for~$\calI$.
\end{proof}

While the reduction works in principle for any version of SAT (given as CNF), choosing Planar Monotone 3-SAT allows to construct a planar graph~$G$.

Note that the created graph is bipartite.
Set the colour of each vertex~$v_i$ to black and of each $p_j$ and~$\neg p_j$ to white.
For some vertex~$x$ on the shortest path from $c_i$ to~$p_j$ (or~$\neg p_j$), set the colour of~$x$ based on its distance to~$p_j$ (or $\neg p_j$), \ie, $x$ is white if $d(x, p_i)$ is even and black otherwise.

Additionally, \name{V.\,B.\,Le}\footnote{University of Rostock, Germany} pointed out that, by slightly modifying the created graph as follows, it can be shown that the problem remains NP-complete even if the graph has the maximum vertex-degree~$3$.
First, increase $k$ to $k = \max \{ 2n - 1, m \}$ and update all distances in the graph accordingly.
Therefore, $2k + 2 > 4n - 2$.
Then, replace each vertex~$v_i$ where $1 \leq i \leq n - 1$ with three vertices $v_i^-$, $v_i'$, and~$v_i^+$ such that $N(v_i^-) = \{ p_i, \neg p_i, v_i' \}$, $N(v_i') = \{ v_i^-, v_i^+ \}$, and $N(v_i^+) = \{ p_{i+1}, \neg p_{i+1}, v_i' \}$.
Note that a path from $v_0$ to~$v_n$ which, for all~$i$, passes through $p_i$ or~$\neg p_i$ has length~$4n - 2$.
This is still a shortest path because each path from $v_0$ to~$v_n$ passing through some $c_i$ has length~$2k + 2 > 4n - 2$.
Also, since $d(p_i, p_{i+1}) = 4$, the graph remains bipartite.
Next, to limit the degree of a vertex~$p_i$ (or~$\neg p_i$), instead of connecting it directly to all clauses containing it, make $p_i$ adjacent to the root of a binary tree~$T_i$ with height~$\lceil \log_2 k \rceil$.
Then, connect each clause containing~$p_i$ to a leaf of~$T_i$ using a path with length~$k - \lceil \log_2 k \rceil - 1$ and, last, remove unused branches of~$T_i$.
Because this does not effect planarity or colouring, we get:

\begin{corollary}
    \label{cor:mespNPc_planarBipMaxDeg}
The decision version of the MESP problem remains NP-complete if restricted to planar bipartite graphs with maximum vertex-degree~\( 3 \).
\end{corollary}

We can slightly modify the MESP problem such that a start vertex~$s$ and an end vertex~$t$ of the path are given.
That is, for a given a graph~$G$ and two vertices~$s$ and~$t$, find a shortest $(s,t)$-path~$P$ such that, for each shortest $(s,t)$-path~$Q$, $\ecc(P) \leq \ecc(Q)$.
We call this the $(s,t)$-MESP problem.
From the reduction above, it follows that the decision version of this problem is NP-complete, too.

\begin{corollary}
    \label{cor:stMespNPc}
The decision version of the \( (s,t) \)-MESP problem is NP-complete, even if restricted to planar bipartite graphs with maximum vertex-degree~\( 3 \).
\end{corollary}

Note that the factor~$k$ in the reduction above depends on the input size.
In~\cite{Kratsch2000}, it was already mentioned that, for $k = 1$, the problem can be solved in $\calO \big( n^3m \big)$ time by modifying an algorithm given in~\cite{DeogunKratsc1995}.
There, the problem was called \emph{Dominating Shortest Path} problem.
Therefore, it is an interesting question how hard MESP is if $k$ is bounded by a constant.

To answer this question, we show next that the problem is W[2]-hard in general and for sparse graphs.
Therefore, we do not expect that MESP is Fixed Parameter Tractable, \ie, there is probably no algorithm that finds an optimal solution in $f(k) \, n^{\calO(1)}$ time.
In Section~\ref{sec:mesp_OptSol}, we generalise the result from~\cite{Kratsch2000} to show that MESP can be solved in pseudo-polynomial time.

\begin{theorem}
    \label{theo:mespW2}
The Minimum Eccentricity Shortest Path problem is W[2]-hard.
\end{theorem}

\begin{proof}
To show W[2]-hardness, we make a parametrised reduction from the Dominating Set problem which is known to be W[2]-complete~\cite{DowneyFellow2013}.

Consider a given graph~$G = (V, E)$ with $V = \{ v_1, v_2, \ldots, v_n \}$ and a given~$k$.
Based on $G$ and~$k$, we construct a graph~$H$ (containing $G$ as subgraph) as follows.
Start with $G$ and add $k$~sets of vertices $U_1$, $U_2$, \dots, $U_k$ with $U_i = \Big \{ u^i_1, u^i_2, \ldots, u^i_n \Big \}$.
For each $j$ with $1 \leq j < k$, make a join between $U_j$ and~$U_{j+1}$.
Add the vertices $s$, $s'$, $t$, and~$t'$ and connect $s$ with~$s'$ and $t$ with~$t'$, respectively, with a path of length~$k$.
Additionally, make $s$ adjacent to all vertices in~$U_1$ and make $t$ adjacent to all vertices in~$U_k$.
Connect each vertex~$u_i^j \in U_j$ with each vertex in~$N_G[v_i]$ with a path of length~$k$ for all $j$ with $1 \leq j \leq k$.
Figure~\ref{fig:mespW2} gives an illustration.

\begin{figure}
    [htb]
    \centering
    \tikzsetnextfilename{fig_mesp_W2}
\begin{tikzpicture}

\coordinate (cs) at (0,0);
\coordinate (csp) at (-1.5,0);

\coordinate (cU1) at (1.25,0);
\coordinate (cUj) at (3.25,0);
\coordinate (cUk) at (5.25,0);
\coordinate (cU1jdots) at (2.25,0);
\coordinate (cUjkdots) at (4.25,0);

\coordinate (cV) at (3.25,2.25);

\coordinate (ct) at (6.5,0);
\coordinate (ctp) at (8,0);

\begin{scope}
    [yscale=2]

\draw (cs) -- ($(cU1)+(113.5782:0.5cm)$); 
\draw (cs) -- ($(cU1)+(-113.5782:0.5cm)$); 

\draw[fill=white] (cU1) circle (0.4999cm);
\draw[fill=white] (cUj) circle (0.4999cm);
\draw[fill=white] (cUk) circle (0.4999cm);

\draw (ct) -- ($(cUk)+(66.4218:0.5cm)$); 
\draw (ct) -- ($(cUk)+(-66.4218:0.5cm)$); 

\draw ($(cU1)+(60:0.5cm)$) -- ++(-30:0.5cm);
\draw ($(cU1)+(-60:0.5cm)$) -- ++(30:0.5cm);

\draw ($(cU1)+(90:0.5cm)$) -- ++(0:0.683cm);
\draw ($(cU1)+(-90:0.5cm)$) -- ++(0:0.683cm);


\draw ($(cUj)+(60:0.5cm)$) -- ++(-30:0.5cm);
\draw ($(cUj)+(-60:0.5cm)$) -- ++(30:0.5cm);

\draw ($(cUj)+(120:0.5cm)$) -- ++(210:0.5cm);
\draw ($(cUj)+(-120:0.5cm)$) -- ++(150:0.5cm);

\draw ($(cUj)+(90:0.5cm)-(0.683cm,0)$) -- ++(0:0.683cm) -- ++(0:0.683cm);
\draw ($(cUj)+(-90:0.5cm)-(0.683cm,0)$) -- ++(0:0.683cm) -- ++(0:0.683cm);


\draw ($(cUk)+(120:0.5cm)$) -- ++(210:0.5cm);
\draw ($(cUk)+(-120:0.5cm)$) -- ++(150:0.5cm);

\draw ($(cUk)+(90:0.5cm)$) -- ++(180:0.683cm);
\draw ($(cUk)+(-90:0.5cm)$) -- ++(180:0.683cm);

\end{scope}

\draw [xscale=2.5] (cV) circle (0.5cm);

\path
    (cUj) to [out=90,in=-90] 
    coordinate [pos=0.2] (cUj_Vr2_base)
    coordinate [pos=0.4] (cUj_Vr4_base)
    coordinate [pos=0.6] (cUj_Vr6_base)
    coordinate [pos=0.8] (cUj_Vr8_base)
    ($(cV)+(1.5*0.4,0)$)
    ;

\path
    (cUj) to [out=90,in=-90]
    coordinate [pos=0.2] (cUj_Vl2_base)
    coordinate [pos=0.4] (cUj_Vl4_base)
    coordinate [pos=0.6] (cUj_Vl6_base)
    coordinate [pos=0.8] (cUj_Vl8_base)
    ($(cV)-(1.5*0.4,0)$)
    ;

% \node [above] at (current bounding box.north) {Current seed is \pgfcurrentseed};
\pgfmathsetseed{109940843}
\foreach \i in {2,4,6,8}
{
    \coordinate (cUj_Vr\i) at ($(cUj_Vr\i_base)+(rand*0.15,0)$);
}

\pgfmathsetseed{109940843}
\foreach \i in {2,4,6,8}
{
    \coordinate (cUj_Vl\i) at ($(cUj_Vl\i_base)-(rand*0.15,0)$);
}

\draw
    [
        rounded corners=3pt,
        draw=white,
        double=white,
        double distance=0.4pt,
        line width=0.7pt
    ]
    (cUj)
    -- (cUj_Vl2)
    -- (cUj_Vl4)
    -- (cUj_Vl6)
    -- (cUj_Vl8)
    to[out=110,in=-90,rounded corners=0pt] ($(cV)-(1.5*0.4,0)$);

\draw
    [
        rounded corners=3pt,
        draw=white,
        double=white,
        double distance=0.4pt,
        line width=0.7pt
    ]
    (cUj)
    -- (cUj_Vr2)
    -- (cUj_Vr4)
    -- (cUj_Vr6)
    -- (cUj_Vr8)
    to[out=80,in=-90,rounded corners=0pt] ($(cV)+(1.5*0.4,0)$);

\draw
    [
        rounded corners=3pt,
    ]
    (cUj)
    -- (cUj_Vl2)
    -- (cUj_Vl4)
    -- (cUj_Vl6)
    -- (cUj_Vl8)
    to[out=110,in=-90,rounded corners=0pt] ($(cV)-(1.5*0.4,0)$);

\draw
    [
        rounded corners=3pt,
    ]
    (cUj)
    -- (cUj_Vr2)
    -- (cUj_Vr4)
    -- (cUj_Vr6)
    -- (cUj_Vr8)
    to[out=80,in=-90,rounded corners=0pt] ($(cV)+(1.5*0.4,0)$);

% \foreach \i in {2,4,6,8}
% {
%     \draw[ultra thin,red] (cUj_Vr\i) -- (cUj_Vr\i_base);
%     \draw[ultra thin,red] (cUj_Vl\i) -- (cUj_Vl\i_base);
% }
% \draw[ultra thin,red]
%     (cUj) to [out=90,in=-90] ($(cV)+(1.5*0.4,0)$);
% \draw[ultra thin,red]
%     (cUj) to [out=90,in=-90] ($(cV)-(1.5*0.4,0)$);
% \draw[red] (cUj_Vr6_base) circle (1pt);

% \draw
%     [
%         draw=white,
%         double=white,
%         double distance=0.4pt,
%         line width=0.7pt
%     ]
%     (cUj) to [out=90,in=-90] ($(cV)+(1.5*0.4,0)$);
% 
% \draw
%     [
%         draw=white,
%         double=white,
%         double distance=0.4pt,
%         line width=0.7pt
%     ]
%     (cUj) to [out=90,in=-90] ($(cV)-(1.5*0.4,0)$);
% 
% \draw
%     (cUj) to [out=90,in=-90] ($(cV)+(1.5*0.4,0)$);
% \draw
%     (cUj) to [out=90,in=-90] ($(cV)-(1.5*0.4,0)$);


%\draw
%    [
%        draw=white,
%        double=gray,
%        double distance=0.2pt,
%        line width=0.6pt
%    ]
%    (cV) -- ($(cV)-(2.5*0.5,0)-(3pt,0)$)
%    node[llbl,inner sep=1pt] {$N_G[v_i]$};

\draw[fill=white,xscale=1.5] (cV) circle (0.4cm);
\node [lbl] at ($(cV)-(1.25*0,0)$) {$N_G[v_i]$};

%\node [sN] (vi) at (cV) {};
%\node [rlbl] at (vi.east) {$v_i$};

\pgfmathsetseed{727830286}
\draw[rounded corners=3pt]
    (csp)
    \foreach \x 
        [evaluate=\x as \xx using {0.25*\x-1.5}]
        in {1,...,5}
    {
        -- ($(\xx,rand*0.15)$)
    }
    -- (cs);

\draw[rounded corners=3pt]
    (ct)
    \foreach \x 
        [evaluate=\x as \xx using {0.25*\x+6.5}]
        in {1,...,5}
    {
        -- ($(\xx,rand*0.15)$)
    }
    -- (ctp);

\node[lbl] at (cU1jdots) {$\hbox{.}\,\hbox{.}\,\hbox{.}$};
\node[lbl] at (cUjkdots) {$\hbox{.}\,\hbox{.}\,\hbox{.}$};


\node [sN] (s) at (cs) {};
\node [sN] (sp) at (csp) {};

\node [sN] (t) at (ct) {};
\node [sN] (tp) at (ctp) {};

\node [sN,anchor=south] (uij) at (cUj) {};

\node [tlbl] at (sp.north) {$s'$};
\node [tlbl] at (s.north) {$s$};
\node [tlbl] at (t.north) {$t$};
\node [tlbl] at (tp.north) {$t'$};

\node [blbl] at (cUj) {$u_i^j$};

\node [blbl] at ($(cs)!0.5!(csp)$) {$k$};
\node [blbl] at ($(ct)!0.5!(ctp)$) {$k$};
\node [lbl,anchor=205] at (cUj_Vr6_base) {$k$};

\node [blbl] at ($(cU1)+(0,-1)$) {$U_1$};
\node [blbl] at ($(cUj)+(0,-1)$) {$U_j$};
\node [blbl] at ($(cUk)+(0,-1)$) {$U_k$};

\node [rlbl] at ($(cV)+(1.25,0)$) {$G$};

\end{tikzpicture}


    \caption
    [%
        Reduction from Dominating Set to $k$-ESP.
    ]
    {
        Reduction from Dominating Set to $k$-ESP.
        Illustration to the graph~$H$ as constructed in the proof of Theorem~\ref{theo:mespW2}.
    }
    \label{fig:mespW2}
\end{figure}


Because $d_H \big( s, s' \big) = d_H \big( t, t' \big) = k$, each shortest path in~$H$ not containing $s$ and~$t$ has an eccentricity larger than~$k$.
Also, a shortest path from $s$ to~$t$ has length $k + 1$, intersects all sets~$U_j$, and does not intersect~$V$.

First, assume that $H$ has a shortest path~$P$ with eccentricity~$k$.
By definition of~$P$ and construction of~$H$, for every~$v \in V$, there is a vertex~$u_i^j \in P$ such that $d_H \! \left( v, u_i^j \right) = d_H(v, P) = k$ and, hence, $v \in N_G[v_i]$.
Therefore, the set $D = \Big \{ \, v_i \in V \Bigm| \text{there is a $j$ with $u_i^j \in P$} \, \Big \}$ is a dominating set for~$G$ with cardinality at most~$k$.

Next, assume there is a dominating set~$D$ for~$G$ with cardinality at most~$k$.
Without loss of generality, let $D = \big \{ v_1, v_2, \ldots, v_k \big \}$.
Then, we define $P = \left \{ s, u^1_1, u^2_2, \ldots, u_k^k, t \right \}$.
By construction of~$H$, each vertex~$v \in N_G[v_i]$ is at distance~$k$ to~$u_i^i$.
Thus, because $D$ is a dominating set, there is a vertex~$u_i^i \in P$ for each vertex~$v \in V$ with $d_H \! \left( v, u_i^i \right) = k$.
Therefore, $P$ has eccentricity~$k$ in~$H$.
\end{proof}

Note that the graph~$H$ constructed in the reduction above has at most $\calO \big( kn^2 \big)$ edges.
Thus, we can transform $H$ into a sparse graph~$H'$ by simply adding $\Theta \big( kn^2 \big)$ pendent vertices which are adjacent to~$s$.
Clearly, $H'$ has a shortest path with eccentricity~$k$ if and only if $H$ has a shortest path with eccentricity~$k$.
Thus, we get the following result.

\begin{corollary}
    \label{cor:mespW2Sparse}
The Minimum Eccentricity Shortest Path problem remains W[2]-hard when restricted to sparse graphs.
\end{corollary}

Later, Corollary~\ref{cor:FPTBoundDegree} shows that MESP is Fixed Parameter Tractable for graphs with bounded degree.

\section{Computing an Optimal Solution}
    \label{sec:mesp_OptSol}

In this section, we investigate how to find an optimal solution for MESP.
First, we present a pseudo-polynomial time algorithm to solve MESP on general graphs.
Then, we analyse the problem for sparse graphs.
Additionally, we present an approach to solve MESP for tree-structured graphs and show that, for some graph classes, the problem is solvable in polynomial time.

\subsection{General Graphs}

The next algorithm shows that the $k$-ESP problem remains polynomial for a fixed~$k$.
Our algorithm is a generalisation of the algorithm mentioned in~\cite{Kratsch2000}.
It is based on Lemma~\ref{lem:localLayerDom} below.
Informally, Lemma~\ref{lem:localLayerDom} states that, if a graph has a shortest path~$P$ with eccentricity~$k$ starting at~$s$, each layer~$L_i^{(s)}$ is dominated by a subpath of $P$ of length at most~$2k$.

\begin{lemma}
    \label{lem:localLayerDom}
Let \( P = \big \{ s = v_0, v_1, \ldots, v_l \big \} \) be a shortest path with eccentricity~\( k \), \( v_i \in L_i^{(s)} \), and \( P_{i, k} = \Big \{ v_{\max \{ 0, i - k \}}, \ldots, v_{\min \{ i + k, l \}} \Big \} \).
Then, \( L_i^{(s)} \subseteq N^k[P_{i,k}] \).
\end{lemma}

\begin{proof}
Assume there is a vertex $u \in L_i^{(s)} \setminus N^k[P_{i,k}]$.
Consider any vertex~$v_j \in P \setminus P_{i,k}$.
By the definition of $P_{i,k}$ it follows that $|i-j| > k$.
Thus, because $u \in L_i^{(s)}$ and $v_j \in L_j^{(s)}$, $d(v_j, u) \geq |i-j| > k$.
This contradicts with $P$ having eccentricity~$k$.
\end{proof}

For Algorithm~\ref{algo:kDSP} below, we say a shortest path $\tau = \{ v_{i-k}, \ldots, v_{j} \}$ with $i \leq j \leq i+k$ is a \emph{layer-dominating path for a layer~\( L_i^{(s)} \)} if
\begin{itemize}
    \item
        $v_l \in L_l^{(s)}$ for $i - k \leq l \leq j$,
    \item
        $j < i + k$ implies that there is no edge $v_jw \in E$ with $w \in L_{j + 1}^{(s)}$, and
    \item
        $N^k[\tau] \supseteq L_i^{(s)}$ with $N^k[\tau] := \bigcup_{l = i - k}^{j} N^k[v_l]$.
\end{itemize}
We say that a layer-dominating path $\sigma = \{ v_{i-k}, \ldots, v_{j} \}$ for layer~$L_i^{(s)}$ is \emph{compatible} with a layer-dominating path $\tau = \{ u_{i+1-k}, \ldots, u_{j'} \}$ for layer~$L_{i+1}^{(s)}$ if $j'- j \in \{ 0, 1 \}$ and $v_l = u_l$ for $i + 1 - k \leq l \leq j$.
That is, $\sigma$ and~$\tau$ share a path of length~$j - i - 1 + k$.

\begin{algorithm}
    [htb]
    \caption
    {
        \label{algo:kDSP}
        Determines if there is a shortest path of eccentricity at most~$k$ starting at a given vertex~$s$.
    }
\KwIn
{%
    A graph $G = (V, E)$ and a positive integer~$k$.
}

\KwOut
{%
    A shortest path with eccentricity at most~$k$ if existent in~$G$.
}

Calculate the layers $L_i^{(s)}=\{ \, v \in V \mid d_G(s,v) = i \, \}$ with $0 \leq i \leq \ecc_G(s)$.

\If
{%
    $\ecc_G(s) \leq 2k$
}
{
    For each shortest path~$P$ from~$s$, determine if $\ecc_G(P) \leq k$.
    In this case, \Return~$P$.
    If there is no such $P$, then $G$ does not contain a shortest path of eccentricity at most~$k$ starting at~$s$.
    \label{line:kDSP_smallEcc}
}

\For
{%
    $i = k$ \KwTo $\ecc_G(s) - k$%
    \label{line:kDSP_kDomTupleLoopStart}
}
{
    Create an empty vertex set $V'_i$.

    \ForEach
    {
        layer-dominating path~$\tau$ for layer~$L_{i}^{(s)}$
    }
    {
        Add a vertex~$v_\tau$, representing the path~$\tau$, to $V'_i$.
        \label{line:kDSP_kDomTupleLoopEnd}
    }
}
\ForEach
{%
    $v_\tau \in V'_{\ecc_G(s) - k}$
}
{
    If $N_G^k[\tau] \nsupseteq \bigcup_{j=\ecc_G(s)-k}^{\ecc_G(s)} L_j^{(s)}$, remove $v_\tau$ from $V'_{\ecc_G(s) - k}$.
    \label{line:kDSP_removePrimeVertices}
}

Create a graph $G' = (V', E')$ with $V' = V'_k \cup \cdots \cup V'_{\ecc_G(s) - k}$ and $E' = \{ \, v_\sigma v_\tau \mid \text{$\sigma$ is compatible with $\tau$} \, \}$.
\label{line:kDSP_createEprime}

$G$ contains a shortest path of eccentricity at most~$k$ starting at~$s$ if and only if $G'$ contains a path from a vertex~$v_\sigma \in V'_k$ to a vertex~$v_\tau \in V'_{\ecc_G(s) - k}$.
\label{line:kDSP_searchGprime}
\end{algorithm}

\begin{theorem}
    \label{theo:optMespAlgo}
Algorithm~\ref{algo:kDSP} determines if there is a shortest path of eccentricity at most~\( k \) starting from a given vertex~\( s \) in \( \calO \big( n^{2k+1} m \big) \) time.
\end{theorem}

\begin{proof}
    [Correctness]
To show the correctness of the algorithm, we need to show that line~\ref{line:kDSP_searchGprime} is correct.
Without loss of generality, we can assume that $\ecc_G(s) > 2k$.
Otherwise, the algorithm would have stopped in line~\ref{line:kDSP_smallEcc}.

First, assume that there is a shortest path $P = \{ s = u_0, \ldots, u_l \}$ of length~$l$ in~$G$ with $\ecc_G(P) \leq k$.
Note that $\ecc_G(s) - k \leq l \leq \ecc_G(s)$.
Then, by Lemma~\ref{lem:localLayerDom}, each subpath $\tau = \{ u_{i-k}, \ldots, u_j \}$ ($k \leq i \leq \ecc_G(s) - k$, $j = \min\{l, i+k\}$) is a layer-dominating path for layer~$L_i^{(s)}$.
Additionally, if $j = l$, then $N_G^k[\tau] \supseteq \bigcup_{j=\ecc_G(s) - k}^{\ecc_G(s)}L_j^{(s)}$.
Thus, the algorithm creates a vertex~$v_\tau \in V'_i$ in line~\ref{line:kDSP_kDomTupleLoopEnd} which represents a subpath of~$P$ for each $i$ with $k \leq i \leq \ecc_G(s) - k$.
If $v_\tau \in V'_i$ and $v_\sigma \in  V'_{i+1}$ represent subpaths of $P$, $v_\tau$ and~$v_\sigma$ are adjacent in~$G'$ because $\tau$ and~$\sigma$ are compatible.
Therefore, there is a path in~$G'$ from a vertex in~$V'_k$ to a vertex in~$V'_{\ecc_G(s)-k}$.

Next, assume that $G'$ contains a path~$P'$ from a vertex~$u \in V'_k$ to a vertex~$v \in V'_{\ecc_G(s) - k}$.
Each vertex~$v_\sigma \in V'_i \cap P'$ represents a layer-dominating path for layer~$L_i^{(s)}$ in~$G$.
By definition of layer-dominating paths, if $v_\sigma \in V'_i$ is adjacent to $v_\tau \in V'_{i+1}$, the paths $\sigma$ and~$\tau$ in~$G$ (of length $2k$) can be combined to a longer path (of length $2k + 1$).
If $\tau$ has length less than $2k$, it is a subpath of~$\sigma$.
Thus, $P'$ represents a path~$P$ in~$G$ from $s$ to a vertex~$w \in L_{q}^{(s)}$ with $\ecc_G(s)-k \leq q \leq \ecc_G(s)$.

Each vertex~$v_\tau \in V'_i \cap P'$ represents a layer-dominating path~$\tau$ for layer~$L_i^{(s)}$.
Because of line~\ref{line:kDSP_removePrimeVertices}, $v_\tau \in V'_{\ecc_G(s)-k}$ implies $N_G^k[\tau] \supseteq \bigcup_{j=\ecc_G(s) - k}^{k}L_j^{(s)}$.
Thus, $P$ is a shortest path starting from~$s$ with $\ecc_G(P) \leq k$.
\end{proof}

\begin{proof}
    [Complexity]
If $\ecc_G(s) \leq 2k$, the algorithm stops after line~\ref{line:kDSP_smallEcc}.
In this case there are at most $\calO \big( n^{2k} \big)$ shortest paths starting from~$s$.
Thus, finding a shortest path with eccentricity~$k$ can be done in $\calO \big( n^{2k} m \big)$ time by deciding in $\calO(m)$ time if a path has eccentricity~$k$.

Next, assume $\ecc_G(s) > 2k$.
The graph can only contain $\calO \big( n^{2k+1} \big)$ layer-dominating paths because each such path has at most $2k + 1$~vertices in it.
Therefore, creating the vertices of $G'$ (line~\ref{line:kDSP_kDomTupleLoopStart} to line~\ref{line:kDSP_removePrimeVertices}) can be done in $\calO \big( n^{2k+1} m \big)$ time.

Store the found layer-dominating paths in a forest structure~$\calT$ as follows.
For each vertex~$v$ of~$G$, $\calT$ contains a tree~$T_v$ rooted at~$v$ of depth at most~$2k$.
This tree~$T_v$ stores all layer-dominating paths of~$G$ starting at~$v$.
% The children of~$v$ in~$T_v$ are all neighbours~$u$ of~$v$ in~$G$ such that $d_G(s, u) = d_G(s, v) + 1$.
Any node~$u$ in~$T_v$ (including the root~$v$) at depth less than~$2k$ has as the children all neighbours~$w$ of~$u$ in~$G$ such that $d_G(s, w) = d_G(s, w) + 1$.
Every node of~$T_v$ represents a unique path of~$G$ corresponding to the path of~$T_v$ from the root~$v$ to this node.
A leaf~$t$ of~$T_v$ has a pointer to a layer-dominating path~$\tau$ (and, hence, to the corresponding vertex~$v_\tau$ in~$G'$) if the path~$\tau = \{ v, \ldots, t \}$ from the root~$v$ to the leaf~$t$ forms a layer-dominating path~$\tau$ in~$G$.

Now, given a layer-dominating path $\sigma = \{ v_{i-k}, v_{i-k+1}, \ldots, v_j \}$, we can determine all layer-dominating paths~$\tau$ which $\sigma$ is compatible with in $\calO(m)$ time as follows.
Take the tree~$T_{v_{i-k+1}}$ in~$\calT$ and, following path~$\sigma$, decent to node~$v_j$ of~$T_{v_{i-k+1}}$ representing path $\{ v_{i-k+1}, \ldots, v_j \}$.
Then, leaves of~$T_{v_{i-k+1}}$ attached to~$v_j$ have pointer to all paths~$\tau$ which $\sigma$ is compatible with.

Since $G'$ has at most $\calO \big( n^{2k+1} \big)$ vertices, creating $G'$ (in line~\ref{line:kDSP_createEprime}) takes at most $\calO \big( n^{2k+1} m \big)$ time.
Thus, the overall running time for Algorithm~\ref{algo:kDSP} is $\calO \big( n^{2k+1} m \big)$.
\end{proof}

Algorithm~\ref{algo:kDSP} determines if there is a shortest path of eccentricity at most~$k$ for a given vertex~$s$.
If a start vertex is not given, iterating Algorithm~\ref{algo:kDSP} over each vertex determines if there is a shortest path of eccentricity at most~$k$ in a given graph~$G$ in $\calO \big( n^{2k+2} m \big)$ time.
If $k$ is unknown, a path with minimum eccentricity can be found by trying different values for~$k$ starting with~$1$.
Then, the runtime is $\calO \big( n^4 m \big) + \calO \big( n^6 m \big) + \cdots + \calO \big( n^{2k+2} m \big) = \calO \big( n^{2k+2} m \big)$.

\begin{corollary}
If a given graph~\( G \) contains a shortest path with eccentricity~\( k \), the MESP problem can be solved for~\( G \) in \( \calO \big( n^{2k+2} m \big) \) time, even if \( k \) is unknown.
\end{corollary}

Algorithm~\ref{algo:kDSP} requires $\calO \big( n^{2k + 1} m \big)$ time because there can be up-to $\calO(2k + 1)$ layer-dominating paths with length~$2k$ (see complexity proof of Theorem~\ref{theo:optMespAlgo}).
Now, consider the case that a given graph~$G$ has maximum vertex-degree~$\Delta$.
Therefore, such a graph contains at most $\calO \big( \Delta^{2k} n \big)$ layer-dominating paths of length~$2k$ and, hence, Algorithm~\ref{algo:kDSP} requires at most $\calO \big( \Delta^{2k} nm \big)$ time.
Thus:

\begin{corollary}
    \label{cor:FPTBoundDegree}
The Minimum Eccentricity Shortest Path problem is Fixed Parameter Tractable for graphs with maximum degree~$\Delta$.
If such a graph~\( G \) contains a shortest path with eccentricity~\( k \), the MESP problem can be solved for~\( G \) in \( \calO \big( \Delta^{2k} n^2m \big) \) time, even if \( k \) is unknown.
\end{corollary}


\subsection{Distance-Hereditary Graphs}

In this subsection, we present an algorithm that solves the Minimum Eccentricity Shortest Path problem for distance-hereditary graphs in linear time.
Our algorithm is based on the following result.

\begin{theorem}
    [\name{Dragan} and \name{Leitert}~\cite{DraganLeiter2016}]
    \label{theo:mespDistHereDiameter}
Let \( x, y \) be a diametral pair of vertices of a distance-hereditary graph~\( G \), and \( k \) be the minimum eccentricity of a shortest path in~\( G \).
Then, there is a shortest path~\( P \) between \( x \) and~\( y \) with \( \ecc(P) = k \).
\end{theorem}

Recall that a diametral path in a tree can be found as follows.
Select an arbitrary vertex~$v$.
Find a most distant vertex~$x$ from~$v$ and then a most distant vertex~$y$ from~$x$.
The shortest path from $x$ to~$y$ is a diametral path.
Thus, it follows from Theorem~\ref{theo:mespDistHereDiameter}:

\begin{corollary}
    \label{cor:mespTrees}
For a tree, a shortest path with minimum eccentricity can be computed in linear time by simply performing two BFS calls.
\end{corollary}

It is known~\cite{DraganNicola2000} that a diametral pair of a distance-hereditary graph can be found in linear time.
Hence, according to Theorem~\ref{theo:mespDistHereDiameter}, to find a shortest path of minimum eccentricity in a distance-hereditary graph in linear time, one needs to efficiently extract a best eccentricity shortest path for a given pair of end-vertices.
In what follows, we demonstrate that, for a distance-hereditary graph, such an extraction can be done in linear time as well.

We will need few auxiliary lemmas.

\begin{lemma}
    \label{lem:dhgGateVertex}
In a distance-hereditary graph~\( G \), for each pair of vertices \( s \) and~\( t \), if \( x \) is a vertex on a shortest path from \( v \) to~\( \Pi_v = \Pr(v, I(s,t)) \) with \( d(x, \Pi_v) = 1 \), then \( \Pi_v \subseteq N(x) \).
\end{lemma}

\begin{proof}
Let $p$ and~$q$ be two vertices in $\Pi_v$ and $d(v, \Pi_v) = r$.
By Lemma~\ref{lem:prelDistHere} (page~\pageref{lem:prelDistHere}), $N(p) \cap L_{r-1}^{(v)} = N(q) \cap L_{r-1}^{(v)}$.
Thus, each vertex $x$ on a shortest path from $v$ to $\Pi_v$ with $d(x,\Pi_v) = 1$ (which is in $N(p) \cap L_{r-1}^{(v)}$ by definition) is adjacent to all vertices in $\Pi_v$, \ie, $\Pi_v \subseteq N(x)$.
\end{proof}

For an interval $I(s, t)$ between two vertices $s$ and~$t$, a \emph{slice~\( S_i(s, t) \)} is defined as the set of vertices in~$I(s, t)$ with distance~$i$ to~$s$, \ie, $S_i(s, t) = L_i^{(s)} \cap I(s, t)$.

\begin{lemma}
    \label{lem:dhgSliceJoin}
In a distance-hereditary graph~\( G \), let \( S_i(s, t) \) and \( S_{i+1}(s, t) \) be two consecutive slices of an interval~\( I(s, t) \).
Each vertex in \( S_i(s, t) \) is adjacent to each vertex in~\( S_{i+1}(s,t) \).
\end{lemma}

\begin{proof}
Consider Lemma~\ref{lem:prelDistHere} (page~\pageref{lem:prelDistHere}) from perspective of~$t$.
Thus, $S_i(s,t) \subseteq N(v)$ for each vertex~$v \in S_{i+1}(s,t)$.
Additionally, from perspective of $s$, $S_{i+1}(s,t) \subseteq N(u)$ for each vertex~$u \in S_{i}(s,t)$.
\end{proof}

\begin{lemma}
    \label{lem:dhgProjSliceInter}
In a distance-hereditary graph~\( G \), if a projection~\( \Pi_v = \Pr(v, I(s, t)) \) intersects two slices of an interval~\( I(s, t) \), each shortest \( (s, t) \)-path intersects~\( \Pi_v \).
\end{lemma}

\begin{proof}
Because of Lemma~\ref{lem:dhgGateVertex}, there is a vertex $x$ with $N(x) \supseteq \Pi_v$ and $d(v,x) = d(v, \Pi_v) - 1$.
Thus, $\Pi_v$ intersects at most two slices of interval $I(s,t)$ and those slices have to be consecutive, otherwise $x$ would be a part of the interval.
Let $S_i(s,t)$ and $S_{i+1}(s,t)$ be these slices.
Note that $d(s, x) = i + 1$.
Thus, by Lemma~\ref{lem:prelDistHere} (page~\pageref{lem:prelDistHere}), $N(x) \cap S_i(s,t) = N(u) \cap S_i(s,t)$ for each $u \in S_{i+1}(s,t)$.
Therefore, $S_i(s,t) \subseteq \Pi_v$, \ie, each shortest path from $s$ to~$t$ intersects $\Pi_v$.
\end{proof}

From the lemmas above, we can conclude that, for determining a shortest $(s,t)$-path with minimal eccentricity, a vertex~$v$ is only \emph{relevant} if $d(v, I(s,t)) = \ecc(I(s,t))$ and the projection of $v$ on the interval $I(s,t)$ only intersects one slice.
Algorithm~\ref{algo:MinEccDistHere} below uses this observation to find  such a path in linear time.

\begin{algorithm}
    [htb]
    \caption
    {
        \label{algo:MinEccDistHere}
        Computes a shortest $(s,t)$-path~$P$ with minimal eccentricity for a given distance-hereditary graph~$G$ and a vertex pair~$s,t$.
    }

\KwIn
{
    A distance-hereditary graph~$G = (V, E)$ and two distinct vertices $s$ and~$t$.
}

\KwOut
{
    A shortest path~$P$ from $s$ to~$t$ with minimal eccentricity.
}

Compute the sets $V_i = \{ \, v \mid d(v, I(s, t)) = i \, \}$ for $1 \leq i \leq \ecc(I(s,t))$.

Each vertex $v \notin I(s,t)$ gets a pointer~$g(v)$ initialised with $g(v) := v$ if $v \in V_1$, and $g(v) := \varnothing$ otherwise.

\For
{
    $i := 2$ \KwTo $\ecc(I(s,t))$
    \label{line:dhgFindGateLoop}
}
{
    For each $v \in V_i$, select a vertex~$u \in V_{i-1} \cap N(v)$ and set $g(v) := g(u)$.
    \label{line:dhgFindGateIteration}
}

\ForEach
{
    $v \in V_{\ecc(I(s,t))}$
}
{
    If $N(g(v))$ intersects only one slice of $I(s,t)$, flag $g(v)$ as \emph{relevant}.
    \label{line:dhgFlagRelevant}
}

Set $P := \{ s, t \}$.

\For
{
    $i := 1$ \KwTo $d(s,t) - 1$
    \label{line:dhgSelectPLoop}
}
{
    Find a vertex~$v \in S_i(s,t)$ for which the number of \emph{relevant} vertices in $N(v)$ is maximal.

    Add $v$ to $P$.
    \label{line:dhgSelectPAddV}
}

\end{algorithm}


\begin{lemma}
    \label{lem:path-pair}
For a distance-hereditary graph~\( G \) and an arbitrary vertex pair~\( s, t \), Algorithm~\ref{algo:MinEccDistHere} computes a shortest \( (s, t ) \)-path with minimal eccentricity in linear time.
\end{lemma}

\begin{proof}
The loop in line~\ref{line:dhgFindGateLoop} determines for each vertex~$v$ outside of the interval~$I(s,t)$ a \emph{gate vertex}~$g(v)$ such that $N(g(v)) \supseteq \Pr(v, I(s,t))$ and $d(v, I(s,t)) = d(v, g(v)) + 1$ (see Lemma~\ref{lem:dhgGateVertex}).
From Lemma~\ref{lem:dhgProjSliceInter} and Lemma~\ref{lem:dhgSliceJoin}, it follows that for a vertex~$v$ which is not in $V_{\ecc(I(s, t))}$ or its projection to $I(s,t)$ is intersecting two slices of $I(s, t)$, $d(v, P(s, t)) \leq \ecc(I(s, t))$ for every shortest path $P(s, t)$ between $s$ and~$t$.
Therefore, line~\ref{line:dhgFlagRelevant} only marks $g(v)$ if $v \in V_{\ecc(I(s, t))}$ and its projection $\Pr(v, I(s,t))$ intersects only one slice.
Because only one slice is intersected and each vertex in a slice is adjacent to all vertices in the consecutive slice (see Lemma~\ref{lem:dhgSliceJoin}), in each slice the vertex of an optimal (of minimum eccentricity) path $P$ can be selected independently from the preceding vertex.
If a vertex~$x$ of a slice~$S_i(s,t)$ has the maximum number of \emph{relevant} vertices in $N(x)$, then $x$ is good to put in~$P$.
Indeed, if $x$ dominates all relevant vertices adjacent to vertices of $S_i(s,t)$, then $x$ is a perfect choice to put in $P$.
Else, any vertex~$y$ of a slice~$S_i(s, t)$ is a good vertex to put in $P$.
Hence, $P$ is optimal if the number of \emph{relevant} vertices adjacent to $P$ is maximal.
Thus, the path selected in line~\ref{line:dhgSelectPLoop} to~line~\ref{line:dhgSelectPAddV} is optimal.
\end{proof}

Running Algorithm~\ref{algo:MinEccDistHere} for a diametral pair of vertices of a distance-hereditary graph~$G$, by Theorem~\ref{theo:mespDistHereDiameter}, we get a shortest path of~$G$ with minimum eccentricity.
Thus, we have proven the following result.

\begin{theorem}
    \label{tm:opt-path}
A shortest path with minimum eccentricity of a distance-hereditary graph can be computed in linear time.
\end{theorem}


\subsection{A General Approach for Tree-Structured Graphs}
    \label{sec:mespProjGapApproach}

In a graph~$G$, consider a shortest path~$P$ which starts at a vertex~$s$.
Each vertex~$x$ has a projection~$\Pi_x = \Pr(x, P)$.
In case of a tree this is a single vertex.
However, in general, $\Pi_x$ can contain multiple vertices and does not necessarily induce a connected subgraph.
In this case, there are two vertices $u$ and~$w$ in $\Pi_x$ such that all vertices~$v$ in the subpath~$Q$ between $u$ and~$w$ are not in~$\Pi_x$.
Formally, $u,w \in \Pi_x$, $Q = \{ \, v \in P \mid d(s,u) < d(s,v) < d(s, w) \, \}$, and $Q \cap \Pi_x = \emptyset$.

Now, assume the cardinality of $Q$ is at most~$\gamma$, \ie, $d(u,w) \leq \gamma + 1$ for each $P$, $x$, $u$ and~$w$.
Then, we refer to $\gamma$ as the \emph{projection gap} of $G$.

\begin{definition}
    [%
        Projection Gap%
    ]
In a graph~\( G \), let \( P = \{ v_0, \ldots, v_l \} \) be a shortest path with \( d(v_0, v_i) = i \).
The \emph{projection gap} of \( G \) is \( \gamma \), \( \pg(G) = \gamma \) for short, if, for every vertex~\( x \) of~\( G \) and every two vertices \( v_i, v_k \in \Pr(x, P) \), \( d(v_i, v_k) > \gamma + 1 \) implies that there is a vertex~\( v_j \in \Pr(x, P) \) with \( i < j < k \).
\end{definition}

Based on this definition, we can make the following observation.

\begin{lemma}
    \label{lem:valProp}
In a graph \( G \) with \( \pg(G) = \gamma \), let \( P \) be a shortest path starting at~\( s \), \( Q \) be a subpath of~\( P \), \( |Q| > \gamma \), \( u \) and~\( w \) be two vertices in \( P \setminus Q \) such that \( d(s, u) < d(s,Q) < d(s, w) \), and \( x \) be an arbitrary vertex in~\( G \).
If \( d(x, u) < d(x, Q) \), then \( d(x, w) \geq d(x, Q) \).
\end{lemma}

\begin{proof}
Assume that $d(x, u) < d(x, Q)$ and $d(x, w) < d(x, Q)$.
Without loss of generality, let $d(x, u) = d(x, w) < d(x, v)$ for all $v \in P$ with $d(s,u) < d(s,v) < d(s, w)$.
Let $P'$ be the subpath of $P$ from $u$ to~$w$.
Note that $\Pr(x, P') = \{ u, w \}$ and $Q \subset P'$.
Thus, $d(u, w) \geq |Q| + 1 > \gamma + 1$.
This contradicts with $\pg(G) = \gamma$.
\end{proof}

Informally, Lemma~\ref{lem:valProp} says that, when exploring a shortest path~$P$, if the distance to a vertex~$x$ did not decrease during the last $\gamma + 1$ vertices of~$P$, it will not decrease when exploring the remaining subpath.
Based on this, we show that a minimum eccentricity shortest path can be found in polynomial time if $\pg(G)$ is bounded by some constant.
Additionally, we show that for some graph classes the projection gap has an upper bound leading to polynomial time algorithms for these classes.

\subsubsection{Algorithm.}

For the remainder of this subsection, we assume that we are given a graph~$G$ with $\pg(G) = \gamma$ containing a vertex~$s$.
We need the following notions and notations:
\begin{itemize}
\item
$Q_{i}$ and~$Q_j$ are subpaths of length~$\gamma$ of some shortest paths starting at~$s$.
They do not need to be subpath of the same shortest path.
Let $v_i \in Q_i$ and $v_j \in Q_j$ be the two vertices such that $d(s, Q_i) = d(s, v_i)$ and  $d(s, Q_j) = d(s, v_j)$.
Without loss of generality, let $d(s, v_i) \leq d(s, v_j)$.
We say, $Q_{i}$ is \emph{compatible} with $Q_j$ (with respect to~$s$) if $|Q_{i} \cap Q_j| = \gamma - 1$, $v_{i}$ is adjacent to~$v_j$, and $d(s, v_i) < d(s, v_j)$.
Let $\calC_s(Q_j)$ denote the set of subpaths compatible with~$Q_j$.

\item
$R_s(Q_j) = \{ \, w \mid Q_j \subseteq I(s, w) \, \} \cup Q_j$ is the set of vertices~$w$ such that there is a shortest path from $s$ to~$w$ containing $Q_j$ (or $w \in Q_j$).

\item
$I_s(Q_j) = I(s, v_j) \cup Q_j$ are the vertices that are on a shortest path from $s$ to~$Q_j$ (or in~$Q_j$).

\item
$V^\downarrow_s(Q_j) = \{ \, x \mid d(x, Q_j) = d(x, R_s(Q_j)) \, \}$ is the set of vertices~$x$ which are closer to~$Q_j$ than to all other vertices in~$R_s(Q_j)$.
Thus, given a shortest path~$P$ containing~$Q_j$ and starting at~$s$, expanding $P$ beyond $Q_j$ will not decrease the distance from $x$ to~$P$.
\end{itemize}
Note that $Q_j \subseteq V^\downarrow_s(Q_j)$ and $Q_j = I_s(Q_j) \cap R_s(Q_j)$.

\begin{lemma}
    \label{lem:kValeyVert}
For each vertex~\( x \) in~\( G \), \( d(x, Q_j) = d(x, I_s(Q_j)) \) or \( d(x, Q_j) = d(x, R_s(Q_j)) \).
\end{lemma}

\begin{proof}
Assume that $d(x, Q_j) > d(x, I_s(Q_j))$ and $d(x, Q_j) > d(x, R_s(Q_j))$.
Then, there is a vertex~$u_i \in I_s(Q_j)$ and a vertex~$u_r \in R_s(Q_j)$ with $d(x, u_i) < d(x, Q_j)$ and $d(x, Q_j) > d(x, u_r)$.
Because $u_i$, $Q_j$, and $u_r$ are on a shortest path starting at~$s$ and $|Q_j| > \gamma$, this contradicts Lemma~\ref{lem:valProp}.
\end{proof}

\begin{lemma}
    \label{lem:kValeyVertSubset}
If \( Q_{i} \) is compatible with \( Q_j \), then \( V^\downarrow_s(Q_{i}) \subseteq V^\downarrow_s(Q_j) \).
\end{lemma}

\begin{proof}
Assume that $V^\downarrow_s(Q_{i}) \nsubseteq V^\downarrow_s(Q_j)$, \ie, there is a vertex~$x \in V^\downarrow_s(Q_{i}) \setminus V^\downarrow_s(Q_j)$.
Then, $d(x, Q_j) > d(x, R_s(Q_j))$.
Thus, by Lemma~\ref{lem:kValeyVert}, $d(x, Q_j) = d(x, I_s(Q_j))$.
Because $Q_{i} \subseteq I_s(Q_j)$, $d(x, Q_{i}) \geq d(x, I_s(Q_j)) = d(x, Q_j)$.
Since $x \in V^\downarrow_s(Q_{i})$, $d(x, Q_{i}) = d(x, R_s(Q_{i}))$.
Also, because $x \notin V^\downarrow_s(Q_{i})$, $d(x, Q_j) > d(x, R_s(Q_j))$.
Thus, $d(x, R_s(Q_{i})) > d(x, R_s(Q_j))$.
On the other hand, because $R_s(Q_{i}) \supseteq R_s(Q_j)$, $d(x, R_s(Q_{i})) \leq d(x, R_s(Q_j))$ and a contradiction arises.
\end{proof}

For a subpath $Q_j$, let $\calP_s(Q_j)$ denote the set of shortest paths~$P$ which start at~$s$ such that $Q_j \subseteq P \subseteq I_s(Q_j)$.
Then, we define $\varepsilon_s(Q_j)$ as follows:
\[
    \varepsilon_s(Q_j) = \min_{P \in \calP_s(Q_j)} \max_{x \in V^\downarrow_s(Q_j)} d(x, P)
\]
Consider a subpath $Q_j$ for which $R_s(Q_j) = Q_j$, \ie, a shortest path containing $Q_j$ cannot be extended any more.
Then, $V^\downarrow_s(Q_j) = V$.
Therefore, for any path $P \in \calP_s(Q_j)$, $\max_{x \in V^\downarrow_s(Q_j)} d(x, P) = \ecc(P)$.

\begin{lemma}
    \label{lem:valEpsilon}
If \( \calC_s(Q_j) \) is non-empty, then
\[
    \varepsilon_s(Q_j)
    =
    \min_{Q_{i} \in \calC_s(Q_j)} \max \left[
        \max_{x \in V^\downarrow_s(Q_j) \setminus V^\downarrow_s(Q_{i})} \min \big( d(x, Q_{i}), d(x, Q_{j}) \big),
        \varepsilon_s(Q_{i})
    \right].
\]
\end{lemma}

\begin{proof}
By definition,
\[
    \varepsilon_s(Q_j)
    =
    \min_{P \in \calP_s(Q_j)} \max_{x \in V^\downarrow_s(Q_j)} d(x, P).
\]
Let $Q_{i}$ be compatible with~$Q_j$.
Because, by Lemma~\ref{lem:kValeyVertSubset}, $V^\downarrow_s(Q_{i}) \subseteq V^\downarrow_s(Q_j)$, we can partition $V^\downarrow_s(Q_j)$ into $V^\downarrow_s(Q_j) \setminus V^\downarrow_s(Q_{i})$ and $V^\downarrow_s(Q_{i})$.
For simplicity, we write $V^\downarrow_s(Q_j) \setminus V^\downarrow_s(Q_{i})$ as $V^\downarrow_s[Q_j|Q_i]$.
Thus, $\varepsilon_s(Q_j) =$
\[
    \min_{Q_{i} \in \calC_s(Q_j)} \min_{P \in \calP_s(Q_{i})} \max \left[
        \max_{x \in V^\downarrow_s[Q_j|Q_i]} d(x, P \cup Q_j),
        \max_{x \in V^\downarrow_s(Q_{i})} d(x, P \cup Q_j)
    \right].
\]
Note that we changed the definition of~$P$ from $P \in \calP_s(Q_j)$ to $P \in \calP_s(Q_{i})$, \ie, $P$ may not contain the last vertex of $Q_j$ any more.

If $x \in V^\downarrow_s[Q_j|Q_i]$, then $d(x, Q_{i}) > d(x, R_s(Q_{i}))$.
Thus, by Lemma~\ref{lem:kValeyVert}, $d(x, Q_{i}) = d(x, I_s(Q_{i}))$.
Note that, by definition of~$P$, $d(x, Q_i) \geq d(x, P) \geq d(x, I_s(Q_i))$.
Therefore, $d(x, P) = d(x, Q_i)$ and
\[
    \max_{x \in V^\downarrow_s[Q_j|Q_i]} d(x, P \cup Q_j)
    =
    \max_{x \in V^\downarrow_s[Q_j|Q_i]} \min \big( d(x, Q_{i}), d(x, Q_{j}) \big).
\]
For simplicity, we define
\[
    \varepsilon_s(Q_i, Q_j)
    :=
    \max_{x \in V^\downarrow_s[Q_j|Q_i]} \min \big( d(x, Q_{i}), d(x, Q_{j}) \big).
\]
Note that $\varepsilon_s(Q_i, Q_j)$ does not depend on $P$.
Therefore, because $\min_{u} \max [ c, f(u) ] = \max [ c, \min_{u} f(u) ]$,
\[
    \varepsilon_s(Q_j)
    =
    \min_{Q_{i} \in \calC_s(Q_j)} \max \left[
        \varepsilon_s(Q_i, Q_j),
        \min_{P \in \calP_s(Q_{i})} \max_{x \in V^\downarrow_s(Q_{i})} d(x, P \cup Q_j)
    \right].
\]
If $x \in V^\downarrow_s(Q_{i})$, then $d(x, Q_{i}) = d(x, R_s(Q_{i})) \leq d(x, R_s(Q_j)) = d(x, Q_j)$.
Therefore,
\[
    \min_{P \in \calP_s(Q_{i})} \max_{x \in V^\downarrow_s(Q_{i})} d(x, P \cup Q_j)
    =
    \min_{P \in \calP_s(Q_{i})} \max_{x \in V^\downarrow_s(Q_{i})} d(x, P)
    =
    \varepsilon_s(Q_{i}).
\]
Thus,
\[
    \varepsilon_s(Q_j)
    =
    \min_{Q_{i} \in \calC_s(Q_j)} \max \left[
        \max_{x \in V^\downarrow_s(Q_j) \setminus V^\downarrow_s(Q_{i})} \min \big( d(x, Q_{i}), d(x, Q_{j}) \big),
        \varepsilon_s(Q_{i})
    \right].
    \qedhere
\]
\end{proof}

Based on Lemma~\ref{lem:valEpsilon}, Algorithm~\ref{algo:valMESP} computes a shortest path starting at~$s$ with minimal eccentricity.
The algorithm has two parts.
First, it computes the pairwise distance of all vertices and $d(x, R_s(v))$ for each vertex pair $x$ and~$v$ where, similarly to $R_s(Q_j)$, $R_s(v) = \{ \, z \in V \mid v \in I(s, z) \, \}$.
This allows to easily determine if a vertex~$x$ is in~$V^\downarrow_s(Q_j)$.
Second, it computes $\varepsilon_s(Q_j)$ for each subpath~$Q_j$.
For this, the algorithm uses dynamic programming.
After calculating $\varepsilon_s(Q_i)$ for all subpaths with distance~$i$ to~$s$, the algorithm uses Lemma~\ref{lem:valEpsilon} to calculate $\varepsilon_s(Q_j)$ for all subpaths~$Q_j$ which $Q_i$ is compatible with.

\begin{algorithm}
    [htb!]
    \caption
    {
        Determines, for a given graph~$G$ with $\pg(G) \leq \gamma$ and a vertex~$s$, a minimal eccentricity shortest path starting at~$s$.
    }
    \label{algo:valMESP}

\KwIn
{
    A graph~$G = (V, E)$, an integer~$\gamma$, and a vertex~$s \in V$.
}

\KwOut
{
    A shortest path~$P$ starting at~$s$ with minimal eccentricity.
}

Determine the pairwise distances of all vertices.
\label{line:pairwDistance}

\ForEach
{
    $v, x \in V$
}
{
    Set $d(x, R_s(v)) := d(x, v)$.
    \label{line:defaultR}
}

\For
{
    $i = \ecc(s) - 1$ \KwDownTo $0$
    \label{line:compRsStart}
}
{
    \ForEach
    {
        $v \in L_i^{(s)}$
    }
    {
        \ForEach
        {
            $w \in N(v) \cap L_{i+1}^{(s)}$
        }
        {
            \ForEach
            {
                $x \in V$
            }
            {
                Set $d(x, R_s(v)) := \min \Big[ d \big( x, R_s(v) \big), d \big( x, R_s(w) \big) \Big]$.
                \label{line:distRv}
            }
        }
    }
}

\For
{
    $j = 0$ \KwTo $\ecc(s) - \gamma$
    \label{line:QjLoop}
}
{
    \ForEach
    {
        $Q_j$ with $d(s, Q_j) = j$
        \label{line:selectQj}
    }
    {
        \ForEach
        {
            $x \in V$
            \label{line:xQjLoop}
        }
        {
            Let $z_j$ be the vertex in $Q_j$ with the largest distance to~$s$.
            If $d(x, Q_j) \leq d(x, R_s(z_j))$, add $x$ to $V^\downarrow_s(Q_j)$ and store $d(x, Q_j)$.
            \label{line:computeVsOj}
        }

        \If
        {
            $j = 0$
        }
        {
            $\displaystyle \varepsilon_s(Q_j) := \max_{x \in V^\downarrow_s(Q_j)} d(x, Q_j)$
            \label{line:compStartEpsilon}
        }
        \Else
        {
            $\displaystyle \varepsilon_s(Q_j) := \infty$
            \label{line:QjEpsilonInf}
        }

        \ForEach
        {
            $Q_i \in \calC_s(Q_j)$
            \label{line:QiLoop}
        }
        {
            $\displaystyle
                \varepsilon'_s(Q_j) :=
                \max \left[
                    \max_{x \in V^\downarrow_s(Q_j) \setminus V^\downarrow_s(Q_i)} \min \big( d(x, Q_i), d(x, Q_j) \big),
                    \varepsilon_s(Q_i)
                \right]
            $
            \label{line:compEpsilonPrime}

            \If
            {
                $\varepsilon'_s(Q_j) < \varepsilon_s(Q_j)$
                \label{line:comareEpsilon}
            }
            {
                Set $\varepsilon_s(Q_j) := \varepsilon'_s(Q_j)$ and $p(Q_j) := Q_i$.
                \label{line:setP}
            }
        }
    }
}

Find a subpath~$Q_j$ such that a shortest path containing $Q_j$ cannot be extended any more and for which $\varepsilon_s(Q_j)$ is minimal.
\label{line:findMinQj}

Construct a path~$P$ from $Q_j$ to $s$ using the $p()$-pointers and output it.
\label{line:constructP}
\end{algorithm}

\begin{theorem}
For a given graph~\( G \) with \( \pg(G) = \gamma \) and a vertex~\( s \), Algorithm~\ref{algo:valMESP} computes a shortest path starting at~\( s \) with minimal eccentricity.
It runs in \( \calO \big( n^{\gamma + 3} \big) \) time if \( \gamma \geq 2 \), in \( \calO \big( n^2 m \big) \) time if \( \gamma = 1 \), and in \( \calO(nm) \) time if \( \gamma = 0 \).
\end{theorem}

\begin{proof}
    [Correctness]
The algorithm has two parts.
The first part (line~\ref{line:pairwDistance} to line~\ref{line:distRv}) is a preprocessing which computes $d(x, R_s(v))$ for each vertex pair $x$ and~$v$.
The second part computes $\varepsilon_s(Q_j)$ which is then used to determine a path with minimal eccentricity.

For the first part, without loss of generality, let $d(s, v) = i$, $N^\uparrow_s(v) = N(v) \cap L_{i+1}^{(s)}$, and let $x$ be an arbitrary vertex.
By definition of $R_s$, $N^\uparrow_s(v) = \emptyset$ implies $R_s(v) = \{ v \}$, \ie, $d(x, R_s(v)) = d(x, v)$.
Therefore, $d(x, R_s(v))$ is correct for all vertices~$v$ with $N^\uparrow_s(v) = \emptyset$ after line~\ref{line:defaultR}.
By induction, assume that $d(x, R_s(w))$ is correct for all vertices~$w \in N^\uparrow_s(v)$.
Because $R_s(v) = \bigcup_{w \in N^\uparrow_s(v)} R_s(w) \cup \{ v \}$, $d(x, R_s(v)) = \min(\min_{w \in N^\uparrow_s(v)} d(x, R_s(w)), d(x, v))$.
Therefore, line~\ref{line:distRv} correctly computes~$d(x, R_s(v))$.

The second part of Algorithm~\ref{algo:valMESP} iterates over all subpaths~$Q_j$ in increasing distance to~$s$.
Line~\ref{line:computeVsOj} checks if a given vertex~$x$ is in $V^\downarrow_s(Q_j)$.
By definition, $R_s(Q_j) = Q_j \cup R_s(z_j)$ where $z_j$ is the vertex in $Q_j$ with the largest distance to~$s$.
Thus, $d(x, R_s(Q_j)) = \min( d(x, R_s(z_j)), d(x, Q_j))$.
By definition of $V^\downarrow_s$, $x \in V^\downarrow_s(Q_j)$ if and only if $d(x, Q_j) = d(x, R_s(Q_j))$.
Therefore, $x \in V^\downarrow_s(Q_j)$ if and only if $d(x, Q_j) \leq d(x, R_s(z_j))$, \ie, line~\ref{line:computeVsOj} computes $V^\downarrow_s(Q_j)$ correctly.

Recall the definition of $\varepsilon_s(Q_j)$:
\[
    \varepsilon_s(Q_j) = \min_{P \in \calP_s(Q_j)} \max_{x \in V^\downarrow_s(Q_j)} d(x, P)
\]
If $d(s, Q_j) = 0$, $d(x, P) = d(x, Q_j)$.
Therefore, $\varepsilon_s(Q_j) = \max_{x \in V^\downarrow_s(Q_j)} d(x, Q_j)$ as computed in line~\ref{line:compStartEpsilon}.
Note that there is no subpath~$Q_i$ which is compatible with $Q_j$, if $d(s, Q_j) = 0$.
Therefore, the loop starting in line~\ref{line:QiLoop} is skipped for these~$Q_j$.
Thus, the algorithm correctly computes $\varepsilon_s(Q_j)$, if $d(s, Q_j) = 0$.

By induction, assume that $\varepsilon_s(Q_i)$ is correct for each $Q_i \in \calC(Q_j)$.
Thus, Lemma~\ref{lem:valEpsilon} can be used to compute $\varepsilon_s(Q_j)$.
This is done in the loop starting in line~\ref{line:QiLoop}.
Therefore, at the beginning of line~\ref{line:findMinQj}, $\varepsilon_s(Q_j)$ is computed correctly for each subpath~$Q_j$.

Recall that, if $P \in \calP(Q_j)$ and $R_s(Q_j) = Q_j$, then $V^\downarrow_s(Q_j) = V$ and, therefore, $\max_{x \in V^\downarrow_s(Q_j)} d(x, P) = \ecc(P)$.
Thus, $R_s(Q_j) = Q_j$ implies that $\varepsilon_s(Q_j)$ is the minimal eccentricity of all shortest paths starting in~$s$ and containing~$Q_j$.
Therefore, if $Q_j$ is picked by line~\ref{line:findMinQj}, then $\varepsilon_s(Q_j)$ is the minimal eccentricity of all shortest paths starting in~$s$.
\end{proof}

\begin{proof}
    [Complexity]
First, we analyse line~\ref{line:pairwDistance} to line~\ref{line:distRv}.
Line~\ref{line:pairwDistance} runs in $\calO(nm)$ time.
This allows to access the distance between two vertices in constant time.
Thus, the total running time for line~\ref{line:defaultR} is~$\calO(n)$.
Because line~\ref{line:distRv} is called at most once for each vertex~$x$ and edge~$vw$, implementing line~\ref{line:compRsStart} to line~\ref{line:distRv} can be done in $\calO(nm)$ time.

For the second part of the algorithm (starting in line~\ref{line:QjLoop}), if $\gamma \geq 2$, let all subpaths be stored in a trie as follows:
There are $\gamma + 1$ layers of internal nodes.
Each internal node is an array of size~$n$ (one entry for each vertex) and each entry points to an internal node of the next layer representing $n$~subtrees.
This requires $\calO \big( n^{\gamma + 1} \big)$ memory.
Leafs are objects representing a subpath.

If $\gamma = 1$, a subpath is a single edge, and, if $\gamma = 0$, a subpath is a single vertex.
Thus, no extra data structure is needed for these cases.
In all three cases, a subpath can be accessed in $\calO(\gamma)$ time.

Next, we analyse the runtime of line~\ref{line:xQjLoop} to line~\ref{line:QjEpsilonInf} for a single subpath~$Q_j$.
Accessing $Q_j$ can be done in $\calO(\gamma)$ time.
Line~\ref{line:computeVsOj} requires at most $\calO(\gamma)$ time for a single call and is called at most $\calO(n)$ times.
Line~\ref{line:compStartEpsilon} requires $\calO(n\gamma)$ time and line~\ref{line:QjEpsilonInf} runs in constant time.
Therefore, for a given subpath, line~\ref{line:xQjLoop} to line~\ref{line:QjEpsilonInf} require $\calO(\gamma n + n)$ time.

For line~\ref{line:compEpsilonPrime} to line~\ref{line:setP}, consider a given pair of compatible subpaths $Q_i$ and~$Q_j$.
Accessing both subpaths can be done in $\calO(\gamma)$ time.
Assuming the vertices in $V^\downarrow_s(Q_i)$ and~$V^\downarrow_s(Q_j)$ are sorted and stored with their distance to $Q_i$ and~$Q_j$, line~\ref{line:compEpsilonPrime} requires at most $\calO(n)$ time.
Note that $Q_i$ and $Q_j$ intersect in $\gamma - 1$ vertices.
Thus, $\min \big( d(x, Q_i), d(x, Q_j) \big) =  \min \big( d(x, v_i), d(x, Q_j) \big)$ where $v_i$ is the vertex in $Q_i$ closest to~$s$.
Line~\ref{line:comareEpsilon} and line~\ref{line:setP} run in constant time.
Therefore, for a given pair of compatible subpaths, line~\ref{line:compEpsilonPrime} to line~\ref{line:setP} require $\calO(n)$ time.

Let $\phi$ be the number of subpaths and $\psi$ be the number of pairs of compatible subpaths.
Then, the overall runtime for line~\ref{line:QjLoop} to line~\ref{line:setP} is $\calO(\phi (\gamma n + n) + \psi n)$ time, $\calO(\phi)$ time for line~\ref{line:findMinQj}, and $\calO(n)$ time for line~\ref{line:constructP}.
Together with the first part of the algorithm, the total runtime of Algorithm~\ref{algo:valMESP} is $\calO(mn + \phi (\gamma n + n) + \psi n)$.

Because a subpath contains $\gamma + 1$ vertices, there are up-to $\calO \big( n^{\gamma + 1} \big)$ subpaths and up-to $\calO \big( n^{\gamma + 2} \big)$ compatible pairs if $\gamma \geq 2$, \ie, $\phi \leq n^{\gamma + 1}$ and $\psi \leq n^{\gamma + 2}$.
Therefore, if $\gamma \geq 2$, Algorithm~\ref{algo:valMESP} runs in $\calO \big( n^{\gamma+3} \big)$ time.

If $\gamma = 1$, a subpath is a single edge and there are at most $mn$ compatible pairs of subpaths, \ie, $\phi \leq m$ and $\psi \leq nm$.
For the case when $\gamma = 0$, a subpath is a single vertex ($\phi \leq n$) and a pair of compatible subpaths is an edge ($\psi \leq m$).
Therefore, Algorithm~\ref{algo:valMESP} runs in $\calO \big( n^2m \big)$ time if $\gamma = 1$, and in $\calO(nm)$ time if $\gamma = 0$.
\end{proof}

Note that Algorithm~\ref{algo:valMESP} computes a shortest path starting in a given vertex~$s$.
Thus, a shortest path with minimum eccentricity among all shortest paths in $G$ can be determined by running Algorithm~\ref{algo:valMESP} for all start vertices~$s$, resulting in the following:

\begin{theorem}
For a given graph~\( G \) with \( \pg(G) = \gamma \), a  minimum eccentricity shortest path can be found in \( \calO \big( n^{\gamma + 4}  \big) \) time if \( \gamma \geq 2 \), in \( \calO \big( n^3 m \big) \) time if \( \gamma = 1 \), and in \( \calO \big( n^2 m \big) \) time if \( \gamma = 0 \).
\end{theorem}

\subsubsection{Projection Gap for Some Special Graph Classes.}

Above, we show that a minimum eccentricity shortest path can be found in polynomial time if the projection gap is bounded by a constant.
In what follows, we determine the projection gap for chordal graphs, dually chordal graphs, graphs with bounded tree-length or tree-breadth, and $\delta$-hyperbolic graphs.

\paragraph{Chordal Graphs.}

\begin{lemma}
If \( G \) is a chordal graph, then \( \pg(G) = 0 \).
\end{lemma}

\begin{proof}
Assume that $\pg(G) \geq 1$.
Then, there is a shortest path~$P = \{ u, \ldots, w \}$ and a vertex~$x$ with $\Pr(x, P) = \{ u, w \}$ and $d(u, w) > 1$.
By Lemma~\ref{lem:ChordalMConvex} (page~\ref{lem:ChordalMConvex}), $u$ and~$w$ are adjacent.
This contradicts with $d(u, w) > 1$.
\end{proof}

\begin{corollary}
    \label{cor:mespChordal}
For chordal graphs, a minimum eccentricity shortest path can be found in \( \calO \big( n^2 m \big) \) time.
\end{corollary}


\paragraph{Dually Chordal Graphs.}

\begin{lemma}
If \( G \) is a dually chordal graph, then \( \pg(G) \leq 1 \).
\end{lemma}

\begin{proof}
Assume there is a shortest path~$P = \{ u, v_1, \ldots, v_i , w \}$ and a vertex~$x$ with $\Pr(x, P) \supseteq \{ u, w \}$.
To show that $\pg(G) \leq 1$, we show that $d(u, w) = i + 1 > 2$ implies there is a vertex~$v_k \in \Pr(x, P)$ with $1 \leq k \leq i$.

Consider a family of disks $\calD = \big \{ N[u], N[v_{1}], \ldots, N[v_{i}], N[w], N^r[x]  \big \}$ where $r = d(x, P) - 1$.
Let $H$ be the intersection graph of $\calD$, $a$ be the vertex in~$H$ representing $N[u]$, $b_k$ representing $N[v_{k}]$ (for $1 \leq k \leq i$), $c$ representing $N[w]$, and $z$ representing~$N^r[x]$.
Because the intersection graph of disks of a dually chordal graph is chordal~\cite{BraDraCheVol1998}, $H$ is chordal, too.
$H$ contains the edges $za$ and~$zc$, $ab_1$, $cb_i$, and $b_kb_{k+1}$ for all $1 \leq k < i$.
Note that, if $d(u, w) > 2$, $a$ and~$c$ are not adjacent in~$H$.
However, the path $\{ a, b_1, \ldots, b_k, c \}$ connects $a$ and~$c$.
Therefore, because $H$ is chordal and by Lemma~\ref{lem:ChordalMConvex} (page~\pageref{lem:ChordalMConvex}), there is a $k$ with $1 \leq k \leq i$ such that $z$ is adjacent to~$b_k$ in~$H$.
Thus, $d(x, v_k) \leq r + 1$, \ie, $v_k \in \Pr(x, P)$.
\end{proof}

\begin{corollary}
For dually chordal graphs, a minimum eccentricity shortest path can be found in \( \calO \big( n^3 m \big) \) time.
\end{corollary}

\paragraph{Graphs with bounded Tree-Length or Tree-Breadth.}

\begin{lemma}
If \( G \) has tree-length~\( \lambda \) or tree-breadth~\( \rho \), a factor~\( \gamma \geq \pg(G) \) can be computed in \( \calO \big( n^3 \big) \) time such that \( \gamma \leq 3 \lambda - 1 \) or \( \gamma \leq 6 \rho - 1 \), respectively.
\end{lemma}

\begin{proof}
To compute $\gamma$, first determine the pairwise distances of all vertices.
Then, compute a layering partition for each vertex~$x$.
Let $\gamma + 1$ be the maximum diameter of all clusters of all layering partitions.

As shown in Lemma~\ref{lem:LayPartDist} (page~\pageref{lem:LayPartDist}), the diameter of each cluster is at most~$3 \lambda$ if $G$ has tree-length~$\lambda$ and at most~$6 \rho$ if $G$ has tree-breadth~$\rho$.
Therefore, for each shortest path~$P$, $\diam(\Pr(x, P)) \leq 3 \lambda$ and $\diam(\Pr(x, P)) \leq 6 \rho$, respectively.
Thus, $\pg(G) \leq \gamma \leq 3 \lambda - 1$ and $\pg(G) \leq \gamma \leq 6 \rho - 1$.

Computing the pairwise distances of all vertices can be done in $\calO(nm)$ time.
A layering partition can be computed in linear time (Lemma~\ref{lem:LayPartLinearTime}, page~\pageref{lem:LayPartLinearTime}).
For a given layering partition, the diameter of each cluster can be computed in $\calO \big( n^2 \big)$ time if the pairwise distances of all vertices are known.
Thus, $\gamma$ can be computed in $\calO \big( n^3 \big)$ time.
\end{proof}

Note that it is not necessary to know the tree-length or tree-breath of~$G$ to compute~$\gamma$.
Thus, by computing~$\gamma$ and then running Algorithm~\ref{algo:valMESP} for each vertex in $G$, we get:

\begin{corollary}
For graphs with tree-length~\( \lambda \) or tree-breadth~\( \rho \), a minimum eccentricity shortest path can be found in \( \calO \big( n^{3\lambda+3} \big) \) time or \( \calO \big( n^{6\rho+3} \big) \) time, respectively.
\end{corollary}


\paragraph{$\delta$-Hyperbolic Graphs.}

\begin{lemma}
If \( G \) is \( \delta \)-hyperbolic, then \( \pg(G) \leq 4 \delta \).
\end{lemma}

\begin{proof}
Consider two vertices $u$ and~$w$ such that $u,w \in \Pr(x, P)$ for some vertex~$x$ and shortest path~$P$.
Let $v \in P$ be a vertex such that $d(u, v) - d(v, w) \leq 1$ and $d(u, v) \geq d(v, w)$, \ie, $v$ is a middle vertex on the subpath from $u$ to~$w$.

Assume that $d(u, w) > 4\delta + 1$.
Thus, $d(u, v) \geq d(v, w) \geq 2\delta + 1$ and $d(u, w) > d(u, v) + 2\delta$.
Therefore, by Lemma~\ref{lem:deltaHyper} (page~\pageref{lem:deltaHyper}), $d(v, x) < \max \{ d(x, u), d(x, w) \}$.
This contradicts that $u,w \in \Pr(x, P)$.
Hence, the diameter of a projection is at most $4 \delta + 1$ and, therefore, $\pg(G) \leq 4 \delta$.
\end{proof}

\begin{corollary}
For \( \delta \)-hyperbolic graphs, a minimum eccentricity shortest path can be found in \( \calO \big( n^{4\delta+4} \big) \) time.
\end{corollary}


\section{Approximation Algorithms}
    \label{sec:mespApprox}

In this section, we present different approximation algorithms.
The algorithms differ in their approximation factor and runtime.
First, we show algorithms for general graphs.
Then, we present approaches to compute an approximation for graphs with bounded tree-length and graphs with bounded hyperbolicity.
Additionally, we present an approach to compute an approximation based on the layering partition of a graph.

\subsection{General Graphs}
    \label{sec:mespGeneralAppr}

In~\cite{DraganLeiter2015}, we show that a spread path gives an $8$-approximation for MESP.
Recently, \name{Birmelé} et al.\,\cite{BirmDeMoPlan2016} were able to improve this result.

\begin{theorem}
    [\name{Birmelé} et al.\,\cite{BirmDeMoPlan2016}]
    \label{theo:mesp5Approx}
Let \( G \) be a graph having a shortest path of eccentricity~\( k \).
Any spread path in~\( G \) has eccentricity at most~\( 5k \).
This bound is tight.
\end{theorem}

From Theorem~\ref{theo:mesp5Approx}, it follows that a $5$-approximation for MESP can be computed in linear time by simply performing two BFS calls.
\name{Birmelé} et al.\,\cite{BirmDeMoPlan2016} additionally show that Algorithm~\ref{algo:mesp3Approx} below computes a $3$-approximation in linear time.
However, the runtime of this approach has a much higher constant factor.

\begin{algorithm}
    [htb]
    \caption
    {%
        \cite{BirmDeMoPlan2016}
        Computes a $3$-approximation for MESP.
    }
    \label{algo:mesp3Approx}
%
\KwIn
{%
    A graph~$G = (V, E)$.
}

\KwOut
{%
    A shortest path with eccentricity at most~$2k$, where $k$ is the minimum eccentricity of all paths in~$G$.
}

Compute a spread path~$P$ in~$G$ from $x$ to~$y$ and determine its eccentricity.

Initialise an empty queue~$Q$ of triples~$(u, v, s)$ where $u$ and~$v$ are vertices of~$G$ and $s$ is an integer.

Insert $(x, y, 0)$ into~$Q$.

\While
{%
    $Q$ is non-empty.
}
{%
    Remove a triple~$(u, v, s)$ from~$Q$.

    Compute a shortest path~$P'$ from $u$ to~$v$ and determine a vertex~$w$ with maximal distance to~$P'$.

    \If
    {%
        $d(P', w) < \ecc(P)$
    }
    {%
        Set $P := P'$.
    }

    \If
    {%
        $s < 8$
    }
    {%
        Insert $(u, w, s + 1)$ and $(v, w, s + 1)$ into~$Q$.
    }
}

Output $P$.
\end{algorithm}

\begin{theorem}
    [\name{Birmelé} et al.\,\cite{BirmDeMoPlan2016}]
Algorithm~\ref{algo:mesp3Approx} calculates a \( 3 \)-approximation for the MESP problem in linear time.
\end{theorem}

Next, we present a 2-approximation algorithm.
It is based on the following two lemmas.

\begin{lemma}
    \label{lem:LayerRadius}
In a graph~\( G \), let \( P \) be a shortest path from \( s \) to~\( t \) of eccentricity at most~\( k \).
For each layer~\( L_i^{(s)} \), there is a vertex \( p_i \in P \) such that the distance from \( p_i \) to each vertex \( v \in L_i^{(s)} \) is at most~\( 2k \).
Additionally, \( p_i \in L_i^{(s)} \) if \( i \leq d(s,t) \), and \( p_i = t \) if \( i \geq d(s,t) \).
\end{lemma}

\begin{proof}
For each vertex~$v$, let $p(v) \in P$ be a vertex with $d(p(v), v) \leq k$.

For each $i \leq d(s,t)$, let $p_i \in P \cap L_i^{(s)}$ be the vertex in~$P$ with distance~$i$ to~$s$.
For an arbitrary vertex~$v \in L_i^{(s)}$, let $j = d(s,p(v))$.
Because $\ecc(P) \leq k$ and $P$ is a shortest path, $|i-j| \leq k$.
Thus, $d(p_i,v) \leq d(p_i,p(v)) + d(p(v),v) \leq 2k$.

Let $L' = \{ \, v \mid d(s,v) \geq d(s,t) \, \}$.
Because $P$ has eccentricity at most~$k$, $d(p,t) \leq k$ for all $p \in \{ \, p(v) \mid v \in L' \, \}$.
Therefore, $d(t,v) \leq 2k$ for all $v \in L'$.
\end{proof}

\begin{lemma}
    \label{lem:3kDomPath}
If \( G \) has a shortest path of eccentricity at most~\( k \) from \( s \) to~\( t \), then every path~\( Q \) with \( s \in Q \) and \( d(s,t) \leq \max_{v \in Q}d(s,v) \) has eccentricity at most~\( 3k \).
\end{lemma}

\begin{proof}
Let $P$ be a shortest path from $s$ to~$t$ with $\ecc(P) \leq k$ and $Q$ an arbitrary path with $s \in Q$ and $d(s,t) \leq \max_{v \in Q}d(s,v)$.
Without loss of generality, we can assume that $Q$ starts at~$s$.
Also, let $u$ be an arbitrary vertex.
Since $\ecc(P) \leq k$, there is a vertex $p \in P$ with $d(u,p) \leq k$.
Because $d(s,t) \leq \max_{v \in Q}d(s,v)$, there is a vertex~$q \in Q$ with $d(s,p) = d(s,q)$.
By Lemma~\ref{lem:LayerRadius}, the distance between $p$ and $q$ is at most $2k$.
Thus, the distance from $q$ to $u$ is at most~$3k$.
\end{proof}

\begin{corollary}
    For a given graph~\( G \) and two vertices \( s \) and~\( t \), each shortest \( (s,t) \)-path is a \( 3 \)-approximation for the \( (s,t) \)-MESP problem.
\end{corollary}

Note that the bounds given in Lemma~\ref{lem:LayerRadius} and Lemma~\ref{lem:3kDomPath} are tight.
Figure~\ref{fig:approxExample} gives an example.

\begin{figure}
    [htb]
    \centering
    \tikzsetnextfilename{fig_mesp_approxExample}
\begin{tikzpicture}
\foreach \x in {-3,-2,-1,0,1,2,3}
{
    \node[sN] (n\x0) at (\x,0) {};
}

\foreach \x in {-1,0,1}
{
    \node[sN] (n\x-1) at (\x,-1) {};
}

\node [sN] at (0,1){};

\node [llbl] at (n-30.west) {$s$};
\node [rlbl] at (n30.east) {$t$};

\node [blbl] at (n-1-1.south) {$u$};
\node [blbl] at (n0-1.south) {$v$};
\node [blbl] at (n1-1.south) {$w$};

\node [lbl,anchor=-135] at (n00.45) {$x$};

\begin{pgfonlayer}{background}

\draw (-3,0) -- (3,0);
\draw (-1,-1) -- (1,-1);
\draw (-2,0) -- (-1,-1);
\draw (2,0) -- (1,-1);
\draw (-1,0) -- (-0,-1);
\draw (1,0) -- (0,-1);
\draw (0,0) -- (0,1);

\end{pgfonlayer}

\end{tikzpicture}

    \caption
    [%
        Example for Lemma~\ref{lem:LayerRadius} and Lemma~\ref{lem:3kDomPath}.
    ]
    {
        Example for Lemma~\ref{lem:LayerRadius} and Lemma~\ref{lem:3kDomPath}.
        The shortest path from $s$ to~$t$ which contains $x$ has eccentricity~$1$ and the distance from $x$ to~$v$ is~$2$.
        The shortest path from $s$ to~$t$ which contains $u$ and~$w$ has eccentricity~$3$.
    }
    \label{fig:approxExample}
\end{figure}

For Algorithm~\ref{algo:2ApproxDSP} below, we say the \emph{layer-wise eccentricity} of a shortest $(s,t)$-path~$Q$ is~$\phi$ if, for each layer~$L_i^{(s)}$, $\max \left \{ \, d(q_i, u) \Bigm| u \in L_i^{(s)} \, \right \} \leq \phi$ where $q_i \in Q \cap L_i^{(s)}$ if $i \leq d(s, t)$ and $q_i = t$ if $i > d(s, t)$.
By Lemma~\ref{lem:LayerRadius}, a shortest path with eccentricity~$k$ has a layer-wise eccentricity of~$2k$.
Therefore, determining a shortest path with minimum layer-wise eccentricity gives a 2-approximation for the MESP problem.
To find such a path, Algorithm~\ref{algo:2ApproxDSP} computes, for each vertex~$s$, the maximal distance of a vertex~$v$ to all other vertices~$u$ in the same layer~$L_i^{(s)}$ and uses a modified BFS to find a shortest path with minimal layer-wise eccentricity starting at~$s$.

\begin{algorithm}
    [htb]
    \caption
    {
        A 2-approximation for the MESP problem.
    }
    \label{algo:2ApproxDSP}

\KwIn
{%
    A graph~$G = (V, E)$.
}

\KwOut
{%
    A shortest path with eccentricity at most~$2k$, where $k$ is the minimum eccentricity of all paths in~$G$.
}

Calculate the distances $d(u,v)$ for all vertex pairs $u$ and~$v$, including $L_i^{(u)}=\{ \, v \in V \mid d(u,v) = i \, \}$ with $0 \leq i \leq \ecc(u)$ for each $u$.
\label{line:Distances}

\ForEach
{%
    $s \in V$
    \label{line:forLoop}
}
{%
    Set $\phi(s) := 0$.

    \For{$i := 1$ \KwTo $\ecc(s)$}
    {
        \label{line:LayerLoop}

        \ForEach{$v \in L_i^{(s)}$}
        {
            \label{line:VertexInLayerLoop}

            Set $\phi(v) := \max_{u \in L_i^{(s)}}d(u,v)$.
            \label{line:LayerDistance}

            Let $N^-(v) = L_{i-1}^{(s)}\cap N(v)$ denote the neighbours of~$v$ in the previous layer.
            Set $\phi(v) := \max \Big \{ \! \min_{u \in N^-(v)}\phi(u), \phi(v) \Big \}$.
            \label{line:prevLayerDistance}

            Set $\phi^+(v) := \max \big \{ \, d(u,v) \mid d(s,u) \geq i \, \big \}$.
            \label{line:laterLayerDistance}
        }
    }

    Calculate a BFS-tree~$T(s)$ starting from $s$.
    If multiple vertices~$u$ are possible as parent for a vertex~$v$, select one with the smallest $\phi(u)$.
    \label{line:BFSTree}


    Let $t$ be the vertex for which $\phi'(t) := \max \big \{ \phi(t), \phi^+(t) \big \}$ is minimal.
    Set $k(s) := \phi'(t)$
    \label{line:SelectMinT}

}

Among all computed pairs $s$ and~$t$, select a pair (and corresponding path in~$T(s)$) for which $k(s)$ is minimal.
\label{line:SelectMinK}

\end{algorithm}

\begin{theorem}
Algorithm~\ref{algo:2ApproxDSP} calculates a 2-approximation for the MESP problem in \( \calO \big( n^3 \big) \) time.
\end{theorem}

\begin{proof}
    [Correctness]
Assume a given graph~$G$ has a shortest path~$P$ from $s$ to~$t$ with $\ecc(P) = k$ and $s$ is the vertex selected by the loop starting in line~\ref{line:forLoop}.
Let $Q$ be a shortest path from $s$ to $v$.

We show that lines~\ref{line:LayerLoop} to~\ref{line:laterLayerDistance} calculate, for each $v$, the minimal $\phi(v)$ such that there is a shortest path~$Q$ from $s$ to~$v$ with a layer-wise eccentricity~$\phi(v)$.

By induction, assume this is true for all vertices~$u \in L_j^{(s)}$ with $j \leq i -1$.
Now let $v$ be an arbitrary vertex in~$L_i^{(s)}$.
Line~\ref{line:LayerDistance} calculates the maximal distance~$\phi(v)$ from $v$ to all other vertices in~$L_i^{(s)}$.
Since $v$ is the only vertex in~$Q \cap L_i^{(s)}$ for every shortest path $Q$ from $s$ to~$v$, the layer-wise eccentricity of each~$Q$ is at least~$\phi(v)$.
Let $u$ be a neighbour of~$v$ in the previous layer.
By induction hypothesis, $\phi(u)$ is optimal.
Therefore, $\phi(v) := \max \Big \{ \min_{u \in N^-[v]}\phi(u), \phi(v) \Big \}$ (line~\ref{line:prevLayerDistance}) is optimal for~$v$.

Since line~\ref{line:BFSTree} selects the vertex~$u$ with the smallest $\phi(u)$ as parent for~$v$, each path~$Q$ from $s$ to~$v$ in~$T(s)$ has an optimal layer-wise eccentricity of~$\phi(v)$.
Line~\ref{line:laterLayerDistance} calculates the maximal distance from $v$ to all vertices in $\big \{ \, u \mid d(s,u) \geq d(s,v) \, \big \}$.
Thus, $\ecc(Q) \leq \phi'(v)$ and line~\ref{line:SelectMinT} and line~\ref{line:SelectMinK} select a shortest path which has an eccentricity at most~$\phi'(v)$.

By Lemma~\ref{lem:LayerRadius}, we know that $P$ has a layer-wise eccentricity of at most~$2k$.
Thus, the path~$Q$ from $s$ to~$t$ in~$T(s)$ has a layer-wise eccentricity of at most~$2k$.
Additionally, Lemma~\ref{lem:LayerRadius} says that $t$ has distance at most~$2k$ to all vertices in $\{ \, v \mid d(s,v) \geq d(s,t) \, \}$.
Therefore, $\ecc(Q) \leq 2k$.
Thus, the path selected in line~\ref{line:SelectMinK} is a shortest path with eccentricity at most~$2k$.
\end{proof}

\begin{proof}
    [Complexity]
Line~\ref{line:Distances} runs in $\calO(nm)$ time.
If the distances are stored in an array, they can be later accessed in constant time.
Therefore, line~\ref{line:LayerDistance} and line~\ref{line:laterLayerDistance} run in $\calO(n)$ time for a given $s$ and~$v$ or in $\calO \big( n^3 \big)$ time overall.
For a given $s$, line~\ref{line:prevLayerDistance} runs in $\calO(m)$ time and, therefore, has an overall runtime of $\calO(nm)$.
Line~\ref{line:BFSTree} has an overall runtime of $\calO(nm)$, line~\ref{line:SelectMinK} takes $\calO \big( n^2 \big)$ time, and line~\ref{line:SelectMinT} runs in $\calO(n)$ time.
Adding all together, the total runtime is $\calO \big( n^3 \big)$.
\end{proof}

For the case that a start vertex~$s$ for a shortest path is given, Algorithm~\ref{algo:2ApproxDSP} can be simplified by having only one iteration of the loop starting in line~\ref{line:forLoop}.
Then, the runtime is $\calO(nm)$.

\begin{corollary}
A 2-approximation for the \( (s,t) \)-MESP problem can be computed in \( \calO(nm) \) time.
\end{corollary}


\subsection{Graphs with Bounded Tree-Length and Bounded Hyperbolicity}

In Section~\ref{sec:mespProjGapApproach}, we show how to find a shortest path with minimum eccentricity~$k$ for several graph classes.
For graphs with tree-length~$\lambda$, this can require up-to $\calO \big( n^{3\lambda + 3} \big)$ time.
In this section, we show that, for graphs with tree-length~$\lambda$, we can find a shortest path with eccentricity at most $k + 2.5\lambda$ in at most $\calO(\lambda m)$ time and, for graphs with hyperbolicity~$\delta$, we can find a shortest path with eccentricity at most $k + \calO(\delta \log n)$ in at most $\calO(\delta m)$~time.

\begin{lemma}
    \label{lem:mutualDistHyper}
Let \( G \) be a graph with hyperbolicity~\( \delta \).
Two vertices \( x \) and~\( y \) in \( G \) with \( \ecc(x) = \ecc(y) = d(x, y) \) can be found in \( \calO(\delta m) \) time.
\end{lemma}

\begin{proof}
Let $u$ and~$v$ be two vertices in $G$ such that $d(u, v) = \diam(G)$.
For an arbitrary vertex~$x_0$ and for $i \geq 0$, let $y_i = x_{i+1}$ be vertices in~$G$ such that $d(x_i, y_i) = \ecc(x_i)$ and $d(x_i, y_i) < d(x_{i+1}, y_{i+1})$.
To prove Lemma~\ref{lem:mutualDistHyper}, we show that there is no vertex~$y_{2\delta + 1}$.

Because $d(x_0, y_0) = \ecc(x_0)$, $d(x_0, y_0) \geq \max \big \{ d(x_0, u), d(x_0 ,v) \big \}$.
Therefore, by Lemma~\ref{lem:deltaHyper} (page~\pageref{lem:deltaHyper}), $d(u, v) \leq \max \big \{ d(u, y_0), d(v, y_0) \big \} + 2\delta$ and, thus, $\diam(G) \leq \ecc(x_1) + 2\delta$.
Since $d(x_i, y_i) < d(x_{i+1}, y_{i+1})$, there is no vertex~$y_j$ with $j \geq 2\delta + 1$, otherwise $d(x_j, y_j) > \diam(G)$.
Therefore, a vertex pair~$x,y$ with $\ecc(x) = \ecc(y) = d(x, y)$ can be found in $\calO(\delta m)$~time as follows:
Pick an arbitrary vertex~$x_0$ and find a vertex~$x_1$ with $d(x_0, x_1) = \ecc(x_0)$ using a BFS.
Next, find a vertex~$x_2$ such that $d(x_1, x_2) = \ecc(x_1)$.
Repeat this (at most $2\delta$ times) until $d(x_i, x_{i+1}) = \ecc(x_i) = \ecc(x_{i+1})$.
\end{proof}

Recall that, if a graph has tree-length~$\lambda$, its hyperbolicity is at most $\lambda$~(Theorem~\ref{theo:hyperbolicTreeLength}, page~\pageref{theo:hyperbolicTreeLength}).
Thus, it follows:

\begin{corollary}
    \label{cor:mutualDistTreeLen}
Let \( G \) be a graph with tree-length~\( \lambda \).
Two vertices \( x \) and~\( y \) in \( G \) with \( \ecc(x) = \ecc(y) = d(x, y) \) can be found in \( \calO(\lambda m) \) time.
\end{corollary}

The next lemma shows that, in a graph with bounded tree-length, a shortest path between two mutually furthest vertices gives an approximation for MESP.

\begin{lemma}
    \label{lem:ApproxTreeLen}
Let \( G \) be a graph with tree-length~\( \lambda \) having a shortest path with eccentricity~\( k \).
Also, let \( x \) and~\( y \) be two mutually furthest vertices, \ie, \( \ecc(x) = \ecc(y) = d(x, y) \).
Then, each shortest path from \( x \) to~\( y \) has eccentricity less than or equal to \( k + 2.5 \lambda \).
\end{lemma}

\begin{proof}
Let $P$ be a shortest path from $s$ to~$t$ with eccentricity~$k$ and $Q$ be a shortest path from $x$ to~$y$.
Consider a tree-decomposition~$\calT$ for $G$ with length~$\lambda$.
We distinguish between two cases:
(1)~There is a bag in~$\calT$ containing a vertex of $P$ and a vertex of $Q$ and (2)~there is no such bag in~$\calT$.

\paragraph{Case 1: There is a bag in~$\calT$ containing a vertex of $P$ and a vertex of~$Q$.}
We define bags $B_x$ and $B_y$ as follows:
Both contain a vertex of~$P$ and a vertex of~$Q$, $B_x$ is a bag closest to a bag containing~$x$, $B_y$ is a bag closest to a bag containing~$y$, and the distance between $B_x$ and~$B_y$ in~$\calT$ is maximal.
Let $\{ B_0, B_1, \ldots, B_l \}$ be a subpath of the shortest path from $B_x$ to~$B_y$ in~$\calT$ such that $B_0$ is a bag closest to a bag containing~$s$, $B_l$ is a bag closest to a bag containing~$t$, $B_i$ is adjacent to~$B_{i+1}$ in $\calT$ ($0 \leq i < l$), and the distance~$l$ between $B_0$ and~$B_l$ is maximal.
Without loss of generality, let $d_{\calT}(B_x, B_0) \leq d_{\calT}(B_x, B_l)$.
Let $p_s$ be the vertex in $B_0 \cap P$ which is closest to~$s$ in~$G$ and let $p_t$ be the vertex in $B_l \cap P$ which is closest to~$t$ in~$G$.
Figure~\ref{fig:TreeDecoEg} gives an illustration.

\begin{figure}
    [htb]
    \centering
    \tikzsetnextfilename{fig_mesp_treeDeco}
\begin{tikzpicture}
    [scale=1.]
\def\bRad{0.5cm}
\def\eLen{1cm}

\coordinate (cBi) at (0,0);

\coordinate (cB0) at ($(cBi)-(1*\bRad,0)-(0.75*\eLen,0)$);
\coordinate (cBl) at ($(cBi)+(1*\bRad,0)+(0.75*\eLen,0)$);

\coordinate (cBx) at (cB0); % ($(cB0)-(2*\bRad,0)-(1.0*\eLen,0)$);
\coordinate (cBy) at ($(cBl)+(2*\bRad,0)+(1.0*\eLen,0)$);

\coordinate (cBxx) at ($(cBx)+(-135:2*\bRad)+(-135:1.0*\eLen)$);
\coordinate (cByy) at ($(cBy)+(-45:2*\bRad)+(-45:1.0*\eLen)$);

\coordinate (cBs) at ($(cB0)+(135:2*\bRad)+(135:1.0*\eLen)$);
\coordinate (cBt) at ($(cBl)+(45:2*\bRad)+(45:1.0*\eLen)$);

\begin{scope}
    [semithick]
\draw (cBxx) -- (cBx) -- (cBy) -- (cByy);
\draw (cBs) -- (cB0);
\draw (cBt) -- (cBl);

%\draw[fill=white] (cBi) circle (\bRad);
\draw[fill=white] (cB0) circle (1.25*\bRad);
\draw[fill=white] (cBl) circle (1.25*\bRad);
%\draw[fill=white] (cBx) circle (1.25*\bRad);
\draw[fill=white] (cBy) circle (1.25*\bRad);
\draw[fill=white] (cBs) circle (\bRad);
\draw[fill=white] (cBt) circle (\bRad);
\draw[fill=white] (cBxx) circle (\bRad);
\draw[fill=white] (cByy) circle (\bRad);
\end{scope}

\coordinate (cpt) at ($(cBl)+(55:0.8*\bRad)$);

\draw
    [
        thick,
        double=black,
        double distance=0.4pt,
        white,
    ]
    (cBt)
    .. controls ++(-135:0.75*\bRad) and ++(135:0.75*\bRad) ..
    (cpt)
    .. controls ++(-45:0.75*\bRad) and ++(90:0.75*\bRad) ..
    (cBy)
    .. controls ++(-90:0.75*\bRad) and ++(-10:2*\bRad) ..
%     ($(cBl)+(-90:0.4*\bRad)$)
%     .. controls ++(180:0.5*\bRad) and ++(-5:0.5*\bRad) ..
    ($(cB0)+(-90:0*\bRad)$)
    .. controls ++(170:2*\bRad) and ++(-90:0.75*\bRad) ..
    coordinate[pos=0.1] (cps)
    (cBs)
    ;

\draw
    [
        thick,
        double=black,
        double distance=0.4pt,
        white,
    ]
    (cByy)
    .. controls ++(180:1*\bRad) and ++(-5:2*\bRad) ..
    ($(cBy)+(-120:0.9*\bRad)$)
    .. controls ++(175:0.75*\bRad) and ++(0:0.75*\bRad) ..
    ($(cBl)+(-90:0.75*\bRad)$)
    .. controls ++(180:0.75*\bRad) and ++(5:0.75*\bRad) ..
    ($(cBx)+(-60:0.9*\bRad)$)
    .. controls ++(185:2*\bRad) and ++(0:1*\bRad) ..
    (cBxx)
    ;

%\fill ($(cBy)+(-120:0.9*\bRad)$) circle (0.75pt);
\node[sN] (npt) at (cpt) {};
\node[blbl] at (npt.south) {$p_t$};

\node[sN] (nps) at (cps) {};
\node[lbl,anchor=south west] at (nps.0) {$p_s$};

%\node at (cBi) {$B_i$};
\node[anchor=north,lbl] at ($(cB0)-(0,1.25*\bRad)+(10pt,0)$) {$B_x = B_0$};
\node[anchor=north] at ($(cBl)-(0,1.25*\bRad)$) {$B_l$};
%\node[anchor=north] at ($(cBx)-(0,1.25*\bRad)$) {$B_x$};
\node[anchor=north] at ($(cBy)-(0,1.25*\bRad)$) {$B_y$};
\node[fill=white,inner sep=2pt,lbl] at (cBs) {$s$};
\node[fill=white,inner sep=2pt,lbl] at (cBt) {$t$};
\node[fill=white,inner sep=2pt,lbl] at (cBxx) {$x$};
\node[fill=white,inner sep=2pt,lbl] at (cByy) {$y$};

\end{tikzpicture}
%
    \caption
    {
        Example for a possible tree-decomposition.
    }
    \label{fig:TreeDecoEg}
\end{figure}

\begin{claim}
    \label{cla:PQdis15Lambda}
For each vertex~\( p \in P \) with \( d_G(s, p_s) \leq d_G(s, p) \leq d_G(s, p_t) \), \( d_G(p, Q) \leq 1.5 \lambda \).
\end{claim}

\begin{proof}
    [Claim]
There is a vertex set $\{ p_s = p_0, p_1, \ldots, p_l, p_{l+1} = p_t \} \subseteq P$, where $p_i \in B_{i-1} \cap B_i$ for all positive $i \leq l$.
Because $p_i, p_{i+1} \in B_i$ for $0 \leq i \leq l$, $d_G(p_i, p_{i+1}) \leq \lambda$.
Thus, because $P$ is a shortest path, for all $p' \in P$ with $d_G(s, p_s) \leq d_G(s, p') \leq d_G(s, p_t)$ there is a vertex~$p_i$ with $0 \leq i \leq l+1$ such that $d_G(p_i, p') \leq 0.5 \lambda$.
By definition of $\calT$, each bag~$B_i$ ($0 \leq i \leq l$) contains a vertex~$q \in Q$, \ie, $d_G(p_i, Q) \leq \lambda$ ($0 \leq i \leq l+1$).
Therefore, for all $p' \in P$ with $d_G(s, p_s) \leq d_G(s, p') \leq d_G(s, p_t)$ there is a vertex~$p_i$ with $0 \leq i \leq l+1$ such that $d_G(p', Q) \leq d_G(p_i, p') + d_G(p_i, Q) \leq 1.5 \lambda$.
\end{proof}

Consider an arbitrary vertex~$v$ in~$G$.
Let $v'$ be a vertex in~$P$ closest to~$v$ and let $P_v$ be a shortest path from $v$ to~$v'$.
If $v'$ is between $p_s$ and $p_t$, \ie, $d_G(s, p_s) \leq d_G(s, v') \leq d_G(s, p_t)$, then, by Claim~\ref{cla:PQdis15Lambda}, $d_G(v, Q) \leq d_G(v, v') + d_G(v', Q) \leq k + 1.5 \lambda$.
If $P_v$ intersects a bag containing a vertex~$q \in Q$, $d_G(v, Q) \leq k + \lambda$.

Next, consider the case when $P_v$ does not intersect a bag containing a vertex of $Q$ and (without loss of generality) $d_G(s, v') > d_G(s, p_t)$.
In this case, each path from $x$ to~$v$ intersects~$B_l$.

\begin{claim}
There is a vertex~\( u \in B_l \) such that \( d_G(u, y) \leq k + 0.5 \lambda \).
\end{claim}

\begin{proof}
    [Claim]
Let $y'$ be a vertex in~$P$ that is closest to~$y$ and let $P_y$ be a shortest path from $y$ to~$y'$.
If $P_y$ intersects~$B_l$, there is a vertex $u \in P_y \cap B_l$ with $d_G(y, u) \leq k$.

If $P_y$ does not intersect~$B_l$, then there is a subpath of~$P$ starting at~$p_t$, containing~$y'$, and ending in a vertex~$p_l \in B_l$.
Because $d_G(p_t, p_l) \leq \lambda$, $d_G(y', \{ p_t, p_l \}) \leq 0.5 \lambda$.
Therefore, $d_G(y, \{ p_t, p_l \}) \leq d_G(y, y') + d_G(y', \{ p_t, p_l \}) \leq k + 0.5 \lambda$.
\end{proof}

Let $u$, $v'$, and~$z$ be vertices in~$B_l$ such that $d_G(u, y) \leq k + 0.5 \lambda$, $v'$ is on a shortest path from $x$ to~$v$, and $z \in Q$.
Because $d_G(x, y) = \ecc(x)$, $d_G(x, v') + d_G(v', v) \leq d_G(x, y)$.
Also, by the triangle inequality, $d_G(x, y) \leq d_G(x, v') + d_G(v', y)$ and $d_G(v', y) \leq d_G(v', u) + d_G(u, y)$.
Because $\{ u, v', z \} \subseteq B_l$ and $d_G(u, y) \leq k + 0.5 \lambda$, $d_G(v', v) \leq k + 1.5 \lambda$ and therefore $d_G(z, v) \leq k + 2.5 \lambda$.

Thus, if there is a bag in~$\calT$ containing a vertex of~$P$ and a vertex of~$Q$, then $d_G(v, Q) \leq k + 2.5 \lambda$ for all vertices~$v$ in~$G$.

\paragraph{Case 2: There is no bag in~$\calT$ containing vertices of $P$ and~$Q$.}
Because there is no such bag, $\calT$ contains a bag~$B$ such that each path from $x$ and~$y$ to~$P$ intersects~$B$ and there is a vertex~$z \in B \cap Q$.

Consider an arbitrary vertex~$v$.
If there is a shortest path~$P_v$ from $v$ to~$P$ which intersects~$B$, then $d_G(z, v) \leq k + \lambda$.
If there is no such path, each path from $x$ to~$v$ intersects~$B$.
Let $v' \in B$ be a vertex on a shortest path from $x$ to~$v$ and let $u \in B$ be a vertex on a shortest path from $y$ to~$P$.
Note that $d_G(u, y) \leq k$.

Because $d_G(x, y) = \ecc(x)$, $d_G(x, v') + d_G(v', v) \leq d_G(x, y)$.
Also, by the triangle inequality, $d_G(x, y) \leq d_G(x, v') + d_G(v', y)$ and $d_G(v', y) \leq d_G(v', u) + d_G(u, y)$.
Because $\{ u, v', z \} \subseteq B$ and $d_G(u, y) < k$, $d_G(v', v) < k + \lambda$ and therefore $d_G(z, v) < k + 2 \lambda$.

Thus, if there is no bag in~$\calT$ containing vertices of~$P$ and~$Q$, $d_G(v, Q) < k + 2 \lambda$ for all vertices~$v$ in~$G$.
\end{proof}

Recall that a $\delta$-hyperbolic graph has tree-length at most~$\calO(\delta \log n)$ (Theorem~\ref{theo:hyperbolicTreeLength}, page~\pageref{theo:hyperbolicTreeLength}).

\begin{corollary}
    \label{cor:ApproxHyperbolic}
Let \( G \) be a graph with hyperbolicity~\( \delta \) having a shortest path with eccentricity~\( k \).
Also, let \( x \) and~\( y \) be two mutually furthest vertices, \ie, \( \ecc(x) = \ecc(y) = d(x, y) \).
Each shortest path from \( x \) to~\( y \) has eccentricity less than or equal to \( k + \calO(\delta \log n) \).
\end{corollary}

Lemma~\ref{lem:mutualDistHyper}, Lemma~\ref{lem:ApproxTreeLen}, Corollary~\ref{cor:mutualDistTreeLen}, and Corollary~\ref{cor:ApproxHyperbolic} imply our next result:

\begin{theorem}
Let \( G \) be a graph having a shortest path with eccentricity~\( k \).
If \( G \) has tree-length~\( \lambda \), a shortest path with eccentricity at most \( k + 2.5\lambda \) can be found in \( \calO(\lambda m) \)~time.
If \( G \) has hyperbolicity~\( \delta \), a shortest path with eccentricity at most \( k + \calO(\delta \log n) \) can be found in \( \calO(\delta m) \)~time.
\end{theorem}

Recall that a graph is chordal if and only if it has tree-length~$1$ (Theorem~\ref{theo:chordalLength}, page~\pageref{theo:chordalLength}).

\begin{corollary}
    \label{cor:plus2ApproxChordal}
If \( G \) is a chordal graph and has a shortest path with eccentricity~\( k \), a shortest path in~\( G \) with eccentricity at most \( k + 2 \) can be found in linear time.
\end{corollary}

Figure~\ref{fig:chordalEg} gives an example that, for chordal graphs, $k + 2$ is a tight upper bound for the eccentricity of the determined shortest path.

\begin{figure}
    [htb]
    \centering
    \tikzsetnextfilename{fig_mesp_chordalEg}
\begin{tikzpicture}
\foreach \x [evaluate=\x as \xx using {min(\x,2)}] in {0,1,2,3}
{
    \foreach \y in {0,...,\xx}
    {
        \node[sN] (n\x\y) at (\x,\y) {};
    }
}

\node[sN] (n40) at (4,0) {};
\node[sN] (n42) at (4,2) {};
\node[sN] (ns) at (-2,0) {};
\node[sN] (nt) at (-1,0) {};

\node[lbl,anchor=south west,inner sep=1.5pt] at (n42.north east) {$u$};
\node[lbl,anchor=south east,inner sep=.5pt] at (n22.north west) {$u'$};
\node[lbl,anchor=south west,inner sep=1.5pt] at (n31.north east) {$v$};
\node[lbl,anchor=south east,inner sep=.5pt] at (n21.north west) {$v'$};
\node[lbl,anchor=south west,inner sep=1.5pt] at (n40.north east) {$w$};
\node[lbl,anchor=south east,inner sep=1.5pt] at (ns.north west) {$s$};

\begin{pgfonlayer}{background}
    \draw (0,0) -- (2,2);
    \draw (1,0) -- (3,2);
    \draw (2,0) -- (3,1);

    \draw (0,0) -- (4,0);
    \draw (1,1) -- (3,1);
    \draw (2,2) -- (4,2);

    \draw (1,0) -- (1,1);
    \draw (2,0) -- (2,2);
    \draw (3,0) -- (3,1);

    \draw (-2,0) -- (0,0);
\end{pgfonlayer}

\end{tikzpicture}
%
    \caption
    [%
        Example for Corollary~\ref{cor:plus2ApproxChordal}.
    ]
    {
        Example for Corollary~\ref{cor:plus2ApproxChordal}.
        For the shown chordal graph~$G$, a shortest path from $s$ to~$v$ passing $v'$ has eccentricity~$2$.
        This is the minimum for all shortest paths in~$G$.
        The diametral path from $s$ to~$u$ passing $u'$ has eccentricity~$4$ because of its distance to~$w$.
    }
    \label{fig:chordalEg}
\end{figure}

\subsection{Using Layering Partition}
In this subsection, we present an algorithm to compute an approximation for MESP using the layering partition of a graph.
For the remainder of this subsection, let $G$ be a graph, let $T$ be a layering partition of~$G$, and let $\Delta$ be the maximum cluster diameter of~$T$.
Additionally, let $k_G$ and~$k_T$ be the minimum eccentricities of any shortest path in $G$ and~$T$, respectively.

First, we show that
\[
    k_T - \frac{1}{2} \Delta \leq k_G \leq k_T + \Delta.
\]

\begin{lemma}
    \label{lem:mespLayPartLowBound}
\[
    k_T - \frac{1}{2} \Delta \leq k_G
\]
\end{lemma}

\begin{proof}
Let $P$ be a path in~$G$ with the start vertex~$s$, the end vertex~$t$, and with eccentricity~$k_G$.
Let $T_P$ be the subtree of~$T$ induced by the clusters of~$T$ which intersect~$P$.
Let $Q$ be the shortest path in~$T$ from $C_s$ to~$C_t$, where $C_s$ and~$C_t$ are the clusters containing $s$ and~$t$, respectively.
Clearly, by the properties of a layering partition, each cluster of~$Q$ is also a cluster of~$T_P$.

\begin{claim}
    \label{cla:PQdistLayPart}
For all \( p \in P \), \( d_G(p, Q) \leq \frac{1}{2} \Delta \).
\end{claim}

\begin{proof}[Claim]
Let $C_P$ be a cluster of~$T_P$ that is not part of~$Q$.
If no such cluster exists, then $T_P = Q$ and, thus, $d_G(p, Q) = 0$ for all \( p \in P \).
Additionally, let $C_Q$ be the cluster of~$Q$ which is closest to~$C_P$ in~$T$ and $p$ be a vertex of~$P$ in~$C_P$.

Due to the properties of a layering partition, there are two vertices~$p_s, p_t \in P \cap C_Q$ with $d_G(s, p_s) < d_G(s, p) < d_G(s, p_t)$.
Because $P$ is a shortest path and the diameter of a cluster is at most~$\Delta$, $d_G(p, \{ p_s, p_t \} ) \leq \frac{1}{2} \Delta$ and, therefore, $d_G(p, Q) \leq \frac{1}{2} \Delta$.
\end{proof}

Let $v$ be an arbitrary vertex of~$G$ and $v'$ be a vertex in~$P$ with minimal distance to~$v$.
By definition of~$P$, $d_G(v, v') \leq k_G$.
Then, by triangle inequality, $d_G(v, Q) \leq d_G(v, v') + d_G(v', Q)$, and, by Claim~\ref{cla:PQdistLayPart}, $d_G(v, Q) \leq k_G + \frac{1}{2} \Delta$.
Recall that, for any two vertices $u$ and~$v$ of~$G$, $d_T(u, v) \leq d_G(u, v)$ (Lemma~\ref{lem:LayPartVertDist}, page~\pageref{lem:LayPartVertDist}).
Thus, for any vertex~$v$ of~$G$, $d_T(v, Q) \leq k_G + \frac{1}{2} \Delta$.

Let $v$ be a vertex of~$G$ with maximal distance to~$Q$ in~$T$, \ie, $d_T(v, Q) = \ecc_T(Q)$.
Therefore, because $k_T \leq \ecc_T(Q)$, $k_T \leq d_T(v, Q) \leq k_G + \frac{1}{2} \Delta$.
\end{proof}

\begin{lemma}
    \label{lem:mespLayPartUpBound}
\[
    k_G \leq k_T + \Delta
\]
\end{lemma}

\begin{proof}
Let $Q$ be a path in~$T$ with minimum eccentricity, \ie, $\ecc_T(Q) = k_T$, such that $Q$ starts at the cluster~$C_s$ and ends at the cluster~$C_t$.
Additionally, let $s$ and~$t$ be two vertices of~$G$ with $s \in C_s$ and $t \in C_t$ and let $P$ be a shortest path from $s$ to~$t$ in~$G$.

Consider a vertex~$v$ of~$G$ with maximal distance to~$P$ in~$G$.
Hence, $k_G \leq \ecc_G(P) = d_G(v, P)$.
Because $d_G(u, v) \leq d_T(u, v) + \Delta$ for each vertex~$u$ of~$G$ (Lemma~\ref{lem:LayPartVertDist}, page~\pageref{lem:LayPartVertDist}), it follows that $d_G(v, P) \leq d_T(v, P) + \Delta$.
Note that, by the properties of a layering partition, $P$ intersects all clusters of~$Q$.
Thus, $d_T(v, P) \leq d_T(v, Q)$.
Additionally, by definition of~$Q$, $d_T(v, Q) \leq \ecc_T(Q) = k_T$.
From combining these observation, it follows that
\[
    k_G \leq \ecc_G(P) = d_G(v, P) \leq d_T(v, P) + \Delta \leq d_T(v, Q) + \Delta \leq \ecc_T(Q) + \Delta = k_T + \Delta.
    \qedhere
\]
\end{proof}

Based on Lemma~\ref{lem:mespLayPartLowBound} and Lemma~\ref{lem:mespLayPartUpBound}, we can compute an approximation for MESP in linear time.

\begin{theorem}
For a graph~\( G \) with minimum eccentricity~\( k \), a shortest path~\( P \) with eccentricity at most~\( k + \frac{3}{2} \Delta \) can be computed in linear time.
\end{theorem}

\begin{proof}
First, construct a layering partition~$T$ for~$G$, starting at an arbitrary vertex.
Next, find a path~$Q$ in~$T$ with minimum eccentricity.
Let $C_s$ and $C_t$ be the start and end clusters of~$Q$.
Pick two arbitrary vertices $s \in C_s$ and~$t \in C_t$.
Then, compute a shortest path~$P$ from $s$ to~$t$ in~$G$ and output it.
Note that each of these steps can be done in linear time, including the construction of~$T$ (Lemma~\ref{lem:LayPartLinearTime}, page~\pageref{lem:LayPartLinearTime}) as well as finding $Q$ in~$T$ (Corollary~\ref{cor:mespTrees}).

Note that the construction of~$P$ is the same as in the proof of Lemma~\ref{lem:mespLayPartUpBound}.
Therefore, as shown in the proof, $\ecc_G(P) \leq \ecc_T(Q) + \Delta$ and, by Lemma~\ref{lem:mespLayPartLowBound}, $\ecc_G(P) \leq k + \frac{3}{2} \Delta$.
\end{proof}



\section{Relation to Other Parameters}

Similar to path-length, one can see the minimum eccentricity~$k$ of all shortest paths of a graph~$G$ as a parameter describing the structure of~$G$.
In this section, we show how $k$ relates to the path-length of~$G$ and other parameters.

\begin{theorem}
    \label{theo:PathLengthMinEcc}
Let \( G \) be a graph with path-length~\( \lambda \) and path-breadth~\( \rho \) and let the minimum eccentricity of all shortest paths be~\( k \).
Then,
\[
    k \leq \lambda \leq 4k + 1 \quad \text{and} \, \quad \frac{1}{2} k \leq \rho \leq 2k + 1.
\]
\end{theorem}

\begin{proof}
As shown in Lemma~\ref{lem:plDomPairRelation} (page~\pageref{lem:plDomPairRelation}), it clearly follows from the definition of path-length that a graph with path-length~$\lambda$ has a shortest path with eccentricity at most~$\lambda$.
Therefore, $k \leq \lambda$ and, since $\lambda \leq 2 \rho$, $\frac{1}{2} k \leq \rho$.

Next, let $P$ be a shortest path in $G$ with the start vertex~$s$ and $\ecc(P) = k$.
As shown in Lemma~\ref{lem:LayerRadius}, the radius of $L_i^{(s)}$ is at most~$2k$.
Thus, its diameter is not larger than~$4k$.
One can now create a path decomposition $\calX = \{X_1, \ldots, X_n\}$ by creating a set $X_i$ including layer~$L_i^{(s)}$ and all vertices from $L_{i-1}^{(s)}$ which are adjacent to vertices in $L_i^{(s)}$, \ie, $X_i = L_i^{(s)} \cup \left \{ \, u \Bigm| uv \in E, u \in L_{i-1}^{(s)}, v \in L_i^{(s)} \right \}$.
It is easy to see that the radius of~$X_i$ is at most~$2k +1$, the diameter of~$X_i$ is at most~$4k +1$, and that $\calX$ is a valid path decomposition for $G$, \ie, $\rho \leq 2k + 1$ and $\lambda \leq 4k + 1$.
\end{proof}

Recently, \name{Völkel} et al.\,\cite{VolBapHabLop2016} defined a closely related problem called \emph{\( k \)-Laminarity} problem.
It asks if a given graph contains a diametral path with eccentricity at most~$k$.
If a graph contains such a path, it is called \emph{\( k \)-laminar}.
Similar, if every diametral path of a graph has eccentricity at most~$k$, it is called \emph{\( k \)-strongly laminar}.

\begin{theorem}
    [\name{Birmelé} et al.\,\cite{BirmDeMoPlan2016}]
Let \( G \) be a graph where the minimum eccentricity of all shortest paths is~\( k \).
Additionally, let $l$ and~$s$ be the minimal values such that $G$ is \( l \)-laminar and \( s \)-strongly laminar.
Then,
\[
    k \leq l \leq 4k - 2 \quad \text{and} \, \quad k \leq s \leq 4k.
\]
These bounds are tight.
\end{theorem}


\section{Solving $k$-Domination using a MESP}
    \label{sec:kDom}

In~\cite{Kratsch2000}, an $\calO \big( n^7 \big)$ time algorithm is presented which finds a minimum dominating set for graphs containing a shortest path with eccentricity~$1$.
Using a similar approach, we generalise this result to find a minimum $k$-dominating set for graphs containing a shortest path with eccentricity~$k$.

Recall that, for a graph~$G = (V, E)$, a vertex set~$D$ is a \emph{\( k \)-dominating set} if $N^k[D] = V$.
Additionally, $D$ is a \emph{minimum} $k$-dominating set if there is no $k$-dominating set~$D'$ for~$G$ with $|D'| < |D|$.

\begin{lemma}
    \label{lem:domCardinality}
Let \( D \) be a minimum \( k \)-dominating set of a graph~\( G \) and let \( G \) have a shortest path of eccentricity at most~\( k \) starting at a vertex~\( s \).
Then, for all non-negative integers \( i \leq \ecc(s) \),
\[
    \left| D \cap \bigcup_{l=i-k}^{i+k} L_l^{(s)} \right| \leq 6k + 1.
\]
\end{lemma}

\begin{proof}
Let $P = \{ s = v_0, v_1, \ldots, v_j \}$ be a shortest path with $j \leq \ecc(s)$  and $\ecc(P) \leq k$.
Also, let $D_{i} = D \cap \bigcup_{l=i-k}^{i+k} L_l^{(s)}$ be a set of $k$-dominating vertices in the layers $L_{i-k}^{(s)}$ to~$L_{i+k}^{(s)}$.
Because $D$ is $k$-dominating, $D_{i}$ can only $k$-dominate vertices in the layers $L_{i-2k}^{(s)}$ to ~$L_{i+2k}^{(s)}$.
By Lemma~\ref{lem:localLayerDom}, these layers are also $k$-dominated by $P_{i} = \{ v_{i-3k}, \ldots, v_{i+3k} \}$.
Thus,
\[
    N^k[D_{i}] \subseteq \bigcup_{l=i-2k}^{i+2k} L_l^{(s)} \subseteq N^k[P_{i}].
\]
Assume that $|D_{i}| > |P_{i}|$.
Note that $|P_{i}| \leq 6k + 1$.
Then, there is a $k$-dominating set $D' = (D \setminus D_{i}) \cup P_{i}$ such that $|D| > |D'|$.
Thus, $D$ is not a minimum $k$-dominating set.
\end{proof}

Based on Lemma~\ref{lem:domCardinality}, Algorithm~\ref{algo:kDomSet} below computes a minimum $k$-dominating set for a given graph~$G = (V, E)$ with a shortest path of eccentricity~$k$ starting at a vertex~$s$ as follows.
In the $i$-th iteration, the algorithm knows all vertex sets~$S$ for which there is a vertex set~$S'$ such that
\begin{enumerate*}[(i),mode=unboxed]
    \item
        $S = S' \cap \left ( L_{i-k}^{(s)} \cup \cdots \cup L_{i-1+k}^{(s)} \right )$,
    \item
        the set~$S^* = S \cup \left ( S' \cap L_{i-1-k}^{(s)} \right )$ $k$-dominates $L_{i-1}^{(s)}$ and has cardinality at most~$6k + 1$, and
    \item
        $S'$ $k$-dominates the layers $L_0^{(s)}$ to~$L_{i-1}^{(s)}$.
\end{enumerate*}
Due to Lemma~\ref{lem:domCardinality}, a set $S^*$ with a larger cardinality cannot be a subset of a minimum dominating set of~$G$ and, hence, neither can be $S$ or~$S'$.
Each such set~$S$ is stored as a pair~$(S, S')$ in a set~$X_{i-1}$ where $S'$ is a corresponding set with minimum cardinality, \ie, $X_{i-1}$ does not contain two pairs $(S, S')$ and~$(T, T')$ with $S = T$.
Note that, since $S'$ has minimum cardinality, it does not contain any vertices from any layer~$L_j^{(s)}$ with $j > i - 1 + k$.
We show later that, this way, $X_{i-1}$ always contains a pair~$(S, S')$ such that $S'$ is subset of some minimum $k$-dominating set.

Then, for each pair~$(S, S') \in X_{i-1}$, the algorithm computes all sets~$S \cup U$ which $k$-dominate the layer~$L_{i}^{(s)}$ and have cardinality at most~$6k + 1$.
For such a set, the sets~$R = (S \cup U) \setminus L_{i-k}^{(s)}$ and~$R' = S' \cup U$ are created and, if the set~$X_i$ does not contain a pair~$(P, P')$ with $P = R$, stored as the pair~$(R, R')$ in~$X_i$.
In the case that $X_i$ already contains such a pair~$(P, P')$, either, if $|R'| < |P'|$, $(P, P')$ is replaced by~$(R, R')$ or, if $|R'| \geq |P'|$, $(R, R')$ is not added to~$X_i$.

Note that $L_{i+k}^{(s)} = \emptyset$ for $i > \ecc(s) - k$.
Therefore, the algorithm can stop after $\ecc(s) - k$ iterations.

\begin{algorithm}
    [htb]
    \caption
    {%
        Determines a minimum $k$-dominating set in a given graph~$G$ containing a shortest path of eccentricity~$k$ starting at~$s$.
    }
    \label{algo:kDomSet}
%

\KwIn
{%
    A graph~$G$, an integer~$k$, and a vertex~$s$ which is start vertex of a shortest path with eccentricity~$k$.
}

\KwOut
{%
    A minimum $k$-dominating set.
}

Compute the layers~$L_0^{(s)}, L_1^{(s)}, \ldots, L_{\ecc(s)}^{(s)}$.

Create the set $X_0 := \Big \{ \, (S, S) \Bigm | S \subseteq N^k[s]; 0 < |S| \leq 6k + 1 \, \Big \}$.
\label{line:CreateX0}

\For
{%
    $i := 1$ \KwTo $\ecc(s) - k$%
    \label{line:layerLoop}%
}
{%
    Create $X_i := \emptyset$.

    \ForEach
    {%
        $(S, S') \in X_{i-1}$
    }
    {%
        \ForEach
        {%
            $U \subseteq L_{i+k}^{(s)}$ with $|S \cup U| \leq 6k + 1$%
            \label{line:Uloop}
        }
        {%
            \If
            {%
                $N^k[S \cup U] \supseteq L_{i}^{(s)}$%
                \label{line:kDomCheck}
            }
            {%
                $R := (S \cup U) \setminus L_{i-k}^{(s)}$

                $R' := S' \cup U$
                \label{line:setRprime}

                \If
                {%
                    There is no pair $(P, P') \in X_i$ with $P = R$
                    \label{line:checkEqualPair}
                }
                {%
                    Insert $(R, R')$ into~$X_i$.
                }

                \If
                {%
                    There is a pair $(P, P') \in X_i$ with $P = R$ and $|R'| < |P'|$
                }
                {%
                    Replace $(P, P')$ in~$X_i$ by $(R, R')$.
                    \label{line:replacePair}
                }
            }
        }
    }
}

Among all pairs~$(S, S') \in X_{\ecc(s) - k}$ for which $S'$ $k$-dominates $G$, determine one with minimum $|S'|$, say $(D, D')$.

Output $D'$.
\end{algorithm}

\begin{theorem}
    \label{theo:kDomFromkESP}
For a given graph~\( G \) and a vertex~\( s \) which is start vertex of a shortest path with eccentricity~\( k \), Algorithm~\ref{algo:kDomSet} determines a minimum \( k \)-dominating set in \( n^{\calO(k)} \) time.
\end{theorem}

\begin{proof}
    [Correctness]
%
To prove the correctness, we show by induction that, for each $i$ with $0 \leq i \leq \ecc(s) - k$, there is a minimum $k$-dominating set~$D$ and a pair~$(S, S') \in X_i$ such that $S' = D \cap \bigcup_{l=0}^{i+k} L_l^{(s)}$.
If this is true for $i = \ecc(s) - k$, then $S' = D$.
Hence, if $(S, S')$ is a pair in $X_{\ecc(s) - k}$ such that $S'$ $k$-dominates $G$ and has minimum cardinality, then $S'$ is a minimum $k$-dominating set of~$G$.

By construction in line~\ref{line:CreateX0}, $X_0$ contains all pairs $(S, S')$ such that $S'$ is a vertex set with cardinality at most~$6k + 1$ which $k$-dominates $L_0^{(s)}$.
Thus, the base case is true.
Next, by induction hypothesis and by definition of the pairs~$(S, S')$, there is a minimum dominating set~$D$ and a pair~$(S, S') \in X_{i-1}$ such that $S = S' \cap \bigcup_{l=i-k}^{i+k-1} L_l^{(s)} = D \cap \bigcup_{l=i-k}^{i+k-1} L_l^{(s)}$.
Let $M = D \cap L_{i+k}^{(s)}$.
By Lemma~\ref{lem:domCardinality}, $\left | D \cap \bigcup_{l=i-k}^{i+k} L_l^{(s)} \right | = |S \cup M| \leq 6k + 1$.
Therefore, there is an iteration of the loop starting in line~\ref{line:Uloop} with $U = M$.
Because $S \cup M = D \cap \bigcup_{l=i-k}^{i+k} L_l^{(s)}$, $S \cup M$ $k$-dominates $L_i^{(s)}$, \ie, $N^k[S \cup M] \supseteq L_{i}^{(s)}$.
Thus, the algorithm creates a pair~$(R, R')$ with $R' = D \cap \bigcup_{l=0}^{i+k} L_l^{(s)}$ (see line~\ref{line:kDomCheck} to line~\ref{line:setRprime}).

Assume that there is a pair~$(P, P') \neq (R, R')$ with $P = R$ and $|P'| \leq |R'|$, \ie, $(R, R')$ will not be stored in~$X_i$ or replaced by $(P, P')$ (see line~\ref{line:checkEqualPair} to line~\ref{line:replacePair}).
Because $P = R$, $P' \cap \bigcup_{l=i-k+1}^{i+k} L_l^{(s)} = D \cap \bigcup_{l=i-k}^{i+k} L_l^{(s)}$.
Let $D' = P' \cup \left ( D \cap \bigcup_{l=i-k+1}^{\ecc(s)-k} L_l^{(s)} \right )$.
By definition, $P'$ $k$-dominates $\bigcup_{l=0}^{i} L_l^{(s)}$.
Thus, $D'$ $k$-dominates $\bigcup_{l=0}^{i} L_l^{(s)}$.
Note that $D' \supseteq D \cap \left ( \bigcup_{l=i-k+1}^{\ecc(s)} L_l^{(s)} \right )$.
Thus, $D'$ also $k$-dominates $\bigcup_{l=i+1}^{\ecc(s)} L_l^{(s)}$.
Therefore, $D'$ is a minimum $k$-dominating set and there is a pair~$(P, P') \in X_i$ such that $P' = D' \cap \bigcup_{l=0}^{i+k} L_l^{(s)}$.
\end{proof}

\begin{proof}
    [Complexity]
%
For a given $i$, there are no two pairs $(S, S')$ and $(T, T')$ in~$X_i$ with $S = T$ (see line~\ref{line:checkEqualPair} to line~\ref{line:replacePair}).
Thus, for each set $U \subseteq L_{i+k}^{(s)}$, $S \cup U \neq T \cup U$.
Additionally, since $S$ and~$T$ intersect at most~$2k$ consecutive layers, $S \neq T$ for all pairs $(S, S') \in X_i$ and~$(T, T') \in X_j$ with $|i - j| \geq 2k$.
Therefore, a set~$S \cup U$ is processed at most $\calO(k)$ times by the loop starting in line~\ref{line:Uloop}.
Hence, because there are at most $n^{6k+1}$ sets~$S \cup U$ with $|S \cup U| \leq 6k + 1$, the loop starting in line~\ref{line:Uloop} has at most $\calO \big( n^{6k+1} k \big)$ iterations.

Next, we show that a single iteration of the loop starting in line~\ref{line:Uloop} requires at most $\calO(m)$ time.
It takes at most $\calO(m)$ time to check if $N^k[S \cup U] \supseteq L_{i}^{(s)}$ (line~\ref{line:kDomCheck}) and at most $\calO(n)$ time to construct $(R, R')$.
Determining if $X_i$ contains a pair~$(P, P')$ with $P = R$ and (if necessary) replacing it can be achieved in $\calO(n)$ time as follows.
One way is to use a tree-structure similar to the one we used for Algorithm~\ref{algo:kDSP}.
An other (less memory efficient) option is to use an array~$A_i$ of size~$n^{6k + 1}$ where each element can store a pair~$(S, S')$.
To determine the index of a pair, assume that each vertex of~$G$ has a unique identifier in the range from $0$ to~$n-1$.
Additionally, assume that the vertices in a set~$S$ are ordered by their identifier.
Therefore, each set~$S$ can be represented by a unique $(6k+1)$-digit number with base~$n$.
This number is the index of a pair~$(S, S')$ in~$A_i$.
Hence, it takes at most $\calO(n)$ time to add $(R, R')$ to~$X_i$ and (if necessary) replace a pair~$(P, P')$.

Therefore, the total runtime of the algorithm is $\calO \big( n^{6k+1} km \big)$.
\end{proof}

If the start vertex~$s$ is unknown, one can use Algorithm~\ref{algo:kDSP} to, first, find a shortest path with eccentricity~$k$ and, then, use Algorithm~\ref{algo:kDomSet} to find a minimum $k$-dominating set.

\section{Open Questions}

\subsubsection{Finding Start and End Vertex.}
The Minimum Eccentricity Shortest Path problem can be naturally split into two subproblems.
First, find the start and end vertices of an optimal path.
Second, for a given vertex pair, find a shortest path between them with the minimum eccentricity.
We know that the second subproblem remains NP-hard (Corollary~\ref{cor:stMespNPc}).
However, is it possible to determine the start and end vertices of an optimal path efficiently?

\subsubsection{APX-Complete?}
In Section~\ref{sec:mespGeneralAppr}, we show multiple algorithms which compute a constant factor approximation in linear or polynomial time.
Therefore, MESP is in APX.
It remains an open question if the problem is APX-complete.

\subsubsection{Complexity for Planar Graphs.}
We show that MESP is NP-complete for planar graphs (Corollary~\ref{cor:mespNPc_planarBipMaxDeg}) and remains W[2]-hard for sparse graphs (Corollary~\ref{cor:mespW2Sparse}).
Additionally, we show that the problem is Fixed Parameter Tractable for graphs with bounded degree (Corollary~\ref{cor:FPTBoundDegree}).
Since every induced subgraph of a planar graphs has an average degree less than~$6$, we conjecture that there is a Fixed Parameter Tractable algorithm to solve MESP on planar graphs.

\subsubsection{Computing Projection Gap.}
In Section~\ref{sec:mespProjGapApproach}, we show that we can solve MESP in polynomial time, if the projection gap of a given graph bounded by a constant.
Additionally, we determine an upper bound for the projection gap of some graph classes.
It would be interesting to know how to compute the projection gap for a given graph.
