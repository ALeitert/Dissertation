% -----------------------------
% Example for layering partition.

\tikzsetnextfilename{fig_prel_LayPartExG}
\begin{tikzpicture}
    [scale=1.2]

\node[nN] at (0,0) {};
\node[nN] at (0,1) {};
\node[nN] at (1,0) {};
\node[nN] at (1,1) {};
\node[nN] at (2,0) {};
\node[nN] at (2,1) {};
\node[nN] at (2,2) {};
\node[nN] (s) at ($(2,2) + (120:1)$) {};
\node[llbl] at (s.west) {$s$};
\node[nN] (b) at ($(2,0) + (-60:1)$) {};
\node[nN] (t) at ($(2,2) + (60:1)$) {};
\node[nN] at (3,0) {};
\node[nN] at (3,1) {};
\node[nN] at (3,2) {};
\node[nN] at (4,0) {};
\node[nN] at (4,1) {};

\begin{pgfonlayer}
    {background}
    \draw 
        (0,0) -- (1,0) -- (2,0) -- (b.center) -- (3,0) -- (4,0) -- (4,1) -- (3,1) -- (3,2) -- (t.center) -- (s.center) -- (2,2) -- (2,1) -- (1,1) -- (0,1) -- cycle;
    \draw 
        (2,0) -- (2,1) 
        (2,2) -- (t.center) 
        (3,0) -- (3,1);
\end{pgfonlayer}

\end{tikzpicture}
